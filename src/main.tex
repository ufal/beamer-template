\documentclass[handout]{beamer}
%
% Choose how your presentation looks.
%
% For more themes, color themes and font themes, see:
% http://deic.uab.es/~iblanes/beamer_gallery/index_by_theme.html
%
\mode<presentation>
{
  \usetheme{default}      % or try Darmstadt, Madrid, Warsaw, ...
  \usecolortheme{default} % or try albatross, beaver, crane, ...
  \usefonttheme{default}  % or try serif, structurebold, ...
  \setbeamertemplate{navigation symbols}{}
  \setbeamertemplate{caption}[numbered]
}

\usepackage[sort&compress,round,comma,authoryear]{natbib}
\usepackage[english]{babel}
%\usepackage[utf8]{inputenc}
\usepackage{fontspec}
\usepackage{todonotes}
\newcommand{\td}[1]{\todo[inline]{#1}}
\usepackage{multirow}
\usepackage{listings}
\usepackage{multicol}


%\usepackage{gnuplot-lua-tikz}
%\usepackage{amsmath}
%\usepackage{amssymb}


\usepackage{ufalslides}

% %%%%%%%%%%%%%%%%%%%%%%%%%%%%%%%%%%%%%%%%%%%%%%%%%%%%%%%%%%%%%%%%%%%%%%%%%%%%%
\def\course{NPFL000 Name of the course}
\def\title{\LaTeX~template for LangTech courses taught at ÚFAL}
\def\author{Jindřich Libovický}
\def\date{September 7, 2018}
% %%%%%%%%%%%%%%%%%%%%%%%%%%%%%%%%%%%%%%%%%%%%%%%%%%%%%%%%%%%%%%%%%%%%%%%%%%%%%


\definecolor{color1}{rgb}{.945098,.5843137,.58431372}
\definecolor{color2}{rgb}{.70235294,.8050980,.40431372}
\definecolor{color3}{rgb}{.40862745,.745098,.5098039215}
\definecolor{color4}{rgb}{.56862745,.498039215,.8}
% #F19595, #E1F195, #95F1B5, #917FCC

\begin{document}

\maketitle

% Uncomment these lines for an automatically generated outline.
%\begin{frame}{Outline} \tableofcontents \end{frame}


\begin{frame}[fragile]
    \frametitle{Content of the title page}

1. Define the content of the title page
    \begin{lstlisting}[language=TeX]
\def\course{NPFL116 Compendium of Neural Machine Translation}
\def\title{Attention Mechanism}
\def\author{Jindřich Libovický, Jindřich Helcl}
\def\date{March 1, 2017}
\end{lstlisting}

2. Generate the title slide using
\begin{lstlisting}[language=TeX]
\maketitle
\end{lstlisting}

    \vspace{5pt}

    Hint: Don't use \textbf{ř} in your code snippets, it will break.

\end{frame}

% -----------------------------------------------------------------------------

\begin{frame}[fragile]
    \frametitle{Equations}

    \centering Sans serif font also for equations

    $$i\hbar\frac{\partial}{\partial t} \Psi(\mathbf{r},t) = \left [ \frac{-\hbar^2}{2\mu}\nabla^2 + V(\mathbf{r},t)\right ] \Psi(\mathbf{r},t) ]$$

    \begin{lstlisting}[language=TeX]
$$ i\hbar\frac{\partial}{\partial t} \Psi(\mathbf{r},t) = \left [ \frac{-\hbar^2}{2\mu}\nabla^2 + V(\mathbf{r},t)\right ] \Psi(\mathbf{r},t) ]$$
    \end{lstlisting}

\end{frame}


% -----------------------------------------------------------------------------

\begin{frame}[fragile]
    \frametitle{Lables from the web}

    \slidesbox{Slides}
    \readingbox{Reading}
    \hwbox{Homework}
    \questionbox{Question}
    \timebox{1 h}
    \calendarbox{Oct 15}
    \pointsbox{100 points}

    \begin{lstlisting}[language=TeX]
\slidesbox{Slides}
\readingbox{Reading}
\hwbox{Homework}
\questionbox{Question}
\timebox{1 h}
\calendarbox{Oct 15}
\pointsbox{100 points}
\slidesbox{Slides}
    \end{lstlisting}
\end{frame}


\begin{frame}[fragile]
    \frametitle{Code listings}

    This code snippet:
    \begin{lstlisting}[language=Python]
print("Hello, ÚFAL".)
x = 3 + 5
    \end{lstlisting}

    is produced with this code:
    \begin{lstlisting}[language=TeX]
\begin{lstlisting}[language=Python]
print("Hello, ÚFAL".)
x = 3 + 5
\ end{lstlisting}
    \end{lstlisting}

    Inline code (\lstinline[language=Python]{import numpy as np}) can be
    inserted with the \lstinline[language=TeX]{\lstinline} command.

    Do not forget to start the frame with \lstinline[language=TeX]{fragile}
    option to beginning of the frame.

\end{frame}


\begin{frame}[fragile]
    \frametitle{References}

    Full citation on slide: \\ {\tiny \bibentry{helcl2017neural}}

\begin{lstlisting}[language=TeX]
Full citation: \\ {\tiny \bibentry{helcl2017neural}}
\end{lstlisting}

    \citet{sennrich2016neural} uses attention \citep{bahdanau2015neural}.

    \begin{lstlisting}[language=TeX]
\citet{sennrich2016neural} uses attention \citep{bahdanau2015neural}.
    \end{lstlisting}

    \tiny
    \citet{snover2006study,lu2016knowing,tu2016modeling,feng2016improving,zhang2016recurrent,alkhouli2016alignment,graves2014neural,specia2016shared,elliott2016multi30k}

\end{frame}


\begin{frame}
    \frametitle{Itemize}

    \begin{itemize}[<+->]

        \item All human beings are born free and equal in dignity and rights.

        \item They are endowed with reason and conscience and should act
            towards one another in a spirit of brotherhood.

        \item Everyone is entitled to all the rights and freedoms set forth in
            this Declaration, without distinction of any kind, such as race,
            colour, sex, language, religion, political or other opinion,
            national or social origin, property, birth or other status.

    \end{itemize}

\end{frame}

\begin{frame}
    \frametitle{Enumerate}

    \begin{enumerate}[<+->]

        \item All human beings are born free and equal in dignity and rights.

        \item They are endowed with reason and conscience and should act
            towards one another in a spirit of brotherhood.

        \item Everyone is entitled to all the rights and freedoms set forth in
            this Declaration, without distinction of any kind, such as race,
            colour, sex, language, religion, political or other opinion,
            national or social origin, property, birth or other status.

    \end{enumerate}

\end{frame}

\begin{frame}[allowframebreaks]
    \frametitle{What happens with too much content?}

    Babakotia, an extinct genus of sloth lemurs, lived in the northern part of
    Madagascar. The name comes from the Malagasy word for the indri, to which
    all sloth lemurs are closely related. Its morphological traits show
    intermediate stages between the slow-moving smaller sloth lemurs and the
    suspensory large sloth lemurs, and suggest a close relationship between
    both groups and the extinct monkey lemurs. All sloth lemurs share many
    traits with living sloths, demonstrating convergent evolution. Babakotia
    had long forearms, curved digits, and highly mobile hip and ankle joints.
    It shared its range with other sloth lemurs, including Palaeopropithecus
    ingens and Mesopropithecus dolichobrachion. It was primarily a leaf-eater,
    though it also ate fruit and hard seeds. It is known only from subfossil
    remains and may have died out shortly after the arrival of humans on the
    island, but not enough radiocarbon dating has been done with this species
    to know for certain. Babakotia radofilai is the sole member of the genus
    Babakotia and belongs to the family Palaeopropithecidae, which includes
    three other genera of sloth lemurs: Palaeopropithecus, Archaeoindris, and
    Mesopropithecus. This family in turn belongs to the infraorder
    Lemuriformes, which includes all the Malagasy lemurs.[5][2]

\end{frame}

{
\setbeamercolor{background canvas}{bg=lightgray}
\begin{frame}[plain]
 \centering

    \colorbox{langtech}{{\bf\tiny\color{lightgray} \scalebox{0.9}{\title}}}

    \vspace{5pt}{\color{langtech}\bfseries\Large Summary}\vspace{5pt}

    {\setlength{\fboxrule}{1.5pt}
    \fcolorbox{langtech}{white}{
        \begin{minipage}{0.7\paperwidth}
            \vspace{3pt}
            \begin{enumerate}

                \item This template is tremendous.

                \item If you don't use the template you will be very very sad.

                \item Believe me. It's tremendous.

            \end{enumerate}
        \end{minipage}
    }}


    \begin{textblock*}{\paperwidth}(0pt,\paperheight - 25pt)
        \centering
        \Large\bf \url{ufal.mff.cuni.cz/courses/npfl000}
    \end{textblock*}
\end{frame}}


\begin{frame}[allowframebreaks]
    \frametitle{References}
    \tiny
    \bibliography{references.bib}
    \bibliographystyle{plainnat}
\end{frame}


\end{document}
