\documentclass[handout,aspectratio=169]{beamer}

\usepackage{polyglossia}
\setmainlanguage{english}
\usepackage[nounicodemath]{ufalslides}
% Supported options:
%   nounicodemath (if you are struggling with a strange compilation issue)
%   custombib (if natbib chucks for you)
%   miktex    (if you are compiling with miktex; implies nounicodemath)
\usepackage{xcolor}
\usepackage{textcomp}

% %%%%%%%%%%%%%%%%%%%%%%%%%%%%%%%%%%%%%%%%%%%%%%%%%%%%%%%%%%%%%%%%%%%%%%%%%%%%%
\def\course{NPFL000 Name of the course}
\def\courseurl{https://ufal.cz/courses/npfl000}
\def\title{\LaTeX~template for LangTech courses taught at ÚFAL}
\def\author{Jindřich Libovický}
\def\date{September 7, 2018}
\def\licence{cc-by-nc-sa}
\def\langtech{}
\def\shownavigation{}
% %%%%%%%%%%%%%%%%%%%%%%%%%%%%%%%%%%%%%%%%%%%%%%%%%%%%%%%%%%%%%%%%%%%%%%%%%%%%%

\begin{document}

\maketitle

\outline{Outline}

\section{A section}
\outlinecurrent{Outline}

\begin{frame}[fragile]
    \frametitle{How to use the template}

    \begin{lstlisting}[language=TeX]
\documentclass[handout,aspectratio=169]{beamer}
\usepackage[english]{babel}
\usepackage{ufalslides}
\end{lstlisting}

    \begin{itemize}
        \item Use \lstinline{handout} option if you want to generate a handout without
    animations.

        \item Before you begin document, define what you want to
            appear in the title slide (see the next slide for more info).
    \end{itemize}

\end{frame}


\begin{frame}[fragile]
    \frametitle{Content of the title page}

1. Define the content of the title page
    \begin{lstlisting}[language=TeX]
\def\course{NPFL116 Compendium of Neural Machine Translation}
\def\courseurl{https://ufal.cz/courses/npfl000}
\def\title{Attention Mechanism}
\def\subtitle{How to attend with a mechanism}
\def\author{Jindřich Libovický, Jindřich Helcl} \def\date{March 1, 2017}
\def\licence{cc-by-nc-sa}
\def\langtech{} % shows the LangTech and the EU logo
\def\shownavigation{} % shows the navigation links in the bottom line
\end{lstlisting}

    {\tt  \textbackslash course} and {\tt  \textbackslash subtitle} are optional, others must be at least
    an empty string

2. Generate the title slide using after beginning of the document by calling
\begin{lstlisting}[language=TeX]
\maketitle
\end{lstlisting}

    \vspace{5pt}

    Hint: Don't use \textbf{ř} in your code snippets, it will break.

\end{frame}

% -----------------------------------------------------------------------------

\begin{frame}[fragile]
    \frametitle{Licence}

    \begin{columns}
        \column{0.25\textwidth}
        \centering

        %LaTeX with PSTricks extensions
%%Creator: inkscape 0.91
%%Please note this file requires PSTricks extensions
\psset{xunit=.5pt,yunit=.5pt,runit=.5pt}
\begin{pspicture}(120,42)
{
\newrgbcolor{curcolor}{0.66666669 0.69803923 0.67058825}
\pscustom[linestyle=none,fillstyle=solid,fillcolor=curcolor]
{
\newpath
\moveto(3.4083297,41.54731536)
\lineto(116.76243,41.34595815)
\curveto(118.34626,41.34595815)(119.76124,41.58079209)(119.76124,38.185433)
\lineto(119.62246,0.856356)
\lineto(0.54733418,0.856356)
\lineto(0.54733418,38.324199)
\curveto(0.54733418,39.9985919)(0.70939995,41.54731536)(3.4083297,41.54731536)
\closepath
}
}
{
\newrgbcolor{curcolor}{0 0 0}
\pscustom[linestyle=none,fillstyle=solid,fillcolor=curcolor]
{
\newpath
\moveto(117.7533,42.00000142)
\lineto(2.2476335,42.00000142)
\curveto(1.0083193,42.00000142)(-0.00002041,40.9917746)(-0.00002041,39.7530859)
\lineto(-0.00002041,0.507022)
\curveto(-0.00002041,0.227065)(0.22707838,0.000002)(0.5075432,0.000002)
\lineto(119.49241,0.000002)
\curveto(119.77289,0.000002)(119.99998,0.227075)(119.99998,0.507022)
\lineto(119.99998,39.7530859)
\curveto(119.99998,40.9917746)(118.99164,42.00000142)(117.7533,42.00000142)
\closepath
\moveto(2.2476335,40.9849779)
\lineto(117.75331,40.9849779)
\curveto(118.43264,40.9849779)(118.98486,40.4323485)(118.98486,39.753076)
\lineto(118.98486,12.532451)
\lineto(36.428267,12.532451)
\curveto(33.402284,7.061445)(27.571583,3.34683)(20.881033,3.34683)
\curveto(14.188536,3.34683)(8.3597829,7.058047)(5.3357378,12.532451)
\lineto(1.0151167,12.532451)
\lineto(1.0151167,39.753076)
\curveto(1.0151068,40.4323485)(1.5682849,40.9849779)(2.2476335,40.9849779)
\closepath
}
}
{
\newrgbcolor{curcolor}{1 1 1}
\pscustom[linestyle=none,fillstyle=solid,fillcolor=curcolor]
{
\newpath
\moveto(34.52218482,22.44874988)
\curveto(34.5269994,14.91907517)(28.42554577,8.81199781)(20.89550614,8.80715775)
\curveto(13.36551857,8.80228299)(7.25722909,14.90308908)(7.25241451,22.43228672)
\lineto(7.25241451,22.44874988)
\curveto(7.24753921,29.97842459)(13.34900152,36.08550194)(20.87904114,36.09036803)
\curveto(28.41003501,36.094705)(34.51827243,29.99446272)(34.52218482,22.46526508)
\lineto(34.52218482,22.44874988)
\closepath
}
}
{
\newrgbcolor{curcolor}{0 0 0}
\pscustom[linestyle=none,fillstyle=solid,fillcolor=curcolor]
{
\newpath
\moveto(31.97172335,33.55331543)
\curveto(34.99481187,30.53008787)(36.5067812,26.82808895)(36.5067812,22.44875855)
\curveto(36.5067812,18.06895108)(35.02096676,14.40624516)(32.04933787,11.46064947)
\curveto(28.89525804,8.35882724)(25.16859873,6.80816767)(20.86735602,6.80816767)
\curveto(16.61852719,6.80816767)(12.95594085,8.34574692)(9.88148812,11.42231928)
\curveto(6.80502281,14.49746043)(5.26774874,18.1727696)(5.26774874,22.44875855)
\curveto(5.26774874,26.72379337)(6.80502281,30.42530654)(9.88148812,33.55331543)
\curveto(12.87831765,36.57749712)(16.54091267,38.08934943)(20.86735602,38.08934943)
\curveto(25.24717616,38.08934943)(28.94768059,36.57749712)(31.97172335,33.55331543)
\closepath
\moveto(11.91658121,31.51940403)
\curveto(9.36027284,28.9377066)(8.08211866,25.91352491)(8.08211866,22.44489864)
\curveto(8.08211866,18.97627238)(9.34766818,15.97776558)(11.87676332,13.44895323)
\curveto(14.40775827,10.91919543)(17.41920505,9.65430785)(20.91394902,9.65430785)
\curveto(24.408693,9.65430785)(27.44629466,10.93181602)(30.02875792,13.48728342)
\curveto(32.48022987,15.86086675)(33.70596151,18.84576411)(33.70596151,22.44489864)
\curveto(33.70596151,26.01638929)(32.45989585,29.04830815)(29.96860602,31.53883368)
\curveto(27.47837452,34.02878672)(24.46016131,35.27428802)(20.91394902,35.27428802)
\curveto(17.36773674,35.27428802)(14.36699482,34.02248076)(11.91658121,31.51940403)
\closepath
\moveto(18.64303688,23.97029967)
\curveto(18.25291701,24.82181276)(17.66869578,25.24783387)(16.88835193,25.24783387)
\curveto(15.51023663,25.24783387)(14.82117899,24.31966039)(14.82117899,22.46426758)
\curveto(14.82117899,20.60844974)(15.51023663,19.6812304)(16.88835193,19.6812304)
\curveto(17.79873284,19.6812304)(18.44892685,20.1329264)(18.83904672,21.03833942)
\lineto(20.74992362,20.02085895)
\curveto(19.83869257,18.40275074)(18.47222769,17.59301139)(16.6506244,17.59301139)
\curveto(15.2452872,17.59301139)(14.11952535,18.02384654)(13.27426707,18.88504844)
\curveto(12.42806323,19.7467274)(12.00490058,20.93499797)(12.00490058,22.44875855)
\curveto(12.00490058,23.93636717)(12.44161346,25.11733427)(13.31313072,25.99208488)
\curveto(14.18454388,26.86688754)(15.27155486,27.30408069)(16.57300121,27.30408069)
\curveto(18.49933681,27.30408069)(19.87745211,26.54521406)(20.71211831,25.02902478)
\lineto(18.64303688,23.97029967)
\closepath
\moveto(27.63362958,23.97029967)
\curveto(27.24255547,24.82181276)(26.66892632,25.24783387)(25.91389589,25.24783387)
\curveto(24.50771722,25.24783387)(23.80404233,24.31966039)(23.80404233,22.46426758)
\curveto(23.80404233,20.60844974)(24.50771722,19.6812304)(25.91389589,19.6812304)
\curveto(26.82618528,19.6812304)(27.46472887,20.1329264)(27.82869385,21.03833942)
\lineto(29.78224269,20.02085895)
\curveto(28.87292012,18.40275074)(27.50846782,17.59301139)(25.68971858,17.59301139)
\curveto(24.28639396,17.59301139)(23.16359026,18.02384654)(22.31823656,18.88504844)
\curveto(21.47488677,19.7467274)(21.0517328,20.93499797)(21.0517328,22.44875855)
\curveto(21.0517328,23.93636717)(21.48071631,25.11733427)(22.33867468,25.99208488)
\curveto(23.1955747,26.86688754)(24.28640264,27.30408069)(25.61210407,27.30408069)
\curveto(27.53463138,27.30408069)(28.91179244,26.54521406)(29.74253758,25.02902478)
\lineto(27.63362958,23.97029967)
\closepath
}
}
{
\newrgbcolor{curcolor}{1 1 1}
\pscustom[linestyle=none,fillstyle=solid,fillcolor=curcolor]
{
\newpath
\moveto(62.50238153,26.77618464)
\curveto(62.50238153,20.84589283)(57.69439216,16.03844146)(51.7634367,16.03844146)
\curveto(45.83248125,16.03844146)(41.02449188,20.84589283)(41.02449188,26.77618464)
\curveto(41.02449188,32.70647644)(45.83248125,37.51392782)(51.7634367,37.51392782)
\curveto(57.69439216,37.51392782)(62.50238153,32.70647644)(62.50238153,26.77618464)
\closepath
}
}
{
\newrgbcolor{curcolor}{0 0 0}
\pscustom[linestyle=none,fillstyle=solid,fillcolor=curcolor]
{
\newpath
\moveto(54.87142746,29.88334337)
\curveto(54.87142746,30.29720675)(54.53563886,30.63248081)(54.12220618,30.63248081)
\lineto(49.37942706,30.63248081)
\curveto(48.96599437,30.63248081)(48.63020578,30.29721669)(48.63020578,29.88334337)
\lineto(48.63020578,25.14060805)
\lineto(49.95297771,25.14060805)
\lineto(49.95297771,19.52452658)
\lineto(53.54767168,19.52452658)
\lineto(53.54767168,25.14060805)
\lineto(54.87141752,25.14060805)
\lineto(54.87141752,29.88334337)
\lineto(54.87142746,29.88334337)
\closepath
}
}
{
\newrgbcolor{curcolor}{0 0 0}
\pscustom[linestyle=none,fillstyle=solid,fillcolor=curcolor]
{
\newpath
\moveto(53.37298867,32.87890741)
\curveto(53.37298867,31.98310883)(52.64671917,31.2569206)(51.75082034,31.2569206)
\curveto(50.85492152,31.2569206)(50.12865202,31.98310883)(50.12865202,32.87890741)
\curveto(50.12865202,33.77470598)(50.85492152,34.50089421)(51.75082034,34.50089421)
\curveto(52.64671917,34.50089421)(53.37298867,33.77470598)(53.37298867,32.87890741)
\closepath
}
}
{
\newrgbcolor{curcolor}{0 0 0}
\pscustom[linestyle=none,fillstyle=solid,fillcolor=curcolor]
{
\newpath
\moveto(51.73578071,38.60754385)
\curveto(48.52442775,38.60754385)(45.80512547,37.48675409)(43.57979187,35.24517459)
\curveto(41.29623266,32.92644658)(40.15493504,30.18173241)(40.15493504,27.01294987)
\curveto(40.15493504,23.84416732)(41.29623266,21.11885952)(43.57979187,18.83798038)
\curveto(45.86335109,16.558552)(48.5826633,15.41787892)(51.73578071,15.41787892)
\curveto(54.92868911,15.41787892)(57.69652764,16.5672963)(60.03831247,18.8675818)
\curveto(62.24423752,21.0509422)(63.34670813,23.76606489)(63.34670813,27.0129598)
\curveto(63.34670813,30.25985471)(62.22481905,33.00360502)(59.98007691,35.24518453)
\curveto(57.73533478,37.48676403)(54.98788863,38.60754385)(51.73578071,38.60754385)
\closepath
\moveto(51.76586246,36.52123515)
\curveto(54.39783128,36.52123515)(56.63189027,35.59306825)(58.46999718,33.73769831)
\curveto(60.32848653,31.90222164)(61.25724425,29.66015524)(61.25724425,27.0129598)
\curveto(61.25724425,24.346358)(60.34789506,22.13340612)(58.52823273,20.37459105)
\curveto(56.61249167,18.48089528)(54.3580403,17.53428587)(51.76586246,17.53428587)
\curveto(49.17174675,17.53428587)(46.93670391,18.47118712)(45.05977993,20.34596343)
\curveto(43.18285595,22.22025284)(42.24438899,24.44242598)(42.24438899,27.0129598)
\curveto(42.24438899,29.58349363)(43.19255525,31.82459617)(45.08888777,33.73769831)
\curveto(46.90856998,35.59305832)(49.13389364,36.52123515)(51.76586246,36.52123515)
\closepath
}
}
{
\newrgbcolor{curcolor}{1 1 1}
\pscustom[linestyle=none,fillstyle=solid,fillcolor=curcolor]
{
\newpath
\moveto(48.40117905,9.25256646)
\curveto(48.7165852,9.25256646)(49.00482139,9.22442574)(49.2649137,9.16911809)
\curveto(49.52500601,9.11381044)(49.74821911,9.02259157)(49.93455301,8.89547142)
\curveto(50.11991301,8.76932506)(50.26354415,8.60047078)(50.36642034,8.39087606)
\curveto(50.46832262,8.18030753)(50.51976071,7.92120818)(50.51976071,7.61166023)
\curveto(50.51976071,7.27784686)(50.44406443,7.00032488)(50.29169796,6.77810061)
\curveto(50.14030539,6.55491248)(49.91514447,6.37344854)(49.6181829,6.23080727)
\curveto(50.0277299,6.11339527)(50.33343675,5.90766592)(50.53529351,5.61364902)
\curveto(50.73715026,5.31963212)(50.83808858,4.96543859)(50.83808858,4.55107837)
\curveto(50.83808858,4.217265)(50.7730655,3.92809721)(50.64301935,3.68355512)
\curveto(50.51297319,3.43998683)(50.3373125,3.24009032)(50.11701117,3.08580324)
\curveto(49.89768375,2.93054236)(49.64632677,2.81604182)(49.36487811,2.74229166)
\curveto(49.08246548,2.6675677)(48.79325539,2.63070256)(48.49531991,2.63070256)
\lineto(45.27911729,2.63070256)
\lineto(45.27911729,9.2525764)
\lineto(48.40117905,9.2525764)
\lineto(48.40117905,9.25256646)
\closepath
\moveto(48.21484515,6.57431885)
\curveto(48.47493746,6.57431885)(48.68844133,6.6364232)(48.85536669,6.7596581)
\curveto(49.02326596,6.88289301)(49.10672367,7.08279946)(49.10672367,7.36032144)
\curveto(49.10672367,7.51460852)(49.0785798,7.64172867)(49.02326596,7.73974424)
\curveto(48.96697821,7.83872367)(48.8932198,7.91538528)(48.80005285,7.97166672)
\curveto(48.7068859,8.02697437)(48.60013397,8.06579705)(48.47978711,8.08714107)
\curveto(48.35847629,8.10848509)(48.23327979,8.11916704)(48.10323363,8.11916704)
\lineto(46.7387229,8.11916704)
\lineto(46.7387229,6.57431885)
\lineto(48.21484515,6.57431885)
\closepath
\moveto(48.30025067,3.76410198)
\curveto(48.44290791,3.76410198)(48.57877762,3.77768544)(48.70882377,3.80582617)
\curveto(48.83886993,3.83396689)(48.95338329,3.88055012)(49.0533477,3.94459213)
\curveto(49.15233821,4.00960793)(49.23094627,4.09791535)(49.29014579,4.20950444)
\curveto(49.34837141,4.32012968)(49.37846309,4.46277095)(49.37846309,4.63550054)
\curveto(49.37846309,4.97513681)(49.28238436,5.2177313)(49.09022691,5.36328402)
\curveto(48.89806945,5.50787289)(48.6438007,5.58064925)(48.32839455,5.58064925)
\lineto(46.73873284,5.58064925)
\lineto(46.73873284,3.76410198)
\lineto(48.30025067,3.76410198)
\closepath
}
}
{
\newrgbcolor{curcolor}{1 1 1}
\pscustom[linestyle=none,fillstyle=solid,fillcolor=curcolor]
{
\newpath
\moveto(51.09525917,9.25256646)
\lineto(52.72762364,9.25256646)
\lineto(54.27846827,6.63738706)
\lineto(55.82057756,9.25256646)
\lineto(57.4442067,9.25256646)
\lineto(54.98498679,5.17211193)
\lineto(54.98498679,2.63068269)
\lineto(53.52537124,2.63068269)
\lineto(53.52537124,5.20898701)
\lineto(51.09525917,9.25256646)
\closepath
}
}
{
\newrgbcolor{curcolor}{1 1 1}
\pscustom[linestyle=none,fillstyle=solid,fillcolor=curcolor]
{
\newpath
\moveto(98.44056654,4.26772354)
\curveto(98.52111247,4.11343647)(98.62690044,3.98825397)(98.75985837,3.89218599)
\curveto(98.8928163,3.79611802)(99.04809455,3.72527931)(99.22763099,3.67870602)
\curveto(99.40620346,3.63212278)(99.59058955,3.60884111)(99.78274701,3.60884111)
\curveto(99.91181926,3.60884111)(100.05060075,3.61951312)(100.19909147,3.64183094)
\curveto(100.3466083,3.66317496)(100.48538979,3.70490908)(100.61543594,3.76701343)
\curveto(100.74450819,3.82814399)(100.85223403,3.91353995)(100.93957742,4.02222753)
\curveto(101.02498295,4.12994132)(101.06864967,4.26773348)(101.06864967,4.43464016)
\curveto(101.06864967,4.61415651)(101.01138803,4.75874538)(100.89687466,4.87034441)
\curveto(100.78332527,4.98193351)(100.63290661,5.07412617)(100.44851058,5.14884019)
\curveto(100.26315059,5.22259035)(100.05254856,5.28760615)(99.81866225,5.34388759)
\curveto(99.58380203,5.39919524)(99.34603004,5.46129959)(99.1053562,5.52922685)
\curveto(98.85788491,5.5913312)(98.61720114,5.66701901)(98.38234092,5.75630023)
\curveto(98.14845461,5.8455715)(97.93785258,5.96201964)(97.75249259,6.10467085)
\curveto(97.56809656,6.24634826)(97.4176779,6.42392696)(97.3041285,6.63740693)
\curveto(97.18961514,6.85088691)(97.13235349,7.10901246)(97.13235349,7.41177365)
\curveto(97.13235349,7.75237378)(97.205138,8.04737441)(97.34974304,8.29772947)
\curveto(97.49531205,8.54808452)(97.68553164,8.75671539)(97.91941795,8.92459587)
\curveto(98.15427816,9.09150255)(98.42019403,9.21473746)(98.71715561,9.29528432)
\curveto(99.01315322,9.37582124)(99.31012474,9.41560777)(99.60612235,9.41560777)
\curveto(99.95258415,9.41560777)(100.284497,9.37679503)(100.60281492,9.29915963)
\curveto(100.92016889,9.22249802)(101.20354548,9.09635166)(101.45005281,8.92362208)
\curveto(101.6975241,8.74992863)(101.89356724,8.52867816)(102.03913625,8.25988059)
\curveto(102.1837413,7.99011917)(102.2565258,7.66406636)(102.2565258,7.28076825)
\lineto(100.84348876,7.28076825)
\curveto(100.83086774,7.47872711)(100.79011279,7.64272234)(100.71926616,7.77178015)
\curveto(100.64841953,7.90181175)(100.55427867,8.00370263)(100.43685353,8.07841665)
\curveto(100.32039236,8.15216681)(100.18647046,8.2055368)(100.03604186,8.23561518)
\curveto(99.88561326,8.26666736)(99.72063571,8.28219841)(99.54206324,8.28219841)
\curveto(99.42560206,8.28219841)(99.30817693,8.26957881)(99.19171575,8.24532333)
\curveto(99.07429061,8.22009406)(98.96850265,8.17739608)(98.87339783,8.11529173)
\curveto(98.7773191,8.05318738)(98.69968494,7.97652577)(98.63757364,7.88336924)
\curveto(98.57643625,7.79021272)(98.5453806,7.67280072)(98.5453806,7.53112331)
\curveto(98.5453806,7.4010917)(98.56963879,7.29532551)(98.61913901,7.21575245)
\curveto(98.66862929,7.13521553)(98.76568192,7.06049158)(98.91126087,6.99256432)
\curveto(99.05683982,6.92463707)(99.25675871,6.85670981)(99.51296533,6.78878256)
\curveto(99.76917196,6.72085531)(100.10496056,6.63352168)(100.51839325,6.52871935)
\curveto(100.64164194,6.50446388)(100.81342689,6.45885444)(101.03178041,6.39383864)
\curveto(101.25110783,6.32882284)(101.46947128,6.22595816)(101.68492296,6.08331689)
\curveto(101.90133861,5.94067562)(102.08767251,5.75047732)(102.24586253,5.51273193)
\curveto(102.40307866,5.27498654)(102.48168672,4.9702877)(102.48168672,4.59863539)
\curveto(102.48168672,4.2958742)(102.4224872,4.01446697)(102.30506206,3.75440377)
\curveto(102.18763692,3.49434056)(102.01294019,3.27018857)(101.78100169,3.08192793)
\curveto(101.54905325,2.89270343)(101.26179097,2.74617691)(100.91824095,2.64137457)
\curveto(100.5756549,2.53559844)(100.17872891,2.48320225)(99.72647914,2.48320225)
\curveto(99.36157277,2.48320225)(99.00831351,2.52881168)(98.66476348,2.61808296)
\curveto(98.32121346,2.70832803)(98.01841839,2.84903165)(97.75638821,3.0411676)
\curveto(97.49338412,3.23330355)(97.2847299,3.4788095)(97.12945165,3.77574779)
\curveto(96.97514731,4.07268608)(96.90041499,4.42493201)(96.90721245,4.83345939)
\lineto(98.32024949,4.83345939)
\curveto(98.32021968,4.61123512)(98.36001066,4.42201062)(98.44056654,4.26772354)
\closepath
}
}
{
\newrgbcolor{curcolor}{1 1 1}
\pscustom[linestyle=none,fillstyle=solid,fillcolor=curcolor]
{
\newpath
\moveto(106.64604566,9.25256646)
\lineto(109.12273623,2.63069262)
\lineto(107.61071862,2.63069262)
\lineto(107.10994254,4.10567591)
\lineto(104.63325196,4.10567591)
\lineto(104.11306734,2.63069262)
\lineto(102.64762824,2.63069262)
\lineto(105.15246268,9.25256646)
\lineto(106.64604566,9.25256646)
\closepath
\moveto(106.72951331,5.19250203)
\lineto(105.89489644,7.62039458)
\lineto(105.8754879,7.62039458)
\lineto(105.01369113,5.19250203)
\lineto(106.72951331,5.19250203)
\closepath
}
}
{
\newrgbcolor{curcolor}{1 1 1}
\pscustom[linestyle=none,fillstyle=solid,fillcolor=curcolor]
{
\newpath
\moveto(72.51783412,9.25256646)
\lineto(75.28469875,4.81016778)
\lineto(75.3002216,4.81016778)
\lineto(75.3002216,9.25256646)
\lineto(76.6666702,9.25256646)
\lineto(76.6666702,2.63069262)
\lineto(75.20996643,2.63069262)
\lineto(72.45377501,7.06435695)
\lineto(72.43534037,7.06435695)
\lineto(72.43534037,2.63069262)
\lineto(71.06889177,2.63069262)
\lineto(71.06889177,9.25256646)
\lineto(72.51783412,9.25256646)
\closepath
}
}
{
\newrgbcolor{curcolor}{1 1 1}
\pscustom[linestyle=none,fillstyle=solid,fillcolor=curcolor]
{
\newpath
\moveto(82.19554981,7.48841539)
\curveto(82.10918032,7.62814521)(82.00048057,7.75041625)(81.87043442,7.85521858)
\curveto(81.74038826,7.96002092)(81.59384535,8.04250543)(81.42983176,8.10072453)
\curveto(81.26581818,8.15991743)(81.09404317,8.18903195)(80.9154707,8.18903195)
\curveto(80.58744353,8.18903195)(80.30891659,8.1259538)(80.07987993,7.99883365)
\curveto(79.85084326,7.8726873)(79.66548327,7.70286916)(79.52379,7.48937925)
\curveto(79.38113277,7.27589927)(79.27728268,7.03330477)(79.2122596,6.76159576)
\curveto(79.14723652,6.48988674)(79.11520696,6.20847951)(79.11520696,5.91833793)
\curveto(79.11520696,5.63984215)(79.14723652,5.36910693)(79.2122596,5.10709613)
\curveto(79.27728268,4.84412147)(79.38112283,4.60734988)(79.52379,4.39774521)
\curveto(79.66548327,4.18717669)(79.85084326,4.01929621)(80.07987993,3.89217606)
\curveto(80.30891659,3.7650559)(80.58744353,3.70197776)(80.9154707,3.70197776)
\curveto(81.35995904,3.70197776)(81.70835871,3.83783227)(81.95874178,4.11050514)
\curveto(82.20912485,4.38221416)(82.36246523,4.74125679)(82.41777907,5.18665925)
\lineto(83.82790434,5.18665925)
\curveto(83.79102513,4.77230897)(83.6949464,4.39773527)(83.54064206,4.06393185)
\curveto(83.38633771,3.72915462)(83.18155924,3.444826)(82.92826439,3.20902821)
\curveto(82.67496954,2.97323041)(82.37799803,2.79370412)(82.03735978,2.66949542)
\curveto(81.6976855,2.54528672)(81.32307983,2.48318237)(80.9154707,2.48318237)
\curveto(80.40790709,2.48318237)(79.95177164,2.57148979)(79.54611032,2.74809469)
\curveto(79.14141296,2.92373573)(78.79883685,3.16730402)(78.52029996,3.47587818)
\curveto(78.24177302,3.78542613)(78.02729525,4.14931787)(77.87880452,4.56658955)
\curveto(77.73032373,4.98482508)(77.65559141,5.43411279)(77.65559141,5.91736413)
\curveto(77.65559141,6.41226127)(77.73032373,6.8712472)(77.87880452,7.29530564)
\curveto(78.02728531,7.71936408)(78.24176308,8.08907872)(78.52029996,8.40444958)
\curveto(78.79883685,8.71982044)(79.14141296,8.96727398)(79.54611032,9.14679033)
\curveto(79.95177164,9.32630668)(80.40790709,9.41558789)(80.9154707,9.41558789)
\curveto(81.28037707,9.41558789)(81.624901,9.3631917)(81.94904248,9.25741557)
\curveto(82.27415787,9.15261323)(82.56530583,8.99929002)(82.82151246,8.79745585)
\curveto(83.07869299,8.59658554)(83.29025899,8.34720428)(83.45718435,8.05026599)
\curveto(83.62410971,7.7533277)(83.72892377,7.41272757)(83.7725905,7.02942946)
\lineto(82.36246523,7.02942946)
\curveto(82.33724307,7.19633615)(82.2819193,7.34964943)(82.19554981,7.48841539)
\closepath
}
}
{
\newrgbcolor{curcolor}{1 1 1}
\pscustom[linestyle=none,fillstyle=solid,fillcolor=curcolor]
{
\newpath
\moveto(114.57975107,27.01925344)
\curveto(114.58460193,21.17416329)(109.84857889,16.43191951)(104.00241227,16.42755982)
\curveto(98.15617111,16.42415032)(93.41438419,21.15816534)(93.4094588,27.0046963)
\lineto(93.4094588,27.01925344)
\curveto(93.4055893,32.8643436)(98.14161854,37.60658737)(103.98687214,37.61094706)
\curveto(109.8331133,37.61530675)(114.57490022,32.87985093)(114.57975107,27.0352514)
\lineto(114.57975107,27.01925344)
\closepath
}
}
{
\newrgbcolor{curcolor}{0 0 0}
\pscustom[linestyle=none,fillstyle=solid,fillcolor=curcolor]
{
\newpath
\moveto(103.91889187,38.61434034)
\curveto(100.70659583,38.61434034)(97.98727942,37.49306775)(95.76290534,35.25245399)
\curveto(93.47939556,32.93322298)(92.33710963,30.18851228)(92.33710963,27.01925965)
\curveto(92.33710963,23.85000703)(93.47939556,21.12564776)(95.76290534,18.84477829)
\curveto(98.04550829,16.56436218)(100.76482471,15.42417897)(103.91889187,15.42417897)
\curveto(107.11185282,15.42417897)(109.87870884,16.57456585)(112.22044749,18.87340196)
\curveto(114.42639951,21.05772478)(115.52986619,23.77287405)(115.52986619,27.01926586)
\curveto(115.52986619,30.26569494)(114.40797744,33.01040564)(112.16221862,35.2524602)
\curveto(109.91752809,37.49306775)(107.17007549,38.61434034)(103.91889187,38.61434034)
\closepath
\moveto(103.94800321,36.52705478)
\curveto(106.57997941,36.52705478)(108.8140552,35.59985672)(110.65311872,33.74398253)
\curveto(112.51067884,31.90854051)(113.43938436,29.66792675)(113.43938436,27.01925965)
\curveto(113.43938436,24.35266938)(112.53008846,22.14018247)(110.71036003,20.38039535)
\curveto(108.79563313,18.48718433)(106.5401726,17.54057881)(103.94800321,17.54057881)
\curveto(101.35484625,17.54057881)(99.11985743,18.47797433)(97.24288769,20.35226852)
\curveto(95.36599248,22.22702228)(94.4275045,24.4487192)(94.4275045,27.01925965)
\curveto(94.4275045,29.58980011)(95.37470663,31.83135784)(97.27199902,33.74398253)
\curveto(99.0907461,35.59985672)(101.316027,36.52705478)(103.94800321,36.52705478)
\closepath
}
}
{
\newrgbcolor{curcolor}{0 0 0}
\pscustom[linestyle=none,fillstyle=solid,fillcolor=curcolor]
{
\newpath
\moveto(98.77920454,28.65532348)
\curveto(99.24116604,31.57326666)(101.294723,33.13264465)(103.86945168,33.13264465)
\curveto(107.57085164,33.13264465)(109.82721899,30.44664693)(109.82721899,26.86498748)
\curveto(109.82721899,23.36967723)(107.4262639,20.6555092)(103.81114206,20.6555092)
\curveto(101.32382812,20.6555092)(99.09657209,22.18532575)(98.69185191,25.18961192)
\lineto(101.61299737,25.18961192)
\curveto(101.70033757,23.63019046)(102.71260076,23.08098748)(104.15861472,23.08098748)
\curveto(105.80653846,23.08098748)(106.87694978,24.61174802)(106.87694978,26.95133667)
\curveto(106.87694978,29.40736384)(105.95211376,30.70622239)(104.21586224,30.70622239)
\curveto(102.94354735,30.70622239)(101.84493152,30.24385302)(101.61300358,28.65532348)
\lineto(102.46217623,28.66116124)
\lineto(100.16312012,26.36133768)
\lineto(97.86398948,28.66116124)
\lineto(98.77920454,28.65532348)
\closepath
}
}
{
\newrgbcolor{curcolor}{1 1 1}
\pscustom[linestyle=none,fillstyle=solid,fillcolor=curcolor]
{
\newpath
\moveto(87.80189623,27.00915545)
\curveto(87.80672823,21.50856845)(83.35029323,17.04723845)(77.84864523,17.04285145)
\curveto(72.34793123,17.03947845)(67.88566023,21.49496145)(67.88179723,26.99461445)
\lineto(67.88179723,27.00915545)
\curveto(67.87789923,32.51023195)(72.33436923,36.97156095)(77.83504823,36.97593685)
\curveto(83.33572823,36.97934415)(87.79799823,32.52382695)(87.80189623,27.02420845)
\lineto(87.80189623,27.00915545)
\closepath
}
}
{
\newrgbcolor{curcolor}{0 0 0}
\pscustom[linestyle=none,fillstyle=solid,fillcolor=curcolor]
{
\newpath
\moveto(86.07086623,35.24481185)
\curveto(83.82574623,37.48710385)(81.07808623,38.60754385)(77.82718623,38.60754385)
\curveto(74.61450623,38.60754385)(71.89586623,37.48710385)(69.67056623,35.24481185)
\curveto(67.38651623,32.92536985)(66.24554623,30.18124945)(66.24554623,27.01104145)
\curveto(66.24554623,23.84153945)(67.38651623,21.11582145)(69.67056623,18.83601845)
\curveto(71.95390623,16.55479845)(74.67254623,15.41453945)(77.82718623,15.41453945)
\curveto(81.02004623,15.41453945)(83.78751623,16.56470545)(86.12889623,18.86432845)
\curveto(88.33509623,21.04858045)(89.43854623,23.76438645)(89.43854623,27.01104145)
\curveto(89.43854623,30.25840245)(88.31598623,33.00322985)(86.07086623,35.24481185)
\closepath
\moveto(84.61917623,20.37193145)
\curveto(82.70317623,18.47787545)(80.44885623,17.53155345)(77.85620623,17.53155345)
\curveto(75.26284623,17.53155345)(73.02762623,18.46867345)(71.15055623,20.34362145)
\curveto(69.27348623,22.21856945)(68.33495623,24.44104245)(68.33495623,27.01104445)
\curveto(68.33495623,28.09821645)(68.50659623,29.12593545)(68.84598623,30.09490645)
\lineto(71.91886623,28.73487945)
\lineto(71.69732623,28.73487945)
\lineto(71.69732623,27.35680545)
\lineto(72.78485623,27.35680545)
\curveto(72.78485623,27.16216145)(72.76539623,26.96822445)(72.76539623,26.77429145)
\lineto(72.76539623,26.44375145)
\lineto(71.69732623,26.44375145)
\lineto(71.69732623,25.06567745)
\lineto(72.95932623,25.06567745)
\curveto(73.13414623,24.03654545)(73.52237623,23.20205445)(74.04649623,22.54168445)
\curveto(75.13366623,21.10486345)(76.88085623,20.32841245)(78.78341623,20.32841245)
\curveto(80.02629623,20.32841245)(81.15169623,20.69717345)(81.81206623,21.06664145)
\lineto(81.34562623,23.22116645)
\curveto(80.93793623,23.00741445)(80.02629623,22.71650845)(79.13305623,22.71650845)
\curveto(78.16196623,22.71650845)(77.24997623,23.00741045)(76.62853623,23.70671245)
\curveto(76.33727623,24.03654545)(76.12387623,24.48316345)(75.98833623,25.06567745)
\lineto(80.20890623,25.06567745)
\lineto(86.20604623,22.41109345)
\curveto(85.79057623,21.67497945)(85.26327623,20.99479045)(84.61917623,20.37193145)
\closepath
\moveto(77.07692623,26.44374445)
\lineto(77.05144623,26.46320845)
\lineto(77.09532623,26.44374445)
\lineto(77.07692623,26.44374445)
\closepath
\moveto(80.68737623,27.35679845)
\lineto(80.86079623,27.35679845)
\lineto(80.86079623,28.73487245)
\lineto(77.57379623,28.73487245)
\lineto(76.23854623,29.32588045)
\curveto(76.35355623,29.58422545)(76.48945623,29.81532145)(76.64799623,29.99616145)
\curveto(77.24997623,30.73509845)(78.10393623,31.04511245)(79.03609623,31.04511245)
\curveto(79.89040623,31.04511245)(80.68596623,30.79313845)(81.19062623,30.57938245)
\lineto(81.73420623,32.79265485)
\curveto(81.03490623,33.10266985)(80.00647623,33.37446285)(78.82233623,33.37446285)
\curveto(76.99764623,33.37446285)(75.44439623,32.63693985)(74.33775623,31.39476145)
\curveto(74.09215623,31.11093745)(73.87910623,30.79242845)(73.68587623,30.45551745)
\lineto(69.87227623,32.14325485)
\curveto(70.24174623,32.69745885)(70.67633623,33.22971985)(71.17957623,33.73720785)
\curveto(72.99931623,35.59304385)(75.22391623,36.52096285)(77.85619623,36.52096285)
\curveto(80.48777623,36.52096285)(82.72299623,35.59304385)(84.56184623,33.73720785)
\curveto(86.41909623,31.90118845)(87.34772623,29.65960645)(87.34772623,27.01103745)
\curveto(87.34772623,26.13832845)(87.24862623,25.31516145)(87.05327623,24.53942045)
\lineto(80.68737623,27.35679845)
\closepath
}
}
\end{pspicture}
 \\ {\tt cc-by-nc-sa}

        \vspace{3pt}
\definecolor{caab2ab}{RGB}{170,178,171}


\def \globalscale {0.550000}
\begin{tikzpicture}[y=1cm, x=1cm, yscale=\globalscale,xscale=\globalscale, every node/.append style={scale=\globalscale}, inner sep=0pt, outer sep=0pt]
  \begin{scope}[cm={ 0.9938,-0.0,-0.0,0.9937,(-11.5655, 0.007)}]
    \path[fill=caab2ab] (11.7287, 1.0994) -- (14.7466, 1.094).. controls (14.7888, 1.094) and (14.8265, 1.1003) .. (14.8265, 1.0099) -- (14.8228, 0.0159) -- (11.6525, 0.0159) -- (11.6525, 1.0136).. controls (11.6525, 1.0581) and (11.6568, 1.0994) .. (11.7287, 1.0994) -- cycle;



    \path[fill] (14.773, 1.1113) -- (11.6978, 1.1113).. controls (11.6648, 1.1113) and (11.6379, 1.0844) .. (11.6379, 1.0514) -- (11.6379, 0.0064).. controls (11.6379, -0.001) and (11.644, -0.0071) .. (11.6514, -0.0071) -- (14.8193, -0.0071).. controls (14.8268, -0.0071) and (14.8328, -0.001) .. (14.8328, 0.0064) -- (14.8328, 1.0514).. controls (14.8328, 1.0844) and (14.806, 1.1113) .. (14.773, 1.1113) -- cycle(11.6978, 1.0842) -- (14.773, 1.0842).. controls (14.7911, 1.0842) and (14.8058, 1.0695) .. (14.8058, 1.0514).. controls (14.8058, 1.0514) and (14.8058, 0.631) .. (14.8058, 0.3268) -- (12.6078, 0.3268).. controls (12.5273, 0.1811) and (12.372, 0.0822) .. (12.1939, 0.0822).. controls (12.0157, 0.0822) and (11.8605, 0.181) .. (11.78, 0.3268) -- (11.665, 0.3268).. controls (11.665, 0.631) and (11.665, 1.0514) .. (11.665, 1.0514).. controls (11.665, 1.0695) and (11.6797, 1.0842) .. (11.6978, 1.0842) -- cycle;



    \begin{scope}[cm={ 0.8729,-0.0,-0.0,0.8729,(1.3262, -3.648)}]
      \path[fill=white] (12.8658, 4.8559).. controls (12.866, 4.6262) and (12.6799, 4.44) .. (12.4502, 4.4398).. controls (12.2205, 4.4397) and (12.0342, 4.6257) .. (12.0341, 4.8554).. controls (12.0341, 4.8556) and (12.0341, 4.8557) .. (12.0341, 4.8559).. controls (12.0339, 5.0856) and (12.22, 5.2719) .. (12.4497, 5.272).. controls (12.6794, 5.2722) and (12.8657, 5.0861) .. (12.8658, 4.8564).. controls (12.8658, 4.8563) and (12.8658, 4.8561) .. (12.8658, 4.8559) -- cycle;



      \begin{scope}[shift={(-7.6627, -2.6211)}]
        \path[fill] (20.4508, 7.8157).. controls (20.543, 7.7235) and (20.5891, 7.6106) .. (20.5891, 7.477).. controls (20.5891, 7.3434) and (20.5438, 7.2317) .. (20.4531, 7.1418).. controls (20.3569, 7.0472) and (20.2433, 6.9999) .. (20.1121, 6.9999).. controls (19.9825, 6.9999) and (19.8708, 7.0468) .. (19.777, 7.1407).. controls (19.6832, 7.2345) and (19.6363, 7.3466) .. (19.6363, 7.477).. controls (19.6363, 7.6074) and (19.6832, 7.7203) .. (19.777, 7.8157).. controls (19.8684, 7.908) and (19.9801, 7.9541) .. (20.1121, 7.9541).. controls (20.2457, 7.9541) and (20.3585, 7.908) .. (20.4508, 7.8157) -- cycle(19.8391, 7.7537).. controls (19.7611, 7.675) and (19.7222, 7.5827) .. (19.7222, 7.4769).. controls (19.7222, 7.3711) and (19.7607, 7.2796) .. (19.8379, 7.2025).. controls (19.9151, 7.1253) and (20.0069, 7.0868) .. (20.1135, 7.0868).. controls (20.2201, 7.0868) and (20.3128, 7.1257) .. (20.3915, 7.2037).. controls (20.4663, 7.2761) and (20.5037, 7.3671) .. (20.5037, 7.4769).. controls (20.5037, 7.5859) and (20.4657, 7.6783) .. (20.3897, 7.7543).. controls (20.3137, 7.8303) and (20.2217, 7.8682) .. (20.1135, 7.8682).. controls (20.0054, 7.8682) and (19.9138, 7.8301) .. (19.8391, 7.7537) -- cycle(20.0443, 7.5234).. controls (20.0324, 7.5494) and (20.0145, 7.5624) .. (19.9907, 7.5624).. controls (19.9487, 7.5624) and (19.9277, 7.5341) .. (19.9277, 7.4775).. controls (19.9277, 7.4209) and (19.9487, 7.3926) .. (19.9907, 7.3926).. controls (20.0185, 7.3926) and (20.0383, 7.4064) .. (20.0502, 7.434) -- (20.1085, 7.403).. controls (20.0807, 7.3536) and (20.039, 7.3289) .. (19.9835, 7.3289).. controls (19.9406, 7.3289) and (19.9063, 7.3421) .. (19.8805, 7.3683).. controls (19.8547, 7.3946) and (19.8418, 7.4308) .. (19.8418, 7.477).. controls (19.8418, 7.5224) and (19.8551, 7.5584) .. (19.8817, 7.5851).. controls (19.9083, 7.6118) and (19.9414, 7.6251) .. (19.9811, 7.6251).. controls (20.0399, 7.6251) and (20.0819, 7.602) .. (20.1074, 7.5557) -- (20.0443, 7.5234) -- cycle(20.3185, 7.5234).. controls (20.3065, 7.5494) and (20.289, 7.5624) .. (20.266, 7.5624).. controls (20.2231, 7.5624) and (20.2017, 7.5341) .. (20.2017, 7.4775).. controls (20.2017, 7.4209) and (20.2231, 7.3926) .. (20.266, 7.3926).. controls (20.2938, 7.3926) and (20.3133, 7.4064) .. (20.3244, 7.434) -- (20.384, 7.403).. controls (20.3563, 7.3536) and (20.3146, 7.3289) .. (20.2592, 7.3289).. controls (20.2164, 7.3289) and (20.1821, 7.3421) .. (20.1563, 7.3683).. controls (20.1306, 7.3946) and (20.1177, 7.4308) .. (20.1177, 7.477).. controls (20.1177, 7.5224) and (20.1308, 7.5584) .. (20.157, 7.5851).. controls (20.1831, 7.6118) and (20.2164, 7.6251) .. (20.2568, 7.6251).. controls (20.3154, 7.6251) and (20.3574, 7.602) .. (20.3828, 7.5557) -- (20.3185, 7.5234) -- cycle;



      \end{scope}
    \end{scope}
    \path[fill=white] (12.9184, 0.2393).. controls (12.9268, 0.2393) and (12.9345, 0.2385) .. (12.9414, 0.2371).. controls (12.9483, 0.2356) and (12.9542, 0.2332) .. (12.9592, 0.2298).. controls (12.9641, 0.2264) and (12.968, 0.2219) .. (12.9707, 0.2163).. controls (12.9734, 0.2107) and (12.9748, 0.2038) .. (12.9748, 0.1956).. controls (12.9748, 0.1867) and (12.9728, 0.1793) .. (12.9687, 0.1734).. controls (12.9647, 0.1675) and (12.9587, 0.1626) .. (12.9508, 0.1588).. controls (12.9617, 0.1557) and (12.9698, 0.1502) .. (12.9752, 0.1424).. controls (12.9806, 0.1346) and (12.9833, 0.1251) .. (12.9833, 0.1141).. controls (12.9833, 0.1052) and (12.9815, 0.0975) .. (12.9781, 0.091).. controls (12.9746, 0.0845) and (12.9699, 0.0792) .. (12.9641, 0.0751).. controls (12.9582, 0.071) and (12.9515, 0.0679) .. (12.944, 0.0659).. controls (12.9365, 0.0639) and (12.9288, 0.063) .. (12.9209, 0.063) -- (12.8353, 0.063) -- (12.8353, 0.2393) -- (12.9184, 0.2393) -- (12.9184, 0.2393) -- cycle(12.9134, 0.168).. controls (12.9203, 0.168) and (12.926, 0.1696) .. (12.9305, 0.1729).. controls (12.9349, 0.1762) and (12.9372, 0.1815) .. (12.9372, 0.1889).. controls (12.9372, 0.193) and (12.9364, 0.1964) .. (12.9349, 0.199).. controls (12.9334, 0.2016) and (12.9315, 0.2037) .. (12.929, 0.2052).. controls (12.9265, 0.2067) and (12.9237, 0.2077) .. (12.9205, 0.2083).. controls (12.9173, 0.2088) and (12.9139, 0.2091) .. (12.9104, 0.2091) -- (12.8741, 0.2091) -- (12.8741, 0.168) -- (12.9134, 0.168) -- cycle(12.9157, 0.0931).. controls (12.9195, 0.0931) and (12.9231, 0.0935) .. (12.9266, 0.0943).. controls (12.93, 0.095) and (12.9331, 0.0962) .. (12.9357, 0.098).. controls (12.9384, 0.0997) and (12.9405, 0.102) .. (12.942, 0.105).. controls (12.9436, 0.108) and (12.9444, 0.1118) .. (12.9444, 0.1163).. controls (12.9444, 0.1254) and (12.9418, 0.1319) .. (12.9367, 0.1357).. controls (12.9316, 0.1396) and (12.9249, 0.1415) .. (12.9164, 0.1415) -- (12.8741, 0.1415) -- (12.8741, 0.0931) -- (12.9157, 0.0931) -- cycle;



    \path[fill=white] (12.9901, 0.2393) -- (13.0336, 0.2393) -- (13.0749, 0.1697) -- (13.1159, 0.2393) -- (13.1591, 0.2393) -- (13.0937, 0.1306) -- (13.0937, 0.063) -- (13.0548, 0.063) -- (13.0548, 0.1316) -- (12.9901, 0.2393) -- cycle;



    \path[fill=white] (13.5686, 0.2393) -- (13.6423, 0.121) -- (13.6427, 0.121) -- (13.6427, 0.2393) -- (13.6791, 0.2393) -- (13.6791, 0.063) -- (13.6403, 0.063) -- (13.5669, 0.181) -- (13.5664, 0.181) -- (13.5664, 0.063) -- (13.5301, 0.063) -- (13.5301, 0.2393) -- (13.5686, 0.2393) -- cycle;



    \path[fill=white] (13.8263, 0.1923).. controls (13.824, 0.196) and (13.8211, 0.1993) .. (13.8176, 0.2021).. controls (13.8142, 0.2049) and (13.8103, 0.2071) .. (13.8059, 0.2086).. controls (13.8015, 0.2102) and (13.797, 0.211) .. (13.7922, 0.211).. controls (13.7835, 0.211) and (13.7761, 0.2093) .. (13.77, 0.2059).. controls (13.7639, 0.2025) and (13.7589, 0.198) .. (13.7552, 0.1923).. controls (13.7514, 0.1867) and (13.7486, 0.1802) .. (13.7469, 0.173).. controls (13.7451, 0.1657) and (13.7443, 0.1582) .. (13.7443, 0.1505).. controls (13.7443, 0.1431) and (13.7451, 0.1359) .. (13.7469, 0.1289).. controls (13.7486, 0.1219) and (13.7514, 0.1156) .. (13.7552, 0.11).. controls (13.7589, 0.1044) and (13.7639, 0.0999) .. (13.77, 0.0966).. controls (13.7761, 0.0932) and (13.7835, 0.0915) .. (13.7922, 0.0915).. controls (13.8041, 0.0915) and (13.8133, 0.0951) .. (13.82, 0.1024).. controls (13.8267, 0.1096) and (13.8307, 0.1192) .. (13.8322, 0.131) -- (13.8698, 0.131).. controls (13.8688, 0.12) and (13.8662, 0.11) .. (13.8621, 0.1011).. controls (13.858, 0.0922) and (13.8526, 0.0846) .. (13.8458, 0.0784).. controls (13.8391, 0.0721) and (13.8312, 0.0673) .. (13.8221, 0.064).. controls (13.813, 0.0607) and (13.8031, 0.059) .. (13.7922, 0.059).. controls (13.7787, 0.059) and (13.7666, 0.0614) .. (13.7558, 0.0661).. controls (13.745, 0.0708) and (13.7359, 0.0773) .. (13.7285, 0.0855).. controls (13.721, 0.0937) and (13.7153, 0.1034) .. (13.7114, 0.1145).. controls (13.7074, 0.1257) and (13.7054, 0.1376) .. (13.7054, 0.1505).. controls (13.7054, 0.1637) and (13.7074, 0.1759) .. (13.7114, 0.1872).. controls (13.7153, 0.1985) and (13.721, 0.2083) .. (13.7285, 0.2167).. controls (13.7359, 0.2251) and (13.745, 0.2317) .. (13.7558, 0.2365).. controls (13.7666, 0.2413) and (13.7787, 0.2436) .. (13.7922, 0.2436).. controls (13.8019, 0.2436) and (13.8111, 0.2422) .. (13.8198, 0.2394).. controls (13.8284, 0.2366) and (13.8361, 0.2325) .. (13.843, 0.2272).. controls (13.8498, 0.2218) and (13.8554, 0.2152) .. (13.8599, 0.2073).. controls (13.8643, 0.1994) and (13.8671, 0.1903) .. (13.8683, 0.1801) -- (13.8307, 0.1801).. controls (13.8301, 0.1845) and (13.8286, 0.1886) .. (13.8263, 0.1923) -- cycle;



    \path[fill=white] (14.2566, 0.2393) -- (14.3302, 0.121) -- (14.3306, 0.121) -- (14.3306, 0.2393) -- (14.367, 0.2393) -- (14.367, 0.063) -- (14.3282, 0.063) -- (14.2549, 0.181) -- (14.2544, 0.181) -- (14.2544, 0.063) -- (14.218, 0.063) -- (14.218, 0.2393) -- (14.2566, 0.2393) -- cycle;



    \path[fill=white] (14.4771, 0.2393).. controls (14.4885, 0.2393) and (14.4991, 0.2375) .. (14.5089, 0.2339).. controls (14.5187, 0.2302) and (14.5272, 0.2248) .. (14.5344, 0.2176).. controls (14.5416, 0.2103) and (14.5472, 0.2013) .. (14.5512, 0.1904).. controls (14.5552, 0.1795) and (14.5573, 0.1668) .. (14.5573, 0.1521).. controls (14.5573, 0.1393) and (14.5556, 0.1274) .. (14.5523, 0.1166).. controls (14.549, 0.1057) and (14.544, 0.0963) .. (14.5374, 0.0884).. controls (14.5307, 0.0805) and (14.5224, 0.0743) .. (14.5124, 0.0698).. controls (14.5024, 0.0652) and (14.4907, 0.063) .. (14.4771, 0.063) -- (14.401, 0.063) -- (14.401, 0.2393) -- (14.4771, 0.2393) -- (14.4771, 0.2393) -- cycle(14.4744, 0.0956).. controls (14.48, 0.0956) and (14.4855, 0.0965) .. (14.4907, 0.0983).. controls (14.496, 0.1001) and (14.5007, 0.1031) .. (14.5048, 0.1074).. controls (14.5089, 0.1115) and (14.5122, 0.117) .. (14.5147, 0.1237).. controls (14.5172, 0.1305) and (14.5184, 0.1387) .. (14.5184, 0.1484).. controls (14.5184, 0.1573) and (14.5176, 0.1653) .. (14.5158, 0.1725).. controls (14.5141, 0.1796) and (14.5112, 0.1858) .. (14.5073, 0.1908).. controls (14.5033, 0.1959) and (14.4981, 0.1999) .. (14.4916, 0.2026).. controls (14.4851, 0.2053) and (14.4771, 0.2066) .. (14.4675, 0.2066) -- (14.4399, 0.2066) -- (14.4399, 0.0956) -- (14.4744, 0.0956) -- (14.4744, 0.0956) -- cycle;



    \begin{scope}[shift={(7.571, -5.5047)}]
      \begin{scope}[cm={ 1.1468,-0.0,-0.0,1.1468,(-1.7764, 0.9454)}]
        \path[fill=white] (7.1336, 4.5967).. controls (7.1338, 4.469) and (7.0303, 4.3654) .. (6.9026, 4.3653).. controls (6.7749, 4.3653) and (6.6713, 4.4687) .. (6.6712, 4.5964).. controls (6.6712, 4.5965) and (6.6712, 4.5966) .. (6.6712, 4.5967).. controls (6.6711, 4.7245) and (6.7746, 4.828) .. (6.9023, 4.8281).. controls (7.03, 4.8282) and (7.1336, 4.7248) .. (7.1336, 4.5971).. controls (7.1336, 4.597) and (7.1336, 4.5968) .. (7.1336, 4.5967) -- cycle;



        \path[fill] (6.9021, 4.8659).. controls (6.9776, 4.8659) and (7.0413, 4.8399) .. (7.0935, 4.7879).. controls (7.1456, 4.7358) and (7.1716, 4.6721) .. (7.1716, 4.5967).. controls (7.1716, 4.5214) and (7.146, 4.4583) .. (7.0948, 4.4076).. controls (7.0404, 4.3542) and (6.9762, 4.3275) .. (6.9021, 4.3275).. controls (6.8289, 4.3275) and (6.7657, 4.354) .. (6.7127, 4.407).. controls (6.6597, 4.4599) and (6.6332, 4.5231) .. (6.6332, 4.5967).. controls (6.6332, 4.6703) and (6.6597, 4.7341) .. (6.7127, 4.7879).. controls (6.7644, 4.8399) and (6.8275, 4.8659) .. (6.9021, 4.8659) -- cycle(6.6938, 4.6691).. controls (6.6857, 4.6463) and (6.6817, 4.6222) .. (6.6817, 4.5967).. controls (6.6817, 4.5371) and (6.7035, 4.4855) .. (6.7471, 4.4419).. controls (6.7907, 4.3984) and (6.8426, 4.3767) .. (6.9028, 4.3767).. controls (6.9629, 4.3767) and (7.0153, 4.3986) .. (7.0597, 4.4426).. controls (7.0746, 4.457) and (7.0869, 4.4727) .. (7.0965, 4.4897) -- (6.995, 4.5349).. controls (6.9882, 4.5008) and (6.9577, 4.4777) .. (6.921, 4.475) -- (6.921, 4.4335) -- (6.8901, 4.4335) -- (6.8901, 4.475).. controls (6.8598, 4.4753) and (6.8307, 4.4877) .. (6.8083, 4.5072) -- (6.8454, 4.5446).. controls (6.8633, 4.5278) and (6.8811, 4.5203) .. (6.9055, 4.5203).. controls (6.9213, 4.5203) and (6.9388, 4.5264) .. (6.9388, 4.547).. controls (6.9388, 4.5543) and (6.936, 4.5594) .. (6.9316, 4.5632) -- (6.9059, 4.5746) -- (6.8739, 4.5888).. controls (6.8581, 4.5959) and (6.8447, 4.6018) .. (6.8313, 4.6078) -- (6.6938, 4.6691) -- cycle(6.9028, 4.8175).. controls (6.8416, 4.8175) and (6.79, 4.796) .. (6.7478, 4.7529).. controls (6.7363, 4.7413) and (6.7263, 4.7292) .. (6.7178, 4.7165) -- (6.8207, 4.6707).. controls (6.83, 4.6992) and (6.8571, 4.7166) .. (6.8901, 4.7185) -- (6.8901, 4.76) -- (6.921, 4.76) -- (6.921, 4.7185).. controls (6.9423, 4.7175) and (6.9656, 4.7116) .. (6.9886, 4.6938) -- (6.9532, 4.6574).. controls (6.9402, 4.6667) and (6.9237, 4.6732) .. (6.9072, 4.6732).. controls (6.8938, 4.6732) and (6.8749, 4.6691) .. (6.8749, 4.6523).. controls (6.8749, 4.6497) and (6.8758, 4.6475) .. (6.8774, 4.6455) -- (6.9118, 4.6302) -- (6.9351, 4.6198).. controls (6.95, 4.6131) and (6.9642, 4.6068) .. (6.9783, 4.6005) -- (7.1163, 4.5391).. controls (7.1208, 4.5572) and (7.1231, 4.5764) .. (7.1231, 4.5967).. controls (7.1231, 4.6582) and (7.1015, 4.7103) .. (7.0584, 4.7529).. controls (7.0157, 4.796) and (6.9639, 4.8175) .. (6.9028, 4.8175) -- cycle;



      \end{scope}
    \end{scope}
    \begin{scope}[cm={ 0.625,-0.0,-0.0,0.625,(8.2775, -7.9688)}]
      \path[fill=white] (10.2615, 13.8899).. controls (10.2617, 13.6409) and (10.0599, 13.4389) .. (9.8109, 13.4387).. controls (9.5619, 13.4385) and (9.3598, 13.6402) .. (9.3597, 13.8893).. controls (9.3597, 13.8895) and (9.3597, 13.8897) .. (9.3597, 13.8899).. controls (9.3595, 14.1389) and (9.5612, 14.341) .. (9.8102, 14.3412).. controls (10.0592, 14.3414) and (10.2613, 14.1396) .. (10.2615, 13.8906).. controls (10.2615, 13.8904) and (10.2615, 13.8902) .. (10.2615, 13.8899) -- cycle;



      \begin{scope}[shift={(-0.6337, 2.3262)}]
        \path[fill] (10.4372, 12.0577).. controls (10.3004, 12.0577) and (10.1846, 12.01) .. (10.0898, 11.9144).. controls (9.9925, 11.8157) and (9.9439, 11.6988) .. (9.9439, 11.5637).. controls (9.9439, 11.4287) and (9.9925, 11.3126) .. (10.0898, 11.2154).. controls (10.187, 11.1183) and (10.3029, 11.0697) .. (10.4372, 11.0697).. controls (10.5733, 11.0697) and (10.6911, 11.1187) .. (10.7909, 11.2166).. controls (10.8848, 11.3097) and (10.9319, 11.4254) .. (10.9319, 11.5637).. controls (10.9319, 11.7021) and (10.884, 11.8189) .. (10.7884, 11.9144).. controls (10.6928, 12.01) and (10.5757, 12.0577) .. (10.4372, 12.0577) -- cycle(10.4384, 11.9688).. controls (10.5506, 11.9688) and (10.6458, 11.9293) .. (10.7241, 11.8502).. controls (10.8032, 11.772) and (10.8428, 11.6765) .. (10.8428, 11.5637).. controls (10.8428, 11.4501) and (10.8041, 11.3558) .. (10.7266, 11.2809).. controls (10.6449, 11.2002) and (10.5489, 11.1599) .. (10.4384, 11.1599).. controls (10.328, 11.1599) and (10.2328, 11.1998) .. (10.1528, 11.2797).. controls (10.0729, 11.3595) and (10.0329, 11.4542) .. (10.0329, 11.5637).. controls (10.0329, 11.6732) and (10.0733, 11.7687) .. (10.1541, 11.8502).. controls (10.2316, 11.9293) and (10.3263, 11.9688) .. (10.4384, 11.9688) -- cycle;



        \path[fill] (10.6245, 11.6808) -- (10.2654, 11.6808) -- (10.2654, 11.5957) -- (10.6245, 11.5957) -- (10.6245, 11.6808) -- cycle(10.6245, 11.522) -- (10.2654, 11.522) -- (10.2654, 11.437) -- (10.6245, 11.437) -- (10.6245, 11.522) -- cycle;



      \end{scope}
    \end{scope}
    \path[fill=white] (13.0161, 0.7061) circle (0.2859cm);



    \path[fill] (13.0989, 0.7888).. controls (13.0989, 0.7998) and (13.0899, 0.8087) .. (13.0789, 0.8087) -- (12.9526, 0.8087).. controls (12.9416, 0.8087) and (12.9327, 0.7998) .. (12.9327, 0.7888) -- (12.9327, 0.6625) -- (12.9679, 0.6625) -- (12.9679, 0.513) -- (13.0636, 0.513) -- (13.0636, 0.6625) -- (13.0989, 0.6625) -- (13.0989, 0.7888) -- (13.0989, 0.7888) -- cycle;



    \path[fill] (13.0158, 0.8686) circle (0.0432cm);



    \path[fill,even odd rule] (13.0154, 1.0211).. controls (12.9299, 1.0211) and (12.8575, 0.9913) .. (12.7982, 0.9316).. controls (12.7374, 0.8698) and (12.707, 0.7967) .. (12.707, 0.7124).. controls (12.707, 0.628) and (12.7374, 0.5554) .. (12.7982, 0.4947).. controls (12.859, 0.434) and (12.9314, 0.4036) .. (13.0154, 0.4036).. controls (13.1004, 0.4036) and (13.1741, 0.4342) .. (13.2364, 0.4955).. controls (13.2951, 0.5536) and (13.3245, 0.6259) .. (13.3245, 0.7124).. controls (13.3245, 0.7988) and (13.2946, 0.8719) .. (13.2349, 0.9316).. controls (13.1751, 0.9913) and (13.102, 1.0211) .. (13.0154, 1.0211) -- cycle(13.0162, 0.9655).. controls (13.0862, 0.9655) and (13.1457, 0.9408) .. (13.1947, 0.8914).. controls (13.2441, 0.8426) and (13.2689, 0.7829) .. (13.2689, 0.7124).. controls (13.2689, 0.6414) and (13.2447, 0.5824) .. (13.1962, 0.5356).. controls (13.1452, 0.4852) and (13.0852, 0.46) .. (13.0162, 0.46).. controls (12.9471, 0.46) and (12.8876, 0.4849) .. (12.8376, 0.5348).. controls (12.7877, 0.5848) and (12.7627, 0.6439) .. (12.7627, 0.7124).. controls (12.7627, 0.7808) and (12.7879, 0.8405) .. (12.8384, 0.8914).. controls (12.8868, 0.9408) and (12.9461, 0.9655) .. (13.0162, 0.9655) -- cycle;



  \end{scope}

\end{tikzpicture}
 \\ {\tt cc-by-cs-nd}

        \vspace{3pt}
\definecolor{caab2ab}{RGB}{170,178,171}


\def \globalscale {0.550000}
\begin{tikzpicture}[y=1cm, x=1cm, yscale=\globalscale,xscale=\globalscale, every node/.append style={scale=\globalscale}, inner sep=0pt, outer sep=0pt]
  \begin{scope}[cm={ 0.9938,-0.0,-0.0,0.9937,(-4.7015, 3.946)}]
    \path[fill=caab2ab] (4.8216, -2.8645) -- (7.8396, -2.8699).. controls (7.8818, -2.8699) and (7.9194, -2.8636) .. (7.9194, -2.954) -- (7.9157, -3.948) -- (4.7455, -3.948) -- (4.7455, -2.9503).. controls (4.7455, -2.9057) and (4.7498, -2.8645) .. (4.8216, -2.8645) -- cycle;



    \begin{scope}[cm={ 0.8729,-0.0,-0.0,0.8729,(1.3262, -3.648)}]
      \path[fill=white] (4.9533, 0.315).. controls (4.9534, 0.0853) and (4.7673, -0.101) .. (4.5376, -0.1011).. controls (4.308, -0.1012) and (4.1217, 0.0848) .. (4.1215, 0.3145).. controls (4.1215, 0.3147) and (4.1215, 0.3148) .. (4.1215, 0.315).. controls (4.1214, 0.5447) and (4.3075, 0.731) .. (4.5371, 0.7311).. controls (4.7668, 0.7312) and (4.9531, 0.5452) .. (4.9533, 0.3155).. controls (4.9533, 0.3153) and (4.9533, 0.3152) .. (4.9533, 0.315) -- cycle;



      \begin{scope}[shift={(-7.6627, -2.6211)}]
        \path[fill] (12.5382, 3.2748).. controls (12.6304, 3.1826) and (12.6765, 3.0697) .. (12.6765, 2.9361).. controls (12.6765, 2.8025) and (12.6312, 2.6908) .. (12.5406, 2.6009).. controls (12.4444, 2.5063) and (12.3307, 2.459) .. (12.1995, 2.459).. controls (12.07, 2.459) and (11.9582, 2.5059) .. (11.8644, 2.5998).. controls (11.7706, 2.6936) and (11.7237, 2.8057) .. (11.7237, 2.9361).. controls (11.7237, 3.0665) and (11.7706, 3.1794) .. (11.8644, 3.2748).. controls (11.9559, 3.3671) and (12.0676, 3.4132) .. (12.1995, 3.4132).. controls (12.3331, 3.4132) and (12.446, 3.3671) .. (12.5382, 3.2748) -- cycle(11.9265, 3.2128).. controls (11.8486, 3.1341) and (11.8096, 3.0418) .. (11.8096, 2.936).. controls (11.8096, 2.8302) and (11.8482, 2.7387) .. (11.9253, 2.6616).. controls (12.0025, 2.5844) and (12.0944, 2.5458) .. (12.201, 2.5458).. controls (12.3075, 2.5458) and (12.4002, 2.5848) .. (12.4789, 2.6628).. controls (12.5537, 2.7352) and (12.5911, 2.8262) .. (12.5911, 2.936).. controls (12.5911, 3.0449) and (12.5531, 3.1374) .. (12.4771, 3.2134).. controls (12.4012, 3.2893) and (12.3091, 3.3273) .. (12.201, 3.3273).. controls (12.0928, 3.3273) and (12.0013, 3.2891) .. (11.9265, 3.2128) -- cycle(12.1317, 2.9825).. controls (12.1198, 3.0085) and (12.102, 3.0215) .. (12.0782, 3.0215).. controls (12.0361, 3.0215) and (12.0151, 2.9932) .. (12.0151, 2.9366).. controls (12.0151, 2.88) and (12.0361, 2.8517) .. (12.0782, 2.8517).. controls (12.1059, 2.8517) and (12.1258, 2.8655) .. (12.1377, 2.8931) -- (12.1959, 2.862).. controls (12.1682, 2.8127) and (12.1265, 2.788) .. (12.0709, 2.788).. controls (12.0281, 2.788) and (11.9937, 2.8011) .. (11.9679, 2.8274).. controls (11.9421, 2.8537) and (11.9292, 2.8899) .. (11.9292, 2.9361).. controls (11.9292, 2.9815) and (11.9425, 3.0175) .. (11.9691, 3.0442).. controls (11.9957, 3.0709) and (12.0288, 3.0842) .. (12.0686, 3.0842).. controls (12.1273, 3.0842) and (12.1694, 3.0611) .. (12.1948, 3.0148) -- (12.1317, 2.9825) -- cycle(12.4059, 2.9825).. controls (12.394, 3.0085) and (12.3765, 3.0215) .. (12.3534, 3.0215).. controls (12.3106, 3.0215) and (12.2891, 2.9932) .. (12.2891, 2.9366).. controls (12.2891, 2.88) and (12.3106, 2.8517) .. (12.3534, 2.8517).. controls (12.3813, 2.8517) and (12.4007, 2.8655) .. (12.4118, 2.8931) -- (12.4714, 2.862).. controls (12.4437, 2.8127) and (12.4021, 2.788) .. (12.3466, 2.788).. controls (12.3038, 2.788) and (12.2695, 2.8011) .. (12.2438, 2.8274).. controls (12.2181, 2.8537) and (12.2052, 2.8899) .. (12.2052, 2.9361).. controls (12.2052, 2.9815) and (12.2183, 3.0175) .. (12.2444, 3.0442).. controls (12.2705, 3.0709) and (12.3038, 3.0842) .. (12.3442, 3.0842).. controls (12.4029, 3.0842) and (12.4449, 3.0611) .. (12.4702, 3.0148) -- (12.4059, 2.9825) -- cycle;



      \end{scope}
    \end{scope}
    \path[fill] (7.866, -2.8528) -- (4.7907, -2.8528).. controls (4.7578, -2.8528) and (4.7309, -2.8796) .. (4.7309, -2.9126) -- (4.7309, -3.9576).. controls (4.7309, -3.9651) and (4.737, -3.9711) .. (4.7444, -3.9711) -- (7.9123, -3.9711).. controls (7.9197, -3.9711) and (7.9258, -3.9651) .. (7.9258, -3.9576) -- (7.9258, -2.9126).. controls (7.9258, -2.8796) and (7.8989, -2.8528) .. (7.866, -2.8528) -- cycle(4.7907, -2.8798) -- (7.866, -2.8798).. controls (7.8841, -2.8798) and (7.8988, -2.8945) .. (7.8988, -2.9126).. controls (7.8988, -2.9126) and (7.8988, -3.3329) .. (7.8988, -3.6371) -- (5.7008, -3.6371).. controls (5.6202, -3.7827) and (5.465, -3.8816) .. (5.2869, -3.8816).. controls (5.1087, -3.8816) and (4.9535, -3.7828) .. (4.873, -3.6371) -- (4.7579, -3.6371).. controls (4.7579, -3.3329) and (4.7579, -2.9126) .. (4.7579, -2.9126).. controls (4.7579, -2.8945) and (4.7727, -2.8798) .. (4.7907, -2.8798) -- cycle;



    \path[fill=white] (6.3283, -3.7247).. controls (6.3367, -3.7247) and (6.3443, -3.7255) .. (6.3513, -3.727).. controls (6.3582, -3.7284) and (6.3641, -3.7309) .. (6.3691, -3.7342).. controls (6.374, -3.7376) and (6.3779, -3.7421) .. (6.3806, -3.7477).. controls (6.3833, -3.7533) and (6.3847, -3.7602) .. (6.3847, -3.7684).. controls (6.3847, -3.7773) and (6.3826, -3.7847) .. (6.3786, -3.7906).. controls (6.3746, -3.7966) and (6.3686, -3.8014) .. (6.3607, -3.8052).. controls (6.3716, -3.8083) and (6.3797, -3.8138) .. (6.3851, -3.8216).. controls (6.3904, -3.8295) and (6.3931, -3.8389) .. (6.3931, -3.8499).. controls (6.3931, -3.8588) and (6.3914, -3.8665) .. (6.3879, -3.873).. controls (6.3845, -3.8795) and (6.3798, -3.8848) .. (6.374, -3.8889).. controls (6.3681, -3.8931) and (6.3614, -3.8961) .. (6.3539, -3.8981).. controls (6.3464, -3.9001) and (6.3387, -3.9011) .. (6.3308, -3.9011) -- (6.2451, -3.9011) -- (6.2451, -3.7247) -- (6.3283, -3.7247) -- (6.3283, -3.7247) -- cycle(6.3233, -3.796).. controls (6.3302, -3.796) and (6.3359, -3.7944) .. (6.3404, -3.7911).. controls (6.3448, -3.7878) and (6.347, -3.7825) .. (6.347, -3.7751).. controls (6.347, -3.771) and (6.3463, -3.7676) .. (6.3448, -3.765).. controls (6.3433, -3.7624) and (6.3414, -3.7603) .. (6.3389, -3.7588).. controls (6.3364, -3.7574) and (6.3336, -3.7563) .. (6.3303, -3.7558).. controls (6.3271, -3.7552) and (6.3238, -3.7549) .. (6.3203, -3.7549) -- (6.284, -3.7549) -- (6.284, -3.796) -- (6.3233, -3.796) -- cycle(6.3256, -3.8709).. controls (6.3294, -3.8709) and (6.333, -3.8705) .. (6.3365, -3.8698).. controls (6.3399, -3.869) and (6.343, -3.8678) .. (6.3456, -3.8661).. controls (6.3483, -3.8643) and (6.3503, -3.862) .. (6.3519, -3.859).. controls (6.3535, -3.8561) and (6.3543, -3.8523) .. (6.3543, -3.8477).. controls (6.3543, -3.8386) and (6.3517, -3.8322) .. (6.3466, -3.8283).. controls (6.3415, -3.8244) and (6.3347, -3.8225) .. (6.3263, -3.8225) -- (6.284, -3.8225) -- (6.284, -3.8709) -- (6.3256, -3.8709) -- (6.3256, -3.8709) -- cycle;



    \path[fill=white] (6.4, -3.7247) -- (6.4435, -3.7247) -- (6.4848, -3.7944) -- (6.5258, -3.7247) -- (6.569, -3.7247) -- (6.5036, -3.8334) -- (6.5036, -3.9011) -- (6.4647, -3.9011) -- (6.4647, -3.8324) -- (6.4, -3.7247) -- cycle;



    \path[fill=white] (7.0321, -3.7247) -- (7.1058, -3.843) -- (7.1062, -3.843) -- (7.1062, -3.7247) -- (7.1426, -3.7247) -- (7.1426, -3.9011) -- (7.1038, -3.9011) -- (7.0304, -3.783) -- (7.0299, -3.783) -- (7.0299, -3.9011) -- (6.9935, -3.9011) -- (6.9935, -3.7247) -- (7.0321, -3.7247) -- cycle;



    \path[fill=white] (7.2898, -3.7717).. controls (7.2875, -3.768) and (7.2846, -3.7647) .. (7.2811, -3.7619).. controls (7.2777, -3.7592) and (7.2738, -3.757) .. (7.2694, -3.7554).. controls (7.265, -3.7538) and (7.2604, -3.7531) .. (7.2557, -3.7531).. controls (7.247, -3.7531) and (7.2395, -3.7547) .. (7.2334, -3.7581).. controls (7.2273, -3.7615) and (7.2224, -3.766) .. (7.2186, -3.7717).. controls (7.2148, -3.7774) and (7.2121, -3.7838) .. (7.2103, -3.7911).. controls (7.2086, -3.7983) and (7.2078, -3.8058) .. (7.2078, -3.8135).. controls (7.2078, -3.8209) and (7.2086, -3.8281) .. (7.2103, -3.8351).. controls (7.2121, -3.8421) and (7.2148, -3.8484) .. (7.2186, -3.854).. controls (7.2224, -3.8596) and (7.2273, -3.8641) .. (7.2334, -3.8675).. controls (7.2395, -3.8708) and (7.247, -3.8725) .. (7.2557, -3.8725).. controls (7.2675, -3.8725) and (7.2768, -3.8689) .. (7.2835, -3.8617).. controls (7.2901, -3.8544) and (7.2942, -3.8449) .. (7.2957, -3.833) -- (7.3332, -3.833).. controls (7.3323, -3.844) and (7.3297, -3.854) .. (7.3256, -3.8629).. controls (7.3215, -3.8718) and (7.316, -3.8794) .. (7.3093, -3.8857).. controls (7.3025, -3.8919) and (7.2946, -3.8967) .. (7.2856, -3.9).. controls (7.2765, -3.9033) and (7.2665, -3.905) .. (7.2557, -3.905).. controls (7.2422, -3.905) and (7.23, -3.9026) .. (7.2192, -3.8979).. controls (7.2085, -3.8933) and (7.1993, -3.8868) .. (7.1919, -3.8785).. controls (7.1845, -3.8703) and (7.1788, -3.8606) .. (7.1748, -3.8495).. controls (7.1709, -3.8384) and (7.1689, -3.8264) .. (7.1689, -3.8135).. controls (7.1689, -3.8004) and (7.1709, -3.7881) .. (7.1748, -3.7768).. controls (7.1788, -3.7656) and (7.1845, -3.7557) .. (7.1919, -3.7473).. controls (7.1993, -3.7389) and (7.2085, -3.7323) .. (7.2192, -3.7276).. controls (7.23, -3.7228) and (7.2422, -3.7204) .. (7.2557, -3.7204).. controls (7.2654, -3.7204) and (7.2746, -3.7218) .. (7.2832, -3.7246).. controls (7.2919, -3.7274) and (7.2996, -3.7315) .. (7.3064, -3.7369).. controls (7.3133, -3.7422) and (7.3189, -3.7488) .. (7.3234, -3.7567).. controls (7.3278, -3.7647) and (7.3306, -3.7737) .. (7.3318, -3.7839) -- (7.2942, -3.7839).. controls (7.2936, -3.7795) and (7.2921, -3.7754) .. (7.2898, -3.7717) -- cycle;



    \begin{scope}[cm={ 1.1468,-0.0,-0.0,1.1468,(12.1163, -4.5593)}]
      \path[fill=white] (-4.0784, 1.1402).. controls (-4.0783, 1.0125) and (-4.1818, 0.9089) .. (-4.3095, 0.9088).. controls (-4.4372, 0.9087) and (-4.5408, 1.0122) .. (-4.5409, 1.1399).. controls (-4.5409, 1.14) and (-4.5409, 1.1401) .. (-4.5409, 1.1402).. controls (-4.541, 1.2679) and (-4.4375, 1.3715) .. (-4.3098, 1.3716).. controls (-4.1821, 1.3717) and (-4.0785, 1.2682) .. (-4.0784, 1.1406).. controls (-4.0784, 1.1404) and (-4.0784, 1.1403) .. (-4.0784, 1.1402) -- cycle;



      \path[fill] (-4.31, 1.4094).. controls (-4.2345, 1.4094) and (-4.1707, 1.3834) .. (-4.1186, 1.3314).. controls (-4.0665, 1.2793) and (-4.0404, 1.2156) .. (-4.0404, 1.1402).. controls (-4.0404, 1.0648) and (-4.0661, 1.0018) .. (-4.1173, 0.9511).. controls (-4.1716, 0.8977) and (-4.2359, 0.871) .. (-4.31, 0.871).. controls (-4.3832, 0.871) and (-4.4463, 0.8975) .. (-4.4994, 0.9504).. controls (-4.5524, 1.0033) and (-4.5789, 1.0666) .. (-4.5789, 1.1402).. controls (-4.5789, 1.2138) and (-4.5524, 1.2775) .. (-4.4994, 1.3314).. controls (-4.4477, 1.3834) and (-4.3846, 1.4094) .. (-4.31, 1.4094) -- cycle(-4.5183, 1.2125).. controls (-4.5263, 1.1898) and (-4.5303, 1.1657) .. (-4.5303, 1.1402).. controls (-4.5303, 1.0805) and (-4.5086, 1.029) .. (-4.465, 0.9854).. controls (-4.4214, 0.9419) and (-4.3695, 0.9201) .. (-4.3093, 0.9201).. controls (-4.2491, 0.9201) and (-4.1968, 0.9421) .. (-4.1523, 0.9861).. controls (-4.1374, 1.0005) and (-4.1251, 1.0162) .. (-4.1155, 1.0332) -- (-4.217, 1.0784).. controls (-4.2239, 1.0442) and (-4.2543, 1.0212) .. (-4.2911, 1.0185) -- (-4.2911, 0.977) -- (-4.322, 0.977) -- (-4.322, 1.0185).. controls (-4.3522, 1.0188) and (-4.3814, 1.0312) .. (-4.4037, 1.0507) -- (-4.3667, 1.0881).. controls (-4.3488, 1.0713) and (-4.3309, 1.0637) .. (-4.3066, 1.0637).. controls (-4.2908, 1.0637) and (-4.2733, 1.0699) .. (-4.2733, 1.0905).. controls (-4.2733, 1.0978) and (-4.2761, 1.1029) .. (-4.2805, 1.1067) -- (-4.3062, 1.1181) -- (-4.3382, 1.1323).. controls (-4.354, 1.1394) and (-4.3673, 1.1453) .. (-4.3808, 1.1513) -- (-4.5183, 1.2125) -- cycle(-4.3093, 1.361).. controls (-4.3704, 1.361) and (-4.4221, 1.3394) .. (-4.4643, 1.2964).. controls (-4.4758, 1.2847) and (-4.4858, 1.2726) .. (-4.4943, 1.26) -- (-4.3914, 1.2142).. controls (-4.3821, 1.2427) and (-4.355, 1.26) .. (-4.322, 1.262) -- (-4.322, 1.3035) -- (-4.2911, 1.3035) -- (-4.2911, 1.262).. controls (-4.2698, 1.2609) and (-4.2465, 1.2551) .. (-4.2235, 1.2373) -- (-4.2588, 1.2009).. controls (-4.2719, 1.2102) and (-4.2884, 1.2167) .. (-4.3048, 1.2167).. controls (-4.3182, 1.2167) and (-4.3371, 1.2126) .. (-4.3371, 1.1958).. controls (-4.3371, 1.1932) and (-4.3363, 1.191) .. (-4.3347, 1.1889) -- (-4.3003, 1.1736) -- (-4.277, 1.1632).. controls (-4.2621, 1.1566) and (-4.2478, 1.1503) .. (-4.2337, 1.144) -- (-4.0958, 1.0826).. controls (-4.0912, 1.1006) and (-4.089, 1.1199) .. (-4.089, 1.1402).. controls (-4.089, 1.2017) and (-4.1105, 1.2537) .. (-4.1536, 1.2964).. controls (-4.1963, 1.3394) and (-4.2482, 1.361) .. (-4.3093, 1.361) -- cycle;



    \end{scope}
    \path[fill=white] (6.4178, -3.258) circle (0.2859cm);



    \path[fill] (6.5005, -3.1752).. controls (6.5005, -3.1642) and (6.4916, -3.1553) .. (6.4806, -3.1553) -- (6.3543, -3.1553).. controls (6.3433, -3.1553) and (6.3344, -3.1642) .. (6.3344, -3.1752) -- (6.3344, -3.3015) -- (6.3696, -3.3015) -- (6.3696, -3.4511) -- (6.4653, -3.4511) -- (6.4653, -3.3015) -- (6.5005, -3.3015) -- (6.5005, -3.1752) -- (6.5005, -3.1752) -- cycle;



    \path[fill] (6.4175, -3.0955) circle (0.0432cm);



    \path[fill,even odd rule] (6.4171, -2.9429).. controls (6.3316, -2.9429) and (6.2592, -2.9728) .. (6.1999, -3.0325).. controls (6.1391, -3.0942) and (6.1087, -3.1673) .. (6.1087, -3.2517).. controls (6.1087, -3.336) and (6.1391, -3.4086) .. (6.1999, -3.4693).. controls (6.2607, -3.53) and (6.3331, -3.5604) .. (6.4171, -3.5604).. controls (6.5021, -3.5604) and (6.5758, -3.5298) .. (6.6381, -3.4686).. controls (6.6968, -3.4104) and (6.7262, -3.3381) .. (6.7262, -3.2517).. controls (6.7262, -3.1652) and (6.6963, -3.0922) .. (6.6366, -3.0325).. controls (6.5768, -2.9728) and (6.5036, -2.9429) .. (6.4171, -2.9429) -- cycle(6.4178, -2.9985).. controls (6.4879, -2.9985) and (6.5474, -3.0232) .. (6.5964, -3.0726).. controls (6.6458, -3.1215) and (6.6706, -3.1812) .. (6.6706, -3.2517).. controls (6.6706, -3.3227) and (6.6464, -3.3816) .. (6.5979, -3.4284).. controls (6.5469, -3.4789) and (6.4869, -3.5041) .. (6.4178, -3.5041).. controls (6.3488, -3.5041) and (6.2893, -3.4791) .. (6.2393, -3.4292).. controls (6.1893, -3.3793) and (6.1644, -3.3201) .. (6.1644, -3.2517).. controls (6.1644, -3.1832) and (6.1896, -3.1235) .. (6.2401, -3.0726).. controls (6.2885, -3.0232) and (6.3478, -2.9985) .. (6.4178, -2.9985) -- cycle;



  \end{scope}

\end{tikzpicture}
 \\ {\tt cc-by-nc}

        \column{0.25\textwidth}
        \centering

        
\definecolor{caab2ab}{RGB}{170,178,171}


\def \globalscale {0.550000}
\begin{tikzpicture}[y=1cm, x=1cm, yscale=\globalscale,xscale=\globalscale, every node/.append style={scale=\globalscale}, inner sep=0pt, outer sep=0pt]
  \begin{scope}[cm={ 0.9938,-0.0,-0.0,0.9937,(-4.7015, 5.9154)}]
    \path[fill=caab2ab] (4.8216, -4.8464) -- (7.8396, -4.8518).. controls (7.8818, -4.8518) and (7.9194, -4.8455) .. (7.9194, -4.936) -- (7.9157, -5.9299) -- (4.7455, -5.9299) -- (4.7455, -4.9323).. controls (4.7455, -4.8877) and (4.7498, -4.8464) .. (4.8216, -4.8464) -- cycle;



    \begin{scope}[cm={ 0.8729,-0.0,-0.0,0.8729,(1.3262, -3.648)}]
      \path[fill=white] (4.9533, -1.9555).. controls (4.9534, -2.1851) and (4.7673, -2.3714) .. (4.5376, -2.3716).. controls (4.308, -2.3717) and (4.1217, -2.1856) .. (4.1215, -1.956).. controls (4.1215, -1.9558) and (4.1215, -1.9556) .. (4.1215, -1.9555).. controls (4.1214, -1.7258) and (4.3075, -1.5395) .. (4.5371, -1.5394).. controls (4.7668, -1.5392) and (4.9531, -1.7253) .. (4.9533, -1.955).. controls (4.9533, -1.9551) and (4.9533, -1.9553) .. (4.9533, -1.9555) -- cycle;



      \begin{scope}[shift={(-7.6627, -2.6211)}]
        \path[fill] (12.5382, 1.0043).. controls (12.6304, 0.9121) and (12.6765, 0.7992) .. (12.6765, 0.6656).. controls (12.6765, 0.532) and (12.6312, 0.4203) .. (12.5406, 0.3304).. controls (12.4444, 0.2358) and (12.3307, 0.1885) .. (12.1995, 0.1885).. controls (12.07, 0.1885) and (11.9582, 0.2354) .. (11.8644, 0.3293).. controls (11.7706, 0.4231) and (11.7237, 0.5352) .. (11.7237, 0.6656).. controls (11.7237, 0.796) and (11.7706, 0.9089) .. (11.8644, 1.0043).. controls (11.9559, 1.0966) and (12.0676, 1.1427) .. (12.1995, 1.1427).. controls (12.3331, 1.1427) and (12.446, 1.0966) .. (12.5382, 1.0043) -- cycle(11.9265, 0.9423).. controls (11.8486, 0.8636) and (11.8096, 0.7713) .. (11.8096, 0.6655).. controls (11.8096, 0.5597) and (11.8482, 0.4683) .. (11.9253, 0.3911).. controls (12.0025, 0.314) and (12.0944, 0.2754) .. (12.201, 0.2754).. controls (12.3075, 0.2754) and (12.4002, 0.3143) .. (12.4789, 0.3923).. controls (12.5537, 0.4647) and (12.5911, 0.5557) .. (12.5911, 0.6655).. controls (12.5911, 0.7745) and (12.5531, 0.8669) .. (12.4771, 0.9429).. controls (12.4012, 1.0188) and (12.3091, 1.0569) .. (12.201, 1.0569).. controls (12.0928, 1.0569) and (12.0013, 1.0187) .. (11.9265, 0.9423) -- cycle(12.1317, 0.712).. controls (12.1198, 0.738) and (12.102, 0.751) .. (12.0782, 0.751).. controls (12.0361, 0.751) and (12.0151, 0.7227) .. (12.0151, 0.6661).. controls (12.0151, 0.6095) and (12.0361, 0.5812) .. (12.0782, 0.5812).. controls (12.1059, 0.5812) and (12.1258, 0.595) .. (12.1377, 0.6226) -- (12.1959, 0.5916).. controls (12.1682, 0.5422) and (12.1265, 0.5175) .. (12.0709, 0.5175).. controls (12.0281, 0.5175) and (11.9937, 0.5307) .. (11.9679, 0.5569).. controls (11.9421, 0.5832) and (11.9292, 0.6195) .. (11.9292, 0.6656).. controls (11.9292, 0.711) and (11.9425, 0.747) .. (11.9691, 0.7737).. controls (11.9957, 0.8004) and (12.0288, 0.8137) .. (12.0686, 0.8137).. controls (12.1273, 0.8137) and (12.1694, 0.7906) .. (12.1948, 0.7443) -- (12.1317, 0.712) -- cycle(12.4059, 0.712).. controls (12.394, 0.738) and (12.3765, 0.751) .. (12.3534, 0.751).. controls (12.3106, 0.751) and (12.2891, 0.7227) .. (12.2891, 0.6661).. controls (12.2891, 0.6095) and (12.3106, 0.5812) .. (12.3534, 0.5812).. controls (12.3813, 0.5812) and (12.4007, 0.595) .. (12.4118, 0.6226) -- (12.4714, 0.5916).. controls (12.4437, 0.5422) and (12.4021, 0.5175) .. (12.3466, 0.5175).. controls (12.3038, 0.5175) and (12.2695, 0.5307) .. (12.2438, 0.5569).. controls (12.2181, 0.5832) and (12.2052, 0.6195) .. (12.2052, 0.6656).. controls (12.2052, 0.711) and (12.2183, 0.747) .. (12.2444, 0.7737).. controls (12.2705, 0.8004) and (12.3038, 0.8137) .. (12.3442, 0.8137).. controls (12.4029, 0.8137) and (12.4449, 0.7906) .. (12.4702, 0.7443) -- (12.4059, 0.712) -- cycle;



      \end{scope}
    \end{scope}
    \path[fill=white] (6.4178, -5.2389) circle (0.2859cm);



    \path[fill] (6.5005, -5.1562).. controls (6.5005, -5.1452) and (6.4916, -5.1362) .. (6.4806, -5.1362) -- (6.3543, -5.1362).. controls (6.3433, -5.1362) and (6.3344, -5.1452) .. (6.3344, -5.1562) -- (6.3344, -5.2825) -- (6.3696, -5.2825) -- (6.3696, -5.432) -- (6.4653, -5.432) -- (6.4653, -5.2825) -- (6.5005, -5.2825) -- (6.5005, -5.1562) -- (6.5005, -5.1562) -- cycle;



    \path[fill] (6.4175, -5.0764) circle (0.0432cm);



    \path[fill,even odd rule] (6.4171, -4.9239).. controls (6.3316, -4.9239) and (6.2592, -4.9537) .. (6.1999, -5.0134).. controls (6.1391, -5.0752) and (6.1087, -5.1483) .. (6.1087, -5.2326).. controls (6.1087, -5.317) and (6.1391, -5.3896) .. (6.1999, -5.4503).. controls (6.2607, -5.511) and (6.3331, -5.5414) .. (6.4171, -5.5414).. controls (6.5021, -5.5414) and (6.5758, -5.5107) .. (6.6381, -5.4495).. controls (6.6968, -5.3914) and (6.7262, -5.3191) .. (6.7262, -5.2326).. controls (6.7262, -5.1462) and (6.6963, -5.0731) .. (6.6366, -5.0134).. controls (6.5768, -4.9537) and (6.5036, -4.9239) .. (6.4171, -4.9239) -- cycle(6.4178, -4.9795).. controls (6.4879, -4.9795) and (6.5474, -5.0042) .. (6.5964, -5.0536).. controls (6.6458, -5.1024) and (6.6706, -5.1621) .. (6.6706, -5.2326).. controls (6.6706, -5.3036) and (6.6464, -5.3626) .. (6.5979, -5.4094).. controls (6.5469, -5.4598) and (6.4869, -5.485) .. (6.4178, -5.485).. controls (6.3488, -5.485) and (6.2893, -5.4601) .. (6.2393, -5.4102).. controls (6.1893, -5.3602) and (6.1644, -5.3011) .. (6.1644, -5.2326).. controls (6.1644, -5.1642) and (6.1896, -5.1045) .. (6.2401, -5.0536).. controls (6.2885, -5.0042) and (6.3478, -4.9795) .. (6.4178, -4.9795) -- cycle;



    \path[fill] (7.866, -4.8348) -- (4.7907, -4.8348).. controls (4.7578, -4.8348) and (4.7309, -4.8616) .. (4.7309, -4.8946) -- (4.7309, -5.9396).. controls (4.7309, -5.9471) and (4.737, -5.9531) .. (4.7444, -5.9531) -- (7.9123, -5.9531).. controls (7.9197, -5.9531) and (7.9258, -5.9471) .. (7.9258, -5.9396) -- (7.9258, -4.8946).. controls (7.9258, -4.8616) and (7.8989, -4.8348) .. (7.866, -4.8348) -- cycle(4.7907, -4.8618) -- (7.866, -4.8618).. controls (7.8841, -4.8618) and (7.8988, -4.8765) .. (7.8988, -4.8946).. controls (7.8988, -4.8946) and (7.8988, -5.3148) .. (7.8988, -5.619) -- (5.7008, -5.619).. controls (5.6202, -5.7647) and (5.465, -5.8636) .. (5.2869, -5.8636).. controls (5.1087, -5.8636) and (4.9535, -5.7648) .. (4.873, -5.619) -- (4.7579, -5.619).. controls (4.7579, -5.3148) and (4.7579, -4.8946) .. (4.7579, -4.8946).. controls (4.7579, -4.8765) and (4.7727, -4.8618) .. (4.7907, -4.8618) -- cycle;



    \path[fill=white] (6.3283, -5.7068).. controls (6.3367, -5.7068) and (6.3443, -5.7075) .. (6.3513, -5.709).. controls (6.3582, -5.7105) and (6.3641, -5.7129) .. (6.3691, -5.7163).. controls (6.374, -5.7196) and (6.3779, -5.7241) .. (6.3806, -5.7297).. controls (6.3833, -5.7353) and (6.3847, -5.7422) .. (6.3847, -5.7504).. controls (6.3847, -5.7593) and (6.3826, -5.7667) .. (6.3786, -5.7727).. controls (6.3746, -5.7786) and (6.3686, -5.7834) .. (6.3607, -5.7872).. controls (6.3716, -5.7903) and (6.3797, -5.7958) .. (6.3851, -5.8037).. controls (6.3904, -5.8115) and (6.3931, -5.8209) .. (6.3931, -5.8319).. controls (6.3931, -5.8408) and (6.3914, -5.8485) .. (6.3879, -5.855).. controls (6.3845, -5.8616) and (6.3798, -5.8669) .. (6.374, -5.871).. controls (6.3681, -5.8751) and (6.3614, -5.8781) .. (6.3539, -5.8801).. controls (6.3464, -5.8821) and (6.3387, -5.8831) .. (6.3308, -5.8831) -- (6.2451, -5.8831) -- (6.2451, -5.7068) -- (6.3283, -5.7068) -- (6.3283, -5.7068) -- cycle(6.3233, -5.7781).. controls (6.3302, -5.7781) and (6.3359, -5.7764) .. (6.3404, -5.7731).. controls (6.3448, -5.7699) and (6.347, -5.7645) .. (6.347, -5.7571).. controls (6.347, -5.753) and (6.3463, -5.7497) .. (6.3448, -5.747).. controls (6.3433, -5.7444) and (6.3414, -5.7424) .. (6.3389, -5.7409).. controls (6.3364, -5.7394) and (6.3336, -5.7384) .. (6.3303, -5.7378).. controls (6.3271, -5.7372) and (6.3238, -5.7369) .. (6.3203, -5.7369) -- (6.284, -5.7369) -- (6.284, -5.7781) -- (6.3233, -5.7781) -- cycle(6.3256, -5.8529).. controls (6.3294, -5.8529) and (6.333, -5.8525) .. (6.3365, -5.8518).. controls (6.3399, -5.851) and (6.343, -5.8498) .. (6.3456, -5.8481).. controls (6.3483, -5.8464) and (6.3503, -5.844) .. (6.3519, -5.8411).. controls (6.3535, -5.8381) and (6.3543, -5.8343) .. (6.3543, -5.8297).. controls (6.3543, -5.8207) and (6.3517, -5.8142) .. (6.3466, -5.8103).. controls (6.3415, -5.8065) and (6.3347, -5.8045) .. (6.3263, -5.8045) -- (6.284, -5.8045) -- (6.284, -5.8529) -- (6.3256, -5.8529) -- (6.3256, -5.8529) -- cycle;



    \path[fill=white] (6.4, -5.7068) -- (6.4435, -5.7068) -- (6.4848, -5.7764) -- (6.5258, -5.7068) -- (6.569, -5.7068) -- (6.5036, -5.8154) -- (6.5036, -5.8831) -- (6.4647, -5.8831) -- (6.4647, -5.8144) -- (6.4, -5.7068) -- cycle;



    \path[fill=white] (7.0188, -5.7068) -- (7.0924, -5.825) -- (7.0929, -5.825) -- (7.0929, -5.7068) -- (7.1292, -5.7068) -- (7.1292, -5.8831) -- (7.0905, -5.8831) -- (7.0171, -5.765) -- (7.0166, -5.765) -- (7.0166, -5.8831) -- (6.9802, -5.8831) -- (6.9802, -5.7068) -- (7.0188, -5.7068) -- cycle;



    \path[fill=white] (7.2394, -5.7068).. controls (7.2508, -5.7068) and (7.2613, -5.7086) .. (7.2711, -5.7122).. controls (7.281, -5.7158) and (7.2894, -5.7213) .. (7.2966, -5.7285).. controls (7.3038, -5.7357) and (7.3094, -5.7448) .. (7.3134, -5.7556).. controls (7.3175, -5.7665) and (7.3195, -5.7793) .. (7.3195, -5.7939).. controls (7.3195, -5.8068) and (7.3178, -5.8186) .. (7.3146, -5.8295).. controls (7.3112, -5.8404) and (7.3063, -5.8497) .. (7.2996, -5.8577).. controls (7.2929, -5.8655) and (7.2846, -5.8718) .. (7.2746, -5.8763).. controls (7.2646, -5.8808) and (7.2529, -5.8831) .. (7.2394, -5.8831) -- (7.1632, -5.8831) -- (7.1632, -5.7068) -- (7.2394, -5.7068) -- (7.2394, -5.7068) -- cycle(7.2367, -5.8504).. controls (7.2423, -5.8504) and (7.2477, -5.8495) .. (7.253, -5.8477).. controls (7.2582, -5.8459) and (7.2629, -5.8429) .. (7.267, -5.8387).. controls (7.2711, -5.8345) and (7.2745, -5.829) .. (7.2769, -5.8223).. controls (7.2794, -5.8156) and (7.2806, -5.8073) .. (7.2806, -5.7976).. controls (7.2806, -5.7887) and (7.2798, -5.7807) .. (7.278, -5.7736).. controls (7.2763, -5.7664) and (7.2735, -5.7603) .. (7.2695, -5.7552).. controls (7.2656, -5.7501) and (7.2603, -5.7462) .. (7.2538, -5.7435).. controls (7.2473, -5.7408) and (7.2393, -5.7394) .. (7.2298, -5.7394) -- (7.2021, -5.7394) -- (7.2021, -5.8504) -- (7.2367, -5.8504) -- (7.2367, -5.8504) -- cycle;



    \begin{scope}[cm={ 0.625,-0.0,-0.0,0.625,(-4.8422, -7.9688)}]
      \path[fill=white] (19.6831, 4.3762).. controls (19.6833, 4.1272) and (19.4816, 3.9252) .. (19.2325, 3.925).. controls (18.9835, 3.9248) and (18.7815, 4.1265) .. (18.7813, 4.3756).. controls (18.7813, 4.3758) and (18.7813, 4.376) .. (18.7813, 4.3762).. controls (18.7811, 4.6252) and (18.9829, 4.8272) .. (19.2319, 4.8274).. controls (19.4809, 4.8277) and (19.6829, 4.6259) .. (19.6831, 4.3769).. controls (19.6831, 4.3767) and (19.6831, 4.3764) .. (19.6831, 4.3762) -- cycle;



      \begin{scope}[shift={(-0.6337, 2.3262)}]
        \path[fill] (19.8589, 2.5439).. controls (19.722, 2.5439) and (19.6062, 2.4962) .. (19.5114, 2.4007).. controls (19.4142, 2.3019) and (19.3655, 2.185) .. (19.3655, 2.05).. controls (19.3655, 1.9149) and (19.4142, 1.7989) .. (19.5114, 1.7017).. controls (19.6087, 1.6045) and (19.7245, 1.556) .. (19.8589, 1.556).. controls (19.9949, 1.556) and (20.1128, 1.605) .. (20.2125, 1.7029).. controls (20.3065, 1.796) and (20.3535, 1.9117) .. (20.3535, 2.05).. controls (20.3535, 2.1883) and (20.3057, 2.3052) .. (20.2101, 2.4007).. controls (20.1144, 2.4962) and (19.9974, 2.5439) .. (19.8589, 2.5439) -- cycle(19.8601, 2.4551).. controls (19.9722, 2.4551) and (20.0674, 2.4155) .. (20.1458, 2.3365).. controls (20.2249, 2.2583) and (20.2645, 2.1628) .. (20.2645, 2.05).. controls (20.2645, 1.9364) and (20.2257, 1.8421) .. (20.1482, 1.7672).. controls (20.0666, 1.6865) and (19.9706, 1.6462) .. (19.8601, 1.6462).. controls (19.7496, 1.6462) and (19.6544, 1.6861) .. (19.5745, 1.766).. controls (19.4945, 1.8458) and (19.4545, 1.9405) .. (19.4545, 2.05).. controls (19.4545, 2.1595) and (19.4949, 2.255) .. (19.5757, 2.3365).. controls (19.6532, 2.4155) and (19.748, 2.4551) .. (19.8601, 2.4551) -- cycle;



        \path[fill] (20.0462, 2.167) -- (19.687, 2.167) -- (19.687, 2.082) -- (20.0462, 2.082) -- (20.0462, 2.167) -- cycle(20.0462, 2.0083) -- (19.687, 2.0083) -- (19.687, 1.9232) -- (20.0462, 1.9232) -- (20.0462, 2.0083) -- cycle;



      \end{scope}
    \end{scope}
  \end{scope}

\end{tikzpicture}
 \\ {\tt cc-by-nd}

        \vspace{3pt}%LaTeX with PSTricks extensions
%%Creator: inkscape 0.91
%%Please note this file requires PSTricks extensions
\psset{xunit=.5pt,yunit=.5pt,runit=.5pt}
\begin{pspicture}(120,42)
{
\newrgbcolor{curcolor}{0.66666669 0.69803923 0.67058825}
\pscustom[linestyle=none,fillstyle=solid,fillcolor=curcolor]
{
\newpath
\moveto(3.40785387,41.52353382)
\lineto(116.76241589,41.32169969)
\curveto(118.34625388,41.32169969)(119.76122865,41.55702048)(119.76122865,38.16165214)
\lineto(119.62244718,0.83258309)
\lineto(0.54733563,0.83258309)
\lineto(0.54733563,38.30041807)
\curveto(0.54733563,39.97432373)(0.70941133,41.52353382)(3.40785387,41.52353382)
\closepath
}
}
{
\newrgbcolor{curcolor}{1 1 1}
\pscustom[linestyle=none,fillstyle=solid,fillcolor=curcolor]
{
\newpath
\moveto(34.52222482,22.42449543)
\curveto(34.52706542,14.89529965)(28.42561219,8.7882238)(20.89557306,8.78290667)
\curveto(13.36555996,8.778535)(7.2568198,14.87883649)(7.25245631,22.40798024)
\lineto(7.25245631,22.42449543)
\curveto(7.24808415,29.95464536)(13.34904291,36.06172121)(20.87908204,36.06611023)
\curveto(28.41007541,36.07098498)(34.51786133,29.97015438)(34.52222482,22.44101063)
\lineto(34.52222482,22.42449543)
\closepath
}
}
{
\newrgbcolor{curcolor}{0 0 0}
\pscustom[linestyle=none,fillstyle=solid,fillcolor=curcolor]
{
\newpath
\moveto(31.97128639,33.52952664)
\curveto(34.99485183,30.50582276)(36.50687311,26.80382476)(36.50687311,22.42450411)
\curveto(36.50687311,18.04469772)(35.02103274,14.38246977)(32.04940405,11.43690082)
\curveto(28.89580155,8.33457627)(25.16863934,6.78391709)(20.86789139,6.78391709)
\curveto(16.61911489,6.78391709)(12.95647674,8.32146993)(9.88101793,11.39806755)
\curveto(6.80502994,14.47368501)(5.26730487,18.14899328)(5.26730487,22.42450411)
\curveto(5.26730487,26.69953787)(6.80502994,30.40105014)(9.88101793,33.52952664)
\curveto(12.87837643,36.55370759)(16.54097121,38.06508246)(20.86789139,38.06508246)
\curveto(25.2472428,38.06508246)(28.94772095,36.55370759)(31.97128639,33.52952664)
\closepath
\moveto(11.9166227,31.49561574)
\curveto(9.36034053,28.91391895)(8.08268957,25.88973801)(8.08268957,22.42063553)
\curveto(8.08268957,18.95248718)(9.34773587,15.95403317)(11.87730796,13.4251694)
\curveto(14.40737453,10.89541222)(17.4192722,9.63052495)(20.91401594,9.63052495)
\curveto(24.40873366,9.63052495)(27.44638717,10.90803281)(30.02837314,13.46355163)
\curveto(32.4798189,15.83660525)(33.70602758,18.82150188)(33.70602758,22.42063553)
\curveto(33.70602758,25.99260237)(32.45990996,29.02404341)(29.96914946,31.51504539)
\curveto(27.4788921,34.00504987)(24.46067909,35.25049881)(20.91401594,35.25049881)
\curveto(17.3673875,35.25049881)(14.36759135,33.99869186)(11.9166227,31.49561574)
\closepath
\moveto(18.64358107,23.94657396)
\curveto(18.25293205,24.7980348)(17.66823374,25.22357873)(16.8884278,25.22357873)
\curveto(15.50986149,25.22357873)(14.82082991,24.29588255)(14.82082991,22.44054224)
\curveto(14.82082991,20.58467282)(15.50986149,19.65697664)(16.8884278,19.65697664)
\curveto(17.7987566,19.65697664)(18.44900261,20.10914959)(18.83912245,21.01455372)
\lineto(20.74999922,19.9970735)
\curveto(19.8391933,18.37896569)(18.47278056,17.56922654)(16.65067425,17.56922654)
\curveto(15.24542389,17.56922654)(14.11965344,18.00006158)(13.27434317,18.86126326)
\curveto(12.42761022,19.72294202)(12.00542786,20.91068318)(12.00542786,22.42449543)
\curveto(12.00542786,23.91258075)(12.44116044,25.09354757)(13.31315477,25.96835)
\curveto(14.18514909,26.84262333)(15.27112768,27.28029344)(16.57302504,27.28029344)
\curveto(18.49896147,27.28029344)(19.87803092,26.52142701)(20.71166475,25.00523809)
\lineto(18.64358107,23.94657396)
\closepath
\moveto(27.63369605,23.94657396)
\curveto(27.24256992,24.7980348)(26.669496,25.22357873)(25.9139885,25.22357873)
\curveto(24.50773185,25.22357873)(23.80413508,24.29588255)(23.80413508,22.44054224)
\curveto(23.80413508,20.58467282)(24.50773185,19.65697664)(25.9139885,19.65697664)
\curveto(26.82574867,19.65697664)(27.46434426,20.10914959)(27.82875164,21.01455372)
\lineto(29.78235239,19.9970735)
\curveto(28.87300386,18.37896569)(27.50849093,17.56922654)(25.68980254,17.56922654)
\curveto(24.28645198,17.56922654)(23.16311919,18.00006158)(22.31832074,18.86126326)
\curveto(21.474971,19.72294202)(21.0523202,20.91068318)(21.0523202,22.42449543)
\curveto(21.0523202,23.91258075)(21.48127767,25.09354757)(22.3387068,25.96835)
\curveto(23.19563279,26.84262333)(24.28646066,27.28029344)(25.612162,27.28029344)
\curveto(27.53468919,27.28029344)(28.91182413,26.52142701)(29.74256921,25.00523809)
\lineto(27.63369605,23.94657396)
\closepath
}
}
{
\newrgbcolor{curcolor}{0 0 0}
\pscustom[linestyle=none,fillstyle=solid,fillcolor=curcolor]
{
\newpath
\moveto(117.75332456,41.9999983)
\lineto(2.24668075,41.9999983)
\curveto(1.00785361,41.9999983)(0.00000097,40.99177164)(0.00000097,39.75308324)
\lineto(0.00000097,0.50750415)
\curveto(0.00000097,0.22754773)(0.22709974,-0.00000256)(0.50707758,-0.00000256)
\lineto(119.49244079,-0.00000256)
\curveto(119.77242856,-0.00000256)(120.00000434,0.22754773)(120.00000434,0.50750415)
\lineto(120.00000434,39.7530733)
\curveto(120.00000434,40.99177164)(118.99215171,41.9999983)(117.75332456,41.9999983)
\closepath
\moveto(2.24668075,40.98497494)
\lineto(117.75332456,40.98497494)
\curveto(118.43266311,40.98497494)(118.98487723,40.43234571)(118.98487723,39.7530733)
\lineto(118.98487723,12.50818861)
\lineto(36.42778781,12.50818861)
\curveto(33.40180508,7.0376709)(27.5720788,3.32305646)(20.88152974,3.32305646)
\curveto(14.1880689,3.32305646)(8.35979354,7.03427255)(5.33527167,12.50818861)
\lineto(1.01417405,12.50818861)
\lineto(1.01417405,39.7530733)
\curveto(1.01415418,40.43234571)(1.5673422,40.98497494)(2.24668075,40.98497494)
\closepath
}
}
{
\newrgbcolor{curcolor}{1 1 1}
\pscustom[linestyle=none,fillstyle=solid,fillcolor=curcolor]
{
\newpath
\moveto(86.2638626,4.26772802)
\curveto(86.34344456,4.11295407)(86.45019648,3.9877716)(86.5831544,3.89170364)
\curveto(86.71611232,3.79612259)(86.87139055,3.724797)(87.04996301,3.67870067)
\curveto(87.22949943,3.63211745)(87.41486934,3.60883578)(87.60605287,3.60883578)
\curveto(87.73512511,3.60883578)(87.87390659,3.61950779)(88.0223973,3.6413387)
\curveto(88.1699141,3.66268272)(88.30869558,3.7048938)(88.43874172,3.76652117)
\curveto(88.56781396,3.82813861)(88.67651369,3.91353456)(88.76191921,4.02173522)
\curveto(88.84926259,4.12993588)(88.89196535,4.26724112)(88.89196535,4.43463467)
\curveto(88.89196535,4.61367402)(88.8347037,4.75873982)(88.72019035,4.87033883)
\curveto(88.60664097,4.9819279)(88.45622232,5.07412055)(88.27085241,5.14883455)
\curveto(88.0864564,5.22258469)(87.87586433,5.28760048)(87.64197805,5.34339502)
\curveto(87.40711785,5.39918955)(87.16934588,5.46080699)(86.92867207,5.52922113)
\curveto(86.6812008,5.59083857)(86.44051705,5.66652637)(86.20565686,5.75629446)
\curveto(85.97177057,5.84556572)(85.76116857,5.96201384)(85.57580859,6.10417812)
\curveto(85.39044862,6.2463424)(85.24099394,6.42392106)(85.12647065,6.63740099)
\curveto(85.01292127,6.85088093)(84.95565962,7.10900643)(84.95565962,7.41176756)
\curveto(84.95565962,7.75236762)(85.02844412,8.04736819)(85.17304915,8.29772319)
\curveto(85.31861815,8.5480782)(85.50883771,8.75670903)(85.7436979,8.92410257)
\curveto(85.97758419,9.09100922)(86.24350003,9.214731)(86.54047152,9.29527784)
\curveto(86.8364691,9.37533779)(87.13344059,9.41560127)(87.42943817,9.41560127)
\curveto(87.77589993,9.41560127)(88.10781275,9.37678854)(88.42613064,9.29915315)
\curveto(88.74348457,9.22200466)(89.02686114,9.09634523)(89.27336844,8.92312878)
\curveto(89.52083971,8.74991233)(89.71688283,8.52867184)(89.86245183,8.25938743)
\curveto(90.00705686,7.99010302)(90.07984136,7.66357332)(90.07984136,7.28027529)
\lineto(88.66680445,7.28027529)
\curveto(88.65418344,7.4782341)(88.61245459,7.6422293)(88.54258187,7.77177398)
\curveto(88.47173524,7.90180556)(88.3775944,8.00369642)(88.26114317,8.07841042)
\curveto(88.14371805,8.15216056)(88.00978622,8.20504365)(87.85935764,8.23560892)
\curveto(87.70796508,8.26666109)(87.54395152,8.28219214)(87.36537906,8.28219214)
\curveto(87.2489179,8.28219214)(87.13149278,8.26957254)(87.01503162,8.24531707)
\curveto(86.89760649,8.2200878)(86.79181853,8.17690293)(86.69671372,8.11528549)
\curveto(86.600635,8.05318116)(86.52202695,7.97603266)(86.46088956,7.88336305)
\curveto(86.39975217,7.79020655)(86.36869652,7.67279457)(86.36869652,7.53063029)
\curveto(86.36869652,7.40059871)(86.39295471,7.2953095)(86.44245493,7.2152595)
\curveto(86.49194521,7.13472259)(86.58997173,7.06048555)(86.73457676,6.99255831)
\curveto(86.87918179,6.92463107)(87.08007456,6.85621693)(87.33628116,6.78828969)
\curveto(87.59248777,6.72036245)(87.92731236,6.63351575)(88.34170897,6.52871344)
\curveto(88.46495766,6.50397107)(88.63576868,6.45884854)(88.85509608,6.39383275)
\curveto(89.07442348,6.32881696)(89.29181301,6.22547534)(89.50823857,6.08331106)
\curveto(89.7246542,5.94066982)(89.91098808,5.75047156)(90.06917809,5.51272622)
\curveto(90.2263942,5.27449398)(90.30500225,4.9697952)(90.30500225,4.59862987)
\curveto(90.30500225,4.29538184)(90.24580273,4.01397466)(90.12837761,3.75439841)
\curveto(90.01095248,3.49433525)(89.83625577,3.27018331)(89.60431729,3.0819227)
\curveto(89.37236888,2.89269824)(89.08510663,2.74617175)(88.74252061,2.64088254)
\curveto(88.39897062,2.53559333)(88.00107076,2.48319714)(87.55075892,2.48319714)
\curveto(87.18585258,2.48319714)(86.83162939,2.52831967)(86.48806946,2.61807783)
\curveto(86.14548344,2.70783598)(85.84172443,2.84902647)(85.57872037,3.04116239)
\curveto(85.31669021,3.2332983)(85.10803602,3.4783173)(84.95275778,3.77525553)
\curveto(84.79845345,4.07268065)(84.72469505,4.42492652)(84.7305186,4.83345382)
\lineto(86.14355551,4.83345382)
\curveto(86.14352569,4.61075263)(86.18331667,4.42201507)(86.2638626,4.26772802)
\closepath
}
}
{
\newrgbcolor{curcolor}{1 1 1}
\pscustom[linestyle=none,fillstyle=solid,fillcolor=curcolor]
{
\newpath
\moveto(94.46935083,9.25256993)
\lineto(96.94700512,2.63069743)
\lineto(95.43401376,2.63069743)
\lineto(94.93323773,4.10568041)
\lineto(92.45558344,4.10568041)
\lineto(91.93637277,2.63069743)
\lineto(90.47093381,2.63069743)
\lineto(92.97576801,9.25256993)
\lineto(94.46935083,9.25256993)
\closepath
\moveto(94.55280854,5.19249638)
\lineto(93.71819175,7.62038845)
\lineto(93.69975712,7.62038845)
\lineto(92.83602256,5.19249638)
\lineto(94.55280854,5.19249638)
\closepath
}
}
{
\newrgbcolor{curcolor}{1 1 1}
\pscustom[linestyle=none,fillstyle=solid,fillcolor=curcolor]
{
\newpath
\moveto(59.99659896,9.25256993)
\curveto(60.31200508,9.25256993)(60.60024123,9.22442921)(60.86033352,9.16912157)
\curveto(61.1204258,9.11332704)(61.34363888,9.02210819)(61.52997277,8.89547496)
\curveto(61.71533274,8.76884173)(61.85993777,8.60047439)(61.96184004,8.3903928)
\curveto(62.06374232,8.18031122)(62.11518041,7.92072502)(62.11518041,7.61166403)
\curveto(62.11518041,7.27785073)(62.03948413,7.00032881)(61.88711767,6.77761768)
\curveto(61.73572512,6.5549165)(61.51056423,6.37296569)(61.21360268,6.23081135)
\curveto(61.62314964,6.11339937)(61.92885646,5.90767006)(62.1307132,5.61365322)
\curveto(62.33256993,5.31963638)(62.43350824,4.96544292)(62.43350824,4.55108279)
\curveto(62.43350824,4.21726949)(62.36848517,3.92810175)(62.23843903,3.68355972)
\curveto(62.10839288,3.43950457)(61.93273221,3.240095)(61.71340481,3.08532105)
\curveto(61.4931035,2.93054711)(61.24174655,2.81603665)(60.96030785,2.74229644)
\curveto(60.67789526,2.6675725)(60.3886852,2.63070737)(60.09074974,2.63070737)
\lineto(56.87454745,2.63070737)
\lineto(56.87454745,9.25257987)
\lineto(59.99659896,9.25257987)
\lineto(59.99659896,9.25256993)
\closepath
\moveto(59.81026508,6.57432286)
\curveto(60.07035736,6.57432286)(60.28386121,6.6359403)(60.45176046,6.75918512)
\curveto(60.6186858,6.88290689)(60.7021435,7.08280337)(60.7021435,7.36033523)
\curveto(60.7021435,7.51462228)(60.67399963,7.64125551)(60.6186858,7.73975795)
\curveto(60.56239806,7.83873736)(60.48863966,7.91539896)(60.39547272,7.97119349)
\curveto(60.30230578,8.02650113)(60.19555385,8.0653238)(60.07520701,8.08666781)
\curveto(59.9538962,8.10849873)(59.82869971,8.11917074)(59.69865357,8.11917074)
\lineto(58.33414298,8.11917074)
\lineto(58.33414298,6.57432286)
\lineto(59.81026508,6.57432286)
\closepath
\moveto(59.89566065,3.76409662)
\curveto(60.03831787,3.76409662)(60.17418757,3.77768008)(60.30423371,3.8058208)
\curveto(60.43427985,3.83347462)(60.54976711,3.88005784)(60.64875761,3.94458673)
\curveto(60.7477481,4.00960252)(60.82635615,4.09790992)(60.88555567,4.20901209)
\curveto(60.94475519,4.32011427)(60.97387296,4.46227855)(60.97387296,4.635495)
\curveto(60.97387296,4.9751312)(60.87779424,5.21772565)(60.68563681,5.36279144)
\curveto(60.49347937,5.50785724)(60.23921064,5.58064352)(59.92380452,5.58064352)
\lineto(58.33414298,5.58064352)
\lineto(58.33414298,3.76409662)
\lineto(59.89566065,3.76409662)
\closepath
}
}
{
\newrgbcolor{curcolor}{1 1 1}
\pscustom[linestyle=none,fillstyle=solid,fillcolor=curcolor]
{
\newpath
\moveto(62.69066887,9.25256993)
\lineto(64.32399714,9.25256993)
\lineto(65.87484161,6.63739106)
\lineto(67.41597685,9.25256993)
\lineto(69.03960582,9.25256993)
\lineto(66.58136006,5.17211622)
\lineto(66.58136006,2.63068749)
\lineto(65.12174466,2.63068749)
\lineto(65.12174466,5.20899129)
\lineto(62.69066887,9.25256993)
\closepath
}
}
{
\newrgbcolor{curcolor}{1 1 1}
\pscustom[linestyle=none,fillstyle=solid,fillcolor=curcolor]
{
\newpath
\moveto(102.40308729,27.01924657)
\curveto(102.40697543,21.17415729)(97.67195262,16.43191423)(91.82573039,16.42755454)
\curveto(85.98047708,16.42315759)(81.23770285,21.15815934)(81.23285199,27.00370819)
\lineto(81.23285199,27.01924657)
\curveto(81.22896386,32.86479541)(85.96494938,37.60608829)(91.8102089,37.6109386)
\curveto(97.65643113,37.61529829)(102.39919916,32.87984317)(102.40308729,27.03474769)
\lineto(102.40308729,27.01924657)
\closepath
}
}
{
\newrgbcolor{curcolor}{0 0 0}
\pscustom[linestyle=none,fillstyle=solid,fillcolor=curcolor]
{
\newpath
\moveto(91.7422659,38.61383489)
\curveto(88.52993275,38.61383489)(85.81160404,37.49305309)(83.58627977,35.25194905)
\curveto(81.30270801,32.93271838)(80.16140969,30.18849871)(80.16140969,27.01924657)
\curveto(80.16140969,23.84999442)(81.30270801,21.12464811)(83.58627977,18.84476644)
\curveto(85.86886398,16.56435067)(88.58816162,15.42416763)(91.7422659,15.42416763)
\curveto(94.93517079,15.42416763)(97.70202667,16.5740575)(100.04479002,18.87388072)
\curveto(102.24975437,21.05721576)(103.35318373,23.77282419)(103.35318373,27.01924657)
\curveto(103.35318373,30.26517833)(102.23034474,33.01038545)(99.98559222,35.25194905)
\curveto(97.74086455,37.49305309)(94.99339966,38.61383489)(91.7422659,38.61383489)
\closepath
\moveto(91.77138344,36.52704027)
\curveto(94.40334088,36.52704027)(96.63838548,35.59984235)(98.47746754,33.74445905)
\curveto(100.33400274,31.90848321)(101.26277033,29.66692582)(101.26277033,27.01924657)
\curveto(101.26277033,24.35261944)(100.3543875,22.13967328)(98.53471506,20.38087388)
\curveto(96.61898207,18.48716631)(94.36452784,17.54056094)(91.77138965,17.54056094)
\curveto(89.17823283,17.54056094)(86.94320686,18.47795631)(85.06627447,20.3517596)
\curveto(83.18936073,22.2270037)(82.25089143,24.44870029)(82.25089143,27.01924036)
\curveto(82.25089143,29.58928981)(83.19906243,31.83133782)(85.09538581,33.74445284)
\curveto(86.9140831,35.59984235)(89.13940737,36.52704027)(91.77138344,36.52704027)
\closepath
}
}
{
\newrgbcolor{curcolor}{0 0 0}
\pscustom[linestyle=none,fillstyle=solid,fillcolor=curcolor]
{
\newpath
\moveto(86.60254777,28.65580077)
\curveto(87.06450924,31.57324668)(89.11903503,33.13263065)(91.69277601,33.13263065)
\curveto(95.39518198,33.13263065)(97.65059893,30.44663333)(97.65059893,26.86493095)
\curveto(97.65059893,23.37011184)(95.25056941,20.65500024)(91.63454714,20.65500024)
\curveto(89.14717743,20.65500024)(86.92087181,22.18576054)(86.51520136,25.18910228)
\lineto(89.43639014,25.18910228)
\curveto(89.52373034,23.62968105)(90.53595,23.08093151)(91.9819701,23.08093151)
\curveto(93.62987512,23.08093151)(94.70128637,24.61172908)(94.70128637,26.95180801)
\curveto(94.70128637,29.40635673)(93.7754442,30.70620254)(92.03923004,30.70620254)
\curveto(90.76691521,30.70620254)(89.6683367,30.24379597)(89.43639014,28.65580077)
\lineto(90.28652546,28.66016046)
\lineto(87.98646328,26.36132469)
\lineto(85.68737002,28.66016046)
\lineto(86.60254777,28.65580077)
\closepath
}
}
{
\newrgbcolor{curcolor}{1 1 1}
\pscustom[linestyle=none,fillstyle=solid,fillcolor=curcolor]
{
\newpath
\moveto(74.09879048,26.78297554)
\curveto(74.09879048,20.8525478)(69.29069042,16.04498624)(63.35959842,16.04498624)
\curveto(57.42850643,16.04498624)(52.62040637,20.8525478)(52.62040637,26.78297554)
\curveto(52.62040637,32.71340327)(57.42850643,37.52096484)(63.35959842,37.52096484)
\curveto(69.29069042,37.52096484)(74.09879048,32.71340327)(74.09879048,26.78297554)
\closepath
}
}
{
\newrgbcolor{curcolor}{0 0 0}
\pscustom[linestyle=none,fillstyle=solid,fillcolor=curcolor]
{
\newpath
\moveto(66.46783056,29.89012945)
\curveto(66.46783056,30.30399275)(66.132042,30.63877985)(65.71860935,30.63877985)
\lineto(60.97583071,30.63877985)
\curveto(60.56239806,30.63877985)(60.2266095,30.30400269)(60.2266095,29.89012945)
\lineto(60.2266095,25.14739508)
\lineto(61.5493813,25.14739508)
\lineto(61.5493813,19.53083779)
\lineto(65.14407491,19.53083779)
\lineto(65.14407491,25.14739508)
\lineto(66.46782062,25.14739508)
\lineto(66.46782062,29.89012945)
\lineto(66.46783056,29.89012945)
\closepath
}
}
{
\newrgbcolor{curcolor}{0 0 0}
\pscustom[linestyle=none,fillstyle=solid,fillcolor=curcolor]
{
\newpath
\moveto(64.96939285,32.88569708)
\curveto(64.96939285,31.98989868)(64.24312342,31.2637106)(63.34722469,31.2637106)
\curveto(62.45132595,31.2637106)(61.72505653,31.98989868)(61.72505653,32.88569708)
\curveto(61.72505653,33.78149547)(62.45132595,34.50768356)(63.34722469,34.50768356)
\curveto(64.24312342,34.50768356)(64.96939285,33.78149547)(64.96939285,32.88569708)
\closepath
}
}
{
\newrgbcolor{curcolor}{0 0 0}
\pscustom[linestyle=none,fillstyle=solid,fillcolor=curcolor]
{
\newpath
\moveto(63.33218413,38.61385121)
\curveto(60.12083148,38.61385121)(57.40104253,37.49354858)(55.17570916,35.25148263)
\curveto(52.89215017,32.93324198)(51.75085267,30.18852837)(51.75085267,27.01974646)
\curveto(51.75085267,23.85096455)(52.89215017,21.1251704)(55.17570916,18.84477862)
\curveto(57.45926814,16.56487374)(60.17858008,15.42467785)(63.33218413,15.42467785)
\curveto(66.5250922,15.42467785)(69.29293045,16.57409499)(71.63375108,18.87438003)
\curveto(73.84063988,21.0572531)(74.94312031,23.77286214)(74.94312031,27.0197564)
\curveto(74.94312031,30.26665066)(73.82123134,33.01040041)(71.57648944,35.25149257)
\curveto(69.33174753,37.49354858)(66.58331781,38.61385121)(63.33218413,38.61385121)
\closepath
\moveto(63.36129196,36.527533)
\curveto(65.99326052,36.527533)(68.22829319,35.59985319)(70.06639991,33.74448362)
\curveto(71.92488907,31.90900731)(72.8536467,29.66694136)(72.8536467,27.01974646)
\curveto(72.8536467,24.3531452)(71.94429761,22.13970686)(70.12463546,20.38137905)
\curveto(68.20889459,18.48768365)(65.95444345,17.54107443)(63.3613019,17.54107443)
\curveto(60.76816036,17.54107443)(58.53311775,18.4779755)(56.65668091,20.35275143)
\curveto(54.77878321,22.22704046)(53.84080329,24.44921315)(53.84080329,27.01974646)
\curveto(53.84080329,29.58980281)(54.78896946,31.83138186)(56.68530179,33.74448362)
\curveto(58.50448693,35.59985319)(60.73029731,36.527533)(63.36129196,36.527533)
\closepath
}
}
\end{pspicture}
 \\ {\tt cc-by-sa}

        \vspace{3pt}%LaTeX with PSTricks extensions
%%Creator: inkscape 0.91
%%Please note this file requires PSTricks extensions
\psset{xunit=.5pt,yunit=.5pt,runit=.5pt}
\begin{pspicture}(120,42)
{
\newrgbcolor{curcolor}{0.66666669 0.69803923 0.67058825}
\pscustom[linestyle=none,fillstyle=solid,fillcolor=curcolor]
{
\newpath
\moveto(3.13998263,41.49249256)
\lineto(116.49364354,41.29113536)
\curveto(118.07746894,41.29113536)(119.49243247,41.52596929)(119.49243247,38.13061021)
\lineto(119.3536521,0.80153365)
\lineto(0.27948713,0.80153365)
\lineto(0.27948713,38.26937617)
\curveto(0.27948713,39.94375912)(0.44156154,41.49249256)(3.13998263,41.49249256)
\closepath
}
}
{
\newrgbcolor{curcolor}{0 0 0}
\pscustom[linestyle=none,fillstyle=solid,fillcolor=curcolor]
{
\newpath
\moveto(117.75236611,41.99999937)
\lineto(2.24714734,41.99999937)
\curveto(1.00833004,41.99999937)(-0.00000153,40.99225941)(-0.00000153,39.75357076)
\lineto(-0.00000153,0.50701991)
\curveto(-0.00000153,0.22706344)(0.22757244,-0)(0.50755799,-0)
\lineto(119.4924424,-0)
\curveto(119.77242795,-0)(120.00000192,0.22707338)(120.00000192,0.50701991)
\lineto(120.00000192,39.75357076)
\curveto(119.99999199,40.99225941)(118.99167035,41.99999937)(117.75236611,41.99999937)
\closepath
\moveto(2.2471374,40.98498575)
\lineto(117.75235617,40.98498575)
\curveto(118.43217627,40.98498575)(118.98486301,40.4328433)(118.98486301,39.75357076)
\lineto(118.98486301,12.47762841)
\lineto(36.15962426,12.47762841)
\curveto(33.13366559,7.0066227)(27.30398565,3.29200751)(20.61348977,3.29200751)
\curveto(13.92008215,3.29200751)(8.09185312,7.00322435)(5.06735529,12.47762841)
\lineto(1.01510757,12.47762841)
\lineto(1.01510757,39.75357076)
\curveto(1.01511751,40.43283337)(1.56780425,40.98498575)(2.2471374,40.98498575)
\closepath
}
}
{
\newrgbcolor{curcolor}{1 1 1}
\pscustom[linestyle=none,fillstyle=solid,fillcolor=curcolor]
{
\newpath
\moveto(73.80897511,9.25257435)
\curveto(74.12437872,9.25257435)(74.41261258,9.22443363)(74.6736767,9.16912598)
\curveto(74.93376691,9.11381833)(75.15600432,9.02259946)(75.34233672,8.89547931)
\curveto(75.52769523,8.76933295)(75.67229911,8.60047868)(75.77420057,8.39088395)
\curveto(75.87610203,8.18031543)(75.92753971,7.92121608)(75.92753971,7.61166812)
\curveto(75.92753971,7.27785476)(75.85184404,7.00033278)(75.69947879,6.77810851)
\curveto(75.54808744,6.55492038)(75.32292834,6.37345643)(75.02596915,6.23081516)
\curveto(75.43551286,6.11340316)(75.74121725,5.90767381)(75.94307238,5.61365691)
\curveto(76.14492751,5.31964001)(76.24586502,4.96544648)(76.24586502,4.55108626)
\curveto(76.24586502,4.2172729)(76.18084246,3.92810511)(76.05079735,3.68356302)
\curveto(75.92075224,3.43999472)(75.74509296,3.24009821)(75.52576731,3.08581113)
\curveto(75.30546775,2.93055026)(75.0541128,2.81604971)(74.77267634,2.74229955)
\curveto(74.49026599,2.6675756)(74.20105823,2.63071046)(73.90312514,2.63071046)
\lineto(70.68692854,2.63071046)
\lineto(70.68692854,9.25258429)
\lineto(73.80897511,9.25258429)
\lineto(73.80897511,9.25257435)
\closepath
\moveto(73.62264271,6.57431681)
\curveto(73.88273292,6.57431681)(74.09623507,6.63642116)(74.26413299,6.75965606)
\curveto(74.431057,6.88289097)(74.51451404,7.08279742)(74.51451404,7.3603194)
\curveto(74.51451404,7.51460648)(74.4863704,7.64172663)(74.431057,7.7397422)
\curveto(74.37476971,7.83872163)(74.3010119,7.91538324)(74.2078457,7.97166468)
\curveto(74.1146795,8.02697233)(74.00792842,8.065795)(73.88758254,8.08713903)
\curveto(73.76723665,8.10848305)(73.64107719,8.119165)(73.51200598,8.119165)
\lineto(72.14654228,8.119165)
\lineto(72.14654228,6.57431681)
\lineto(73.62264271,6.57431681)
\closepath
\moveto(73.7080376,3.76409994)
\curveto(73.85069369,3.76409994)(73.98656231,3.7776834)(74.11660741,3.80582412)
\curveto(74.24665252,3.83396485)(74.36116496,3.88054808)(74.46112857,3.94459008)
\curveto(74.56011828,4.00960589)(74.63872571,4.0979133)(74.69792475,4.2095024)
\curveto(74.7571238,4.32012764)(74.78624134,4.46276891)(74.78624134,4.63549849)
\curveto(74.78624134,4.97513476)(74.69016339,5.21772926)(74.49800748,5.36328198)
\curveto(74.30585157,5.50787085)(74.05158486,5.58064721)(73.73618125,5.58064721)
\lineto(72.14653234,5.58064721)
\lineto(72.14653234,3.76409994)
\lineto(73.7080376,3.76409994)
\closepath
}
}
{
\newrgbcolor{curcolor}{1 1 1}
\pscustom[linestyle=none,fillstyle=solid,fillcolor=curcolor]
{
\newpath
\moveto(76.5030236,9.25257435)
\lineto(78.13633889,9.25257435)
\lineto(79.68717103,6.63739495)
\lineto(81.22829401,9.25257435)
\lineto(82.85191008,9.25257435)
\lineto(80.39368387,5.17211982)
\lineto(80.39368387,2.63069058)
\lineto(78.93408007,2.63069058)
\lineto(78.93408007,5.2089949)
\lineto(76.5030236,9.25257435)
\closepath
}
}
{
\newrgbcolor{curcolor}{1 1 1}
\pscustom[linestyle=none,fillstyle=solid,fillcolor=curcolor]
{
\newpath
\moveto(34.25408708,22.39392648)
\curveto(34.25892764,14.8642521)(28.15752292,8.75717503)(20.627561,8.75230894)
\curveto(13.09759908,8.7474949)(6.98890748,14.84778895)(6.98453535,22.37741127)
\lineto(6.98453535,22.39392648)
\curveto(6.98016323,29.92360085)(13.08109951,36.03067793)(20.61106143,36.03506694)
\curveto(28.14200361,36.03988099)(34.24971495,29.93958694)(34.25408708,22.40991257)
\lineto(34.25408708,22.39392648)
\closepath
}
}
{
\newrgbcolor{curcolor}{0 0 0}
\pscustom[linestyle=none,fillstyle=solid,fillcolor=curcolor]
{
\newpath
\moveto(31.70315158,33.49848285)
\curveto(34.72669298,30.47525543)(36.23870224,26.77277961)(36.23870224,22.39392648)
\curveto(36.23870224,18.0141192)(34.75289971,14.35141345)(31.78129464,11.40584391)
\curveto(28.62771721,8.30351873)(24.90055861,6.75285924)(20.59987087,6.75285924)
\curveto(16.35112815,6.75285924)(12.68851911,8.29091548)(9.61305872,11.36698461)
\curveto(6.53712121,14.44262871)(4.99938234,18.11793772)(4.99938234,22.39392648)
\curveto(4.99938234,26.6689611)(6.53712121,30.37047411)(9.61305872,33.49848285)
\curveto(12.6103934,36.5226644)(16.27298509,38.03451665)(20.59987087,38.03451665)
\curveto(24.97919614,38.03452532)(28.67966223,36.52267308)(31.70315158,33.49848285)
\closepath
\moveto(11.64864732,31.46458022)
\curveto(9.09241149,28.8828829)(7.81477069,25.85870135)(7.81477069,22.38959817)
\curveto(7.81477069,18.92144913)(9.07980693,15.92294247)(11.60935891,13.39413024)
\curveto(14.13937934,10.86437254)(17.1512791,9.59996209)(20.64599506,9.59996209)
\curveto(24.14071102,9.59996209)(27.17828833,10.87696712)(29.76025377,13.43246043)
\curveto(32.21170607,15.80604364)(33.437905,18.79041176)(33.437905,22.38959817)
\curveto(33.437905,25.96156573)(32.19179729,28.99347577)(29.70105659,31.48395782)
\curveto(27.210793,33.9739628)(24.19260398,35.21946404)(20.64599506,35.21946404)
\curveto(17.09936879,35.21945537)(14.0996225,33.96765684)(11.64864732,31.46458022)
\closepath
\moveto(18.37555221,23.9154762)
\curveto(17.98493232,24.76698925)(17.40023866,25.19301034)(16.62043891,25.19301034)
\curveto(15.24185754,25.19301034)(14.55283144,24.2648369)(14.55283144,22.40944417)
\curveto(14.55283144,20.55362642)(15.24185754,19.62640712)(16.62043891,19.62640712)
\curveto(17.53076047,19.62640712)(18.18097529,20.0781031)(18.57111806,20.98345537)
\lineto(20.48197964,19.96602698)
\curveto(19.57118096,18.34791884)(18.20475306,17.5386566)(16.38268726,17.5386566)
\curveto(14.97744807,17.5386566)(13.85168657,17.96949173)(13.00638302,18.83021652)
\curveto(12.15963077,19.69189545)(11.7374778,20.87963685)(11.7374778,22.39392648)
\curveto(11.7374778,23.88153503)(12.17320691,25.06250208)(13.0451943,25.93725265)
\curveto(13.9171817,26.81205526)(15.00312563,27.24924839)(16.30503866,27.24924839)
\curveto(18.23095978,27.24924839)(19.60999225,26.4903818)(20.44364547,24.97419259)
\lineto(18.37555221,23.9154762)
\closepath
\moveto(27.36562175,23.9154762)
\curveto(26.97449872,24.76698925)(26.40145538,25.19301034)(25.64590185,25.19301034)
\curveto(24.2396824,25.19301034)(23.5360652,24.2648369)(23.5360652,22.40944417)
\curveto(23.5360652,20.55362642)(24.2396824,19.62640712)(25.64590185,19.62640712)
\curveto(26.55770681,19.62640712)(27.19624528,20.0781031)(27.56068446,20.98345537)
\lineto(29.51426968,19.96602698)
\curveto(28.60490236,18.34791884)(27.24040027,17.5386566)(25.42172633,17.5386566)
\curveto(24.01841296,17.5386566)(22.8950891,17.96949173)(22.05027134,18.83021652)
\curveto(21.2069283,19.69189545)(20.78430689,20.87963685)(20.78430689,22.39392648)
\curveto(20.78430689,23.88153503)(21.21323492,25.06250208)(22.07065724,25.93725265)
\curveto(22.92760244,26.81205526)(24.01842164,27.24924839)(25.34411244,27.24924839)
\curveto(27.26662435,27.24924839)(28.64372232,26.4903818)(29.47446079,24.97419259)
\lineto(27.36562175,23.9154762)
\closepath
}
}
{
\newrgbcolor{curcolor}{1 1 1}
\pscustom[linestyle=none,fillstyle=solid,fillcolor=curcolor]
{
\newpath
\moveto(87.0065449,26.78345945)
\curveto(87.0065449,20.85289916)(82.19837656,16.04523013)(76.26720034,16.04523013)
\curveto(70.33602411,16.04523013)(65.52785577,20.85289916)(65.52785577,26.78345945)
\curveto(65.52785577,32.71401974)(70.33602411,37.52168877)(76.26720034,37.52168877)
\curveto(82.19837656,37.52168877)(87.0065449,32.71401974)(87.0065449,26.78345945)
\closepath
}
}
{
\newrgbcolor{curcolor}{0 0 0}
\pscustom[linestyle=none,fillstyle=solid,fillcolor=curcolor]
{
\newpath
\moveto(79.37467918,29.89013802)
\curveto(79.37467918,30.30400141)(79.03889328,30.63927547)(78.62643782,30.63927547)
\lineto(73.88272299,30.63927547)
\curveto(73.47026752,30.63927547)(73.13448163,30.30401135)(73.13448163,29.89013802)
\lineto(73.13448163,25.14740271)
\lineto(74.45724292,25.14740271)
\lineto(74.45724292,19.53132124)
\lineto(78.05190795,19.53132124)
\lineto(78.05190795,25.14740271)
\lineto(79.37466924,25.14740271)
\lineto(79.37466924,29.89013802)
\lineto(79.37467918,29.89013802)
\closepath
}
}
{
\newrgbcolor{curcolor}{0 0 0}
\pscustom[linestyle=none,fillstyle=solid,fillcolor=curcolor]
{
\newpath
\moveto(77.87722624,32.88570367)
\curveto(77.87722624,31.98963621)(77.1507446,31.26323001)(76.25458408,31.26323001)
\curveto(75.35842356,31.26323001)(74.63194192,31.98963621)(74.63194192,32.88570367)
\curveto(74.63194192,33.78177112)(75.35842356,34.50817732)(76.25458408,34.50817732)
\curveto(77.1507446,34.50817732)(77.87722624,33.78177112)(77.87722624,32.88570367)
\closepath
}
}
{
\newrgbcolor{curcolor}{0 0 0}
\pscustom[linestyle=none,fillstyle=solid,fillcolor=curcolor]
{
\newpath
\moveto(76.24003157,38.6143385)
\curveto(73.0277405,38.6143385)(70.30845017,37.49354875)(68.08313449,35.25196925)
\curveto(65.79959366,32.93324124)(64.65830523,30.18852707)(64.65830523,27.01974452)
\curveto(64.65830523,23.85096198)(65.79959366,21.12565418)(68.08313449,18.84526194)
\curveto(70.36667532,16.5653566)(73.08596564,15.42516048)(76.24003157,15.42516048)
\curveto(79.43291426,15.42516048)(82.20073051,16.57409095)(84.54153253,18.87437646)
\curveto(86.74743983,21.05773686)(87.85087546,23.77285955)(87.85087546,27.01975446)
\curveto(87.85087546,30.26664937)(86.72803144,33.01039968)(84.48427134,35.25197918)
\curveto(82.23857338,37.49354875)(79.49113941,38.6143385)(76.24003157,38.6143385)
\closepath
\moveto(76.26913918,36.52801987)
\curveto(78.90108681,36.52801987)(81.13610171,35.59985297)(82.97419382,33.74448303)
\curveto(84.83169431,31.90900636)(85.76044456,29.66742686)(85.76044456,27.01974452)
\curveto(85.76044456,24.35314272)(84.8511027,22.14019084)(83.03241897,20.38185273)
\curveto(81.11571943,18.48767006)(78.86226011,17.54106065)(76.26913918,17.54106065)
\curveto(73.67504435,17.54106065)(71.44001951,18.47796191)(69.56408453,20.35273822)
\curveto(67.68620176,22.22751452)(66.74871625,24.44968766)(66.74871625,27.01973459)
\curveto(66.74871625,29.59026841)(67.69687488,31.83137095)(69.59319213,33.7444731)
\curveto(71.4118858,35.59985297)(73.63719154,36.52801987)(76.26913918,36.52801987)
\closepath
}
}
\end{pspicture}
 \\ {\tt cc-by}

        \column{0.5\textwidth}
        Specify licence by:
\begin{lstlisting}[language=TeX]
\def\licence{cc-by-nc-sa}
\end{lstlisting}

    or omit the command entirely if you do not want to specify one.

    \end{columns}

%    If you do not want to specify licence, do not \lstinline{\def\licence{}}.

\end{frame}

% -----------------------------------------------------------------------------

\begin{frame}[fragile]
    \frametitle{Sections}

    Sections have a title slide. To disable the title slide do:

    \begin{lstlisting}[language=TeX]
        \AtBeginSection{}
    \end{lstlisting}

    You can generate an outline of the slides by:

    \begin{lstlisting}[language=TeX]
        \outline{Title of your outline slide}
        \outlinecurrent{Outline with highlighted current section}
    \end{lstlisting}

\end{frame}


% -----------------------------------------------------------------------------

\begin{frame}[fragile]
    \frametitle{Equations}

    \centering Serif font also for equations

    $$i\hbar\frac{\partial}{\partial t} \Psi(\mathbf{r},t) = \left [ \frac{-\hbar^2}{2\mu}\nabla^2 + V(\mathbf{r},t)\right ] \Psi(\mathbf{r},t) ]$$

    \begin{lstlisting}[language=TeX]
$$ i\hbar\frac{\partial}{\partial t} \Psi(\mathbf{r},t) = \left [ \frac{-\hbar^2}{2\mu}\nabla^2 + V(\mathbf{r},t)\right ] \Psi(\mathbf{r},t) ]$$
    \end{lstlisting}

    Equations may break compilation with Mik\TeX. If this is your case use the
    package with option \lstinline[language=TeX]{miktex}.

\begin{lstlisting}[language=TeX]
    \usepackage[miktex]{ufallslides}
\end{lstlisting}

\end{frame}

% -----------------------------------------------------------------------------

\begin{frame}
    \frametitle{Color Palette}

    Named colors which are the same as the boostrap colors at ÚFAL web page.\\[15pt]

    \begin{center}
    \scalebox{1.2}{%
    \begin{tabular}{cc}

    \colorbox{ufal}{\bf ufal} &     \color{ufal} \bf ufal \\
    \colorbox{ufalblue}{\bf ufalblue} &     \color{ufalblue} \bf ufalblue \\
    \colorbox{ufalred}{\bf ufalred} &     \color{ufalred} \bf ufalred \\
    \colorbox{ufallightblue}{\bf ufallightblue} &     \color{ufallightblue} \bf ufallightblue \\
    \colorbox{ufalgreen}{\bf ufalgreen} &     \color{ufalgreen} \bf ufalgreen \\
    \end{tabular}}
    \end{center}

\end{frame}


% -----------------------------------------------------------------------------

\begin{frame}[fragile]
    \frametitle{Labels from the web}

    \slidesbox{Slides}
    \readingbox{Reading}
    \hwbox{Homework}
    \questionbox{Question}
    \timebox{1 h}
    \calendarbox{Oct 15}
    \pointsbox{100 points}

    \begin{lstlisting}[language=TeX]
\slidesbox{Slides}
\readingbox{Reading}
\hwbox{Homework}
\questionbox{Question}
\timebox{1 h}
\calendarbox{Oct 15}
\pointsbox{100 points}
\slidesbox{Slides}
    \end{lstlisting}
\end{frame}

% -----------------------------------------------------------------------------
\section{Another section}
% -----------------------------------------------------------------------------

\begin{frame}[fragile]
    \frametitle{Code listings}

    This code snippet:
    \begin{lstlisting}[language=Python]
print("Hello, ÚFAL.")
x = 3 + 5
    \end{lstlisting}
%
    is produced by putting the code between {\tt  \textbackslash
    begin\{lstlisting\}} and {\tt \textbackslash end\{lstlisting\}}

    Inline code (\lstinline[language=Python]{import numpy as np}) can be
    inserted with the \lstinline[language=TeX]{\lstinline} command.

    Do not forget to start the frame with \lstinline[language=TeX]{fragile}
    option to beginning of the frame.

\end{frame}

% -----------------------------------------------------------------------------

\begin{frame}[fragile]
    \frametitle{References}

    Full citation on slide: \\ {\tiny \bibentry{helcl2017neural}}

\begin{lstlisting}[language=TeX]
Full citation: \\ {\tiny \bibentry{helcl2017neural}}
\end{lstlisting}

    \citet{sennrich2016neural} uses attention \citep{bahdanau2015neural}.

    \begin{lstlisting}[language=TeX]
\citet{sennrich2016neural} uses attention \citep{bahdanau2015neural}.
    \end{lstlisting}

    \tiny
    \citet{snover2006study,lu2016knowing,tu2016modeling,feng2016improving,zhang2016recurrent,alkhouli2016alignment,graves2014neural,specia2016shared,elliott2016multi30k}

If you prefer managing bibliography on your own, use the package with option
    \lstinline[language=TeX]{custombibset}.

If you want bibliography to have its own section, use 
    \lstinline[language=TeX]{\section{References}} before the 
	\lstinline[language=TeX]{\references} command. 

\end{frame}

% -----------------------------------------------------------------------------

\begin{frame}[fragile]
    \frametitle{Summary, outline, references}

    The summary slide can be inserted by calling:

    \begin{lstlisting}[language=TeX]
\summary{Name of the summary slide}{%
    Content of the summary slide
}
    \end{lstlisting}

    Outline with optionally highlighted current section can be inserted by:
    \begin{lstlisting}[language=TeX]
\outline{Outline slide title}
\outlinecurrent{Whatever outline title you wish}
    \end{lstlisting}

    To show the references do:

    \begin{lstlisting}[language=TeX]
\references{pathToYourBibFile.bib}
    \end{lstlisting}

\end{frame}

% -----------------------------------------------------------------------------

\begin{frame}
    \frametitle{Itemize}

    \begin{itemize}[<+->]

        \item All human beings are born free and equal in dignity and rights.

        \item They are endowed with reason and conscience and should act
            towards one another in a spirit of brotherhood.

        \item Everyone is entitled to all the rights and freedoms set forth in
            this Declaration, without distinction of any kind, such as race,
            colour, sex, language, religion, political or other opinion,
            national or social origin, property, birth or other status.

    \end{itemize}

\end{frame}

% -----------------------------------------------------------------------------

\begin{frame}
    \frametitle{Enumerate}

    \begin{enumerate}[<+->]

        \item All human beings are born free and equal in dignity and rights.

        \item They are endowed with reason and conscience and should act
            towards one another in a spirit of brotherhood.

        \item Everyone is entitled to all the rights and freedoms set forth in
            this Declaration, without distinction of any kind, such as race,
            colour, sex, language, religion, political or other opinion,
            national or social origin, property, birth or other status.

    \end{enumerate}

\end{frame}

% -----------------------------------------------------------------------------

\begin{frame}[allowframebreaks]
    \frametitle{What happens with too much content?}

    Babakotia, an extinct genus of sloth lemurs, lived in the northern part of
    Madagascar. The name comes from the Malagasy word for the indri, to which
    all sloth lemurs are closely related. Its morphological traits show
    intermediate stages between the slow-moving smaller sloth lemurs and the
    suspensory large sloth lemurs, and suggest a close relationship between
    both groups and the extinct monkey lemurs. All sloth lemurs share many
    traits with living sloths, demonstrating convergent evolution. Babakotia
    had long forearms, curved digits, and highly mobile hip and ankle joints.
    It shared its range with other sloth lemurs, including Palaeopropithecus
    ingens and Mesopropithecus dolichobrachion. It was primarily a leaf-eater,
    though it also ate fruit and hard seeds. It is known only from subfossil
    remains and may have died out shortly after the arrival of humans on the
    island, but not enough radiocarbon dating has been done with this species
    to know for certain. Babakotia radofilai is the sole member of the genus
    Babakotia and belongs to the family Palaeopropithecidae, which includes
    three other genera of sloth lemurs: Palaeopropithecus, Archaeoindris, and
    Mesopropithecus. This family in turn belongs to the infraorder
    Lemuriformes, which includes all the Malagasy lemurs.[5][2]

\end{frame}


% -----------------------------------------------------------------------------

\summary{Summary}{%
    \begin{enumerate}

        \item This template is tremendous.

        \item If you don't use the template you will be very very sad.

        \item Believe me. It's tremendous.

    \end{enumerate}
}
% -----------------------------------------------------------------------------

%Use this if you want references as a section,
%i.e. in the ouline and with a sectionpage
%\section{References}

\references{references.bib}

\end{document}
