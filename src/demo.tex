\documentclass[handout,aspectratio=169]{beamer}

\usepackage{polyglossia}
\setmainlanguage{english}
\usepackage[nounicodemath]{ufalslides}
% Supported options:
%   nounicodemath (if you are struggling with a strange compilation issue)
%   custombib (if natbib chucks for you)
%   miktex    (if you are compiling with miktex; implies nounicodemath)
%   czech (selfexplanatory)
\usepackage{xcolor}
\usepackage{textcomp}

% %%%%%%%%%%%%%%%%%%%%%%%%%%%%%%%%%%%%%%%%%%%%%%%%%%%%%%%%%%%%%%%%%%%%%%%%%%%%%
\def\course{NPFL000 Name of the course}
\def\courseurl{https://ufal.cz/courses/npfl000}
\def\title{\LaTeX~template for LangTech courses taught at ÚFAL}
\def\author{Jindřich Libovický}
\def\date{September 7, 2018}
\def\licence{cc-by-nc-sa}
%\def\langtech{}
%\def\lindat{}
\def\shownavigation{}
\def\showtitleoneveryframe{}
% %%%%%%%%%%%%%%%%%%%%%%%%%%%%%%%%%%%%%%%%%%%%%%%%%%%%%%%%%%%%%%%%%%%%%%%%%%%%%

\begin{document}

\maketitle

\outline{Outline}

\section{A section}
\outlinecurrent{Outline}

\begin{frame}[fragile]
    \frametitle{How to use the template}

    \begin{lstlisting}[language=TeX]
\documentclass[handout,aspectratio=169]{beamer}
\usepackage[english]{babel}
\usepackage{ufalslides}
\end{lstlisting}

    \begin{itemize}
        \item Use \lstinline{handout} option if you want to generate a handout without
    animations.

        \item Before you begin document, define what you want to
            appear in the title slide (see the next slide for more info).
    \end{itemize}

\end{frame}


\begin{frame}[fragile]
    \frametitle{Content of the title page}

1. Define the content of the title page
    \begin{lstlisting}[language=TeX]
\def\course{NPFL116 Compendium of Neural Machine Translation}
\def\courseurl{https://ufal.cz/courses/npfl000}
\def\title{Attention Mechanism}
\def\subtitle{How to attend with a mechanism}
\def\author{Jindřich Libovický, Jindřich Helcl} \def\date{March 1, 2017}
\def\licence{cc-by-nc-sa}
%\def\langtech{} % shows the LangTech and the EU logo
\def\lindat{} % shows the Lindat logo
\def\shownavigation{} % shows the navigation links in the bottom line
\end{lstlisting}

    {\tt  \textbackslash course} and {\tt  \textbackslash subtitle} are optional, others must be at least
    an empty string

2. Generate the title slide using after beginning of the document by calling
\begin{lstlisting}[language=TeX]
\maketitle
\end{lstlisting}

    \vspace{5pt}

    Hint: Don't use \textbf{ř} in your code snippets, it will break.

\end{frame}

% -----------------------------------------------------------------------------

\begin{frame}[fragile]
    \frametitle{Licence}

    \begin{columns}
        \column{0.25\textwidth}
        \centering

        
\definecolor{caab2ab}{RGB}{170,178,171}


\def \globalscale {0.550000}
\begin{tikzpicture}[y=1cm, x=1cm, yscale=\globalscale,xscale=\globalscale, every node/.append style={scale=\globalscale}, inner sep=0pt, outer sep=0pt]
  \begin{scope}[cm={ 0.9938,-0.0,-0.0,0.9937,(-11.5656, 1.9765)}]
    \path[fill=caab2ab] (11.7287, -0.8828) -- (14.7466, -0.8882).. controls (14.7888, -0.8882) and (14.8264, -0.8819) .. (14.8264, -0.9723) -- (14.8228, -1.9663) -- (11.6525, -1.9663) -- (11.6525, -0.9686).. controls (11.6525, -0.9241) and (11.6568, -0.8828) .. (11.7287, -0.8828) -- cycle;



    \path[fill] (14.773, -0.8708) -- (11.6978, -0.8708).. controls (11.6648, -0.8708) and (11.6379, -0.8976) .. (11.6379, -0.9306) -- (11.6379, -1.9756).. controls (11.6379, -1.9831) and (11.644, -1.9891) .. (11.6514, -1.9891) -- (14.8193, -1.9891).. controls (14.8268, -1.9891) and (14.8328, -1.9831) .. (14.8328, -1.9756) -- (14.8328, -0.9306).. controls (14.8328, -0.8976) and (14.806, -0.8708) .. (14.773, -0.8708) -- cycle(11.6978, -0.8978) -- (14.773, -0.8978).. controls (14.7911, -0.8978) and (14.8058, -0.9125) .. (14.8058, -0.9306).. controls (14.8058, -0.9306) and (14.8058, -1.3512) .. (14.8058, -1.6554) -- (12.6078, -1.6554).. controls (12.5272, -1.8011) and (12.372, -1.9) .. (12.1939, -1.9).. controls (12.0157, -1.9) and (11.8605, -1.8012) .. (11.78, -1.6554) -- (11.665, -1.6554).. controls (11.665, -1.3512) and (11.665, -0.9306) .. (11.665, -0.9306).. controls (11.665, -0.9125) and (11.6797, -0.8978) .. (11.6978, -0.8978) -- cycle;



    \begin{scope}[cm={ 0.8729,-0.0,-0.0,0.8729,(1.3262, -3.648)}]
      \path[fill=white] (12.8658, 2.5852).. controls (12.8659, 2.3555) and (12.6798, 2.1692) .. (12.4502, 2.169).. controls (12.2205, 2.1689) and (12.0342, 2.355) .. (12.0341, 2.5847).. controls (12.0341, 2.5848) and (12.0341, 2.585) .. (12.0341, 2.5852).. controls (12.0339, 2.8148) and (12.22, 3.0011) .. (12.4497, 3.0013).. controls (12.6794, 3.0014) and (12.8657, 2.8153) .. (12.8658, 2.5857).. controls (12.8658, 2.5855) and (12.8658, 2.5853) .. (12.8658, 2.5852) -- cycle;



      \begin{scope}[shift={(-7.6627, -2.6211)}]
        \path[fill] (20.4507, 5.545).. controls (20.543, 5.4528) and (20.5891, 5.3398) .. (20.5891, 5.2063).. controls (20.5891, 5.0727) and (20.5437, 4.9609) .. (20.4531, 4.8711).. controls (20.3569, 4.7765) and (20.2433, 4.7292) .. (20.1121, 4.7292).. controls (19.9825, 4.7292) and (19.8708, 4.7761) .. (19.777, 4.8699).. controls (19.6832, 4.9637) and (19.6363, 5.0758) .. (19.6363, 5.2063).. controls (19.6363, 5.3367) and (19.6832, 5.4496) .. (19.777, 5.545).. controls (19.8684, 5.6372) and (19.9801, 5.6833) .. (20.1121, 5.6833).. controls (20.2457, 5.6833) and (20.3585, 5.6372) .. (20.4507, 5.545) -- cycle(19.8391, 5.4829).. controls (19.7611, 5.4042) and (19.7221, 5.3119) .. (19.7221, 5.2061).. controls (19.7221, 5.1003) and (19.7607, 5.0089) .. (19.8379, 4.9317).. controls (19.9151, 4.8546) and (20.0069, 4.816) .. (20.1135, 4.816).. controls (20.2201, 4.816) and (20.3127, 4.855) .. (20.3915, 4.9329).. controls (20.4663, 5.0053) and (20.5036, 5.0964) .. (20.5036, 5.2061).. controls (20.5036, 5.3151) and (20.4656, 5.4076) .. (20.3897, 5.4835).. controls (20.3137, 5.5595) and (20.2216, 5.5975) .. (20.1135, 5.5975).. controls (20.0053, 5.5975) and (19.9138, 5.5593) .. (19.8391, 5.4829) -- cycle(20.0442, 5.2527).. controls (20.0323, 5.2786) and (20.0145, 5.2916) .. (19.9907, 5.2916).. controls (19.9487, 5.2916) and (19.9277, 5.2633) .. (19.9277, 5.2067).. controls (19.9277, 5.1501) and (19.9487, 5.1218) .. (19.9907, 5.1218).. controls (20.0185, 5.1218) and (20.0383, 5.1356) .. (20.0502, 5.1632) -- (20.1085, 5.1322).. controls (20.0807, 5.0828) and (20.039, 5.0581) .. (19.9835, 5.0581).. controls (19.9406, 5.0581) and (19.9063, 5.0713) .. (19.8805, 5.0976).. controls (19.8547, 5.1238) and (19.8418, 5.1601) .. (19.8418, 5.2063).. controls (19.8418, 5.2516) and (19.8551, 5.2877) .. (19.8817, 5.3143).. controls (19.9082, 5.341) and (19.9414, 5.3544) .. (19.9811, 5.3544).. controls (20.0398, 5.3544) and (20.0819, 5.3312) .. (20.1073, 5.285) -- (20.0442, 5.2527) -- cycle(20.3184, 5.2527).. controls (20.3065, 5.2786) and (20.289, 5.2916) .. (20.266, 5.2916).. controls (20.2231, 5.2916) and (20.2016, 5.2633) .. (20.2016, 5.2067).. controls (20.2016, 5.1501) and (20.2231, 5.1218) .. (20.266, 5.1218).. controls (20.2938, 5.1218) and (20.3133, 5.1356) .. (20.3244, 5.1632) -- (20.384, 5.1322).. controls (20.3562, 5.0828) and (20.3146, 5.0581) .. (20.2591, 5.0581).. controls (20.2163, 5.0581) and (20.1821, 5.0713) .. (20.1563, 5.0976).. controls (20.1306, 5.1238) and (20.1177, 5.1601) .. (20.1177, 5.2063).. controls (20.1177, 5.2516) and (20.1308, 5.2877) .. (20.1569, 5.3143).. controls (20.1831, 5.341) and (20.2163, 5.3544) .. (20.2568, 5.3544).. controls (20.3154, 5.3544) and (20.3574, 5.3312) .. (20.3828, 5.285) -- (20.3184, 5.2527) -- cycle;



      \end{scope}
    \end{scope}
    \path[fill=white] (13.0161, -1.2761) circle (0.2859cm);



    \path[fill] (13.0988, -1.1934).. controls (13.0988, -1.1824) and (13.0899, -1.1735) .. (13.0789, -1.1735) -- (12.9526, -1.1735).. controls (12.9416, -1.1735) and (12.9327, -1.1824) .. (12.9327, -1.1934) -- (12.9327, -1.3197) -- (12.9679, -1.3197) -- (12.9679, -1.4692) -- (13.0636, -1.4692) -- (13.0636, -1.3197) -- (13.0988, -1.3197) -- (13.0988, -1.1934) -- (13.0988, -1.1934) -- cycle;



    \path[fill] (13.0157, -1.1136) circle (0.0432cm);



    \path[fill,even odd rule] (13.0153, -0.9611).. controls (12.9298, -0.9611) and (12.8574, -0.9909) .. (12.7982, -1.0506).. controls (12.7374, -1.1124) and (12.707, -1.1855) .. (12.707, -1.2698).. controls (12.707, -1.3542) and (12.7374, -1.4268) .. (12.7982, -1.4875).. controls (12.859, -1.5482) and (12.9314, -1.5786) .. (13.0153, -1.5786).. controls (13.1004, -1.5786) and (13.174, -1.548) .. (13.2364, -1.4867).. controls (13.2951, -1.4286) and (13.3245, -1.3563) .. (13.3245, -1.2698).. controls (13.3245, -1.1834) and (13.2946, -1.1103) .. (13.2348, -1.0506).. controls (13.1751, -0.9909) and (13.1019, -0.9611) .. (13.0153, -0.9611) -- cycle(13.0161, -1.0167).. controls (13.0862, -1.0167) and (13.1457, -1.0414) .. (13.1946, -1.0908).. controls (13.2441, -1.1396) and (13.2688, -1.1993) .. (13.2688, -1.2698).. controls (13.2688, -1.3408) and (13.2446, -1.3998) .. (13.1962, -1.4466).. controls (13.1452, -1.497) and (13.0852, -1.5222) .. (13.0161, -1.5222).. controls (12.9471, -1.5222) and (12.8876, -1.4973) .. (12.8376, -1.4474).. controls (12.7876, -1.3974) and (12.7626, -1.3383) .. (12.7626, -1.2698).. controls (12.7626, -1.2014) and (12.7879, -1.1417) .. (12.8384, -1.0908).. controls (12.8868, -1.0414) and (12.9461, -1.0167) .. (13.0161, -1.0167) -- cycle;



    \path[fill=white] (12.9266, -1.7427).. controls (12.935, -1.7427) and (12.9426, -1.7435) .. (12.9496, -1.745).. controls (12.9565, -1.7464) and (12.9624, -1.7489) .. (12.9674, -1.7522).. controls (12.9723, -1.7556) and (12.9761, -1.7601) .. (12.9789, -1.7657).. controls (12.9816, -1.7713) and (12.983, -1.7782) .. (12.983, -1.7864).. controls (12.983, -1.7953) and (12.981, -1.8027) .. (12.9769, -1.8086).. controls (12.9729, -1.8146) and (12.9669, -1.8194) .. (12.959, -1.8232).. controls (12.9699, -1.8263) and (12.978, -1.8318) .. (12.9834, -1.8396).. controls (12.9888, -1.8475) and (12.9914, -1.8569) .. (12.9914, -1.8679).. controls (12.9914, -1.8768) and (12.9897, -1.8845) .. (12.9862, -1.891).. controls (12.9828, -1.8975) and (12.9781, -1.9028) .. (12.9722, -1.9069).. controls (12.9664, -1.9111) and (12.9597, -1.9141) .. (12.9522, -1.9161).. controls (12.9447, -1.9181) and (12.937, -1.9191) .. (12.9291, -1.9191) -- (12.8434, -1.9191) -- (12.8434, -1.7427) -- (12.9266, -1.7427) -- (12.9266, -1.7427) -- cycle(12.9216, -1.814).. controls (12.9285, -1.814) and (12.9342, -1.8124) .. (12.9387, -1.8091).. controls (12.9431, -1.8058) and (12.9453, -1.8005) .. (12.9453, -1.7931).. controls (12.9453, -1.789) and (12.9446, -1.7856) .. (12.9431, -1.783).. controls (12.9416, -1.7804) and (12.9397, -1.7783) .. (12.9372, -1.7768).. controls (12.9347, -1.7754) and (12.9319, -1.7743) .. (12.9287, -1.7738).. controls (12.9254, -1.7732) and (12.9221, -1.7729) .. (12.9186, -1.7729) -- (12.8823, -1.7729) -- (12.8823, -1.814) -- (12.9216, -1.814) -- cycle(12.9239, -1.8889).. controls (12.9277, -1.8889) and (12.9313, -1.8885) .. (12.9348, -1.8878).. controls (12.9382, -1.887) and (12.9413, -1.8858) .. (12.9439, -1.8841).. controls (12.9466, -1.8823) and (12.9487, -1.88) .. (12.9502, -1.877).. controls (12.9518, -1.8741) and (12.9526, -1.8703) .. (12.9526, -1.8657).. controls (12.9526, -1.8566) and (12.95, -1.8502) .. (12.9449, -1.8463).. controls (12.9398, -1.8424) and (12.933, -1.8405) .. (12.9246, -1.8405) -- (12.8823, -1.8405) -- (12.8823, -1.8889) -- (12.9239, -1.8889) -- cycle;



    \path[fill=white] (12.9983, -1.7427) -- (13.0417, -1.7427) -- (13.083, -1.8124) -- (13.1241, -1.7427) -- (13.1673, -1.7427) -- (13.1018, -1.8514) -- (13.1018, -1.9191) -- (13.063, -1.9191) -- (13.063, -1.8504) -- (12.9983, -1.7427) -- cycle;



    \path[fill=white] (14.2588, -1.8755).. controls (14.261, -1.8796) and (14.2638, -1.8829) .. (14.2673, -1.8855).. controls (14.2708, -1.888) and (14.275, -1.8899) .. (14.2798, -1.8912).. controls (14.2845, -1.8924) and (14.2894, -1.893) .. (14.2945, -1.893).. controls (14.298, -1.893) and (14.3017, -1.8927) .. (14.3056, -1.8921).. controls (14.3096, -1.8916) and (14.3132, -1.8905) .. (14.3167, -1.8888).. controls (14.3201, -1.8872) and (14.323, -1.8849) .. (14.3253, -1.882).. controls (14.3276, -1.8791) and (14.3288, -1.8755) .. (14.3288, -1.871).. controls (14.3288, -1.8662) and (14.3273, -1.8624) .. (14.3242, -1.8594).. controls (14.3212, -1.8564) and (14.3172, -1.854) .. (14.3123, -1.852).. controls (14.3073, -1.85) and (14.3017, -1.8483) .. (14.2955, -1.8468).. controls (14.2892, -1.8453) and (14.2829, -1.8437) .. (14.2765, -1.8419).. controls (14.2699, -1.8402) and (14.2635, -1.8382) .. (14.2573, -1.8358).. controls (14.251, -1.8335) and (14.2454, -1.8304) .. (14.2405, -1.8266).. controls (14.2356, -1.8228) and (14.2316, -1.8181) .. (14.2286, -1.8124).. controls (14.2255, -1.8067) and (14.224, -1.7998) .. (14.224, -1.7918).. controls (14.224, -1.7827) and (14.2259, -1.7748) .. (14.2298, -1.7682).. controls (14.2336, -1.7615) and (14.2387, -1.7559) .. (14.2449, -1.7515).. controls (14.2512, -1.747) and (14.2583, -1.7437) .. (14.2662, -1.7416).. controls (14.2741, -1.7395) and (14.282, -1.7384) .. (14.2898, -1.7384).. controls (14.2991, -1.7384) and (14.3079, -1.7394) .. (14.3164, -1.7415).. controls (14.3248, -1.7435) and (14.3324, -1.7469) .. (14.3389, -1.7515).. controls (14.3455, -1.7561) and (14.3507, -1.762) .. (14.3546, -1.7692).. controls (14.3585, -1.7764) and (14.3604, -1.785) .. (14.3604, -1.7952) -- (14.3228, -1.7952).. controls (14.3224, -1.79) and (14.3214, -1.7856) .. (14.3195, -1.7822).. controls (14.3176, -1.7787) and (14.3151, -1.776) .. (14.312, -1.774).. controls (14.3089, -1.772) and (14.3053, -1.7706) .. (14.3013, -1.7698).. controls (14.2973, -1.769) and (14.2929, -1.7686) .. (14.2881, -1.7686).. controls (14.285, -1.7686) and (14.2819, -1.7689) .. (14.2788, -1.7696).. controls (14.2757, -1.7702) and (14.2729, -1.7714) .. (14.2703, -1.773).. controls (14.2678, -1.7747) and (14.2657, -1.7767) .. (14.2641, -1.7792).. controls (14.2624, -1.7817) and (14.2616, -1.7848) .. (14.2616, -1.7886).. controls (14.2616, -1.792) and (14.2622, -1.7949) .. (14.2636, -1.797).. controls (14.2649, -1.7991) and (14.2675, -1.8011) .. (14.2713, -1.8029).. controls (14.2752, -1.8047) and (14.2805, -1.8065) .. (14.2874, -1.8083).. controls (14.2942, -1.8101) and (14.3031, -1.8125) .. (14.3141, -1.8153).. controls (14.3174, -1.8159) and (14.322, -1.8171) .. (14.3278, -1.8189).. controls (14.3336, -1.8206) and (14.3394, -1.8233) .. (14.3452, -1.8271).. controls (14.3509, -1.8309) and (14.3559, -1.836) .. (14.3601, -1.8423).. controls (14.3643, -1.8486) and (14.3664, -1.8568) .. (14.3664, -1.8667).. controls (14.3664, -1.8747) and (14.3648, -1.8822) .. (14.3617, -1.8891).. controls (14.3586, -1.8961) and (14.3539, -1.902) .. (14.3477, -1.907).. controls (14.3416, -1.9121) and (14.3339, -1.916) .. (14.3248, -1.9188).. controls (14.3157, -1.9216) and (14.3051, -1.923) .. (14.293, -1.923).. controls (14.2833, -1.923) and (14.2739, -1.9218) .. (14.2648, -1.9194).. controls (14.2556, -1.917) and (14.2476, -1.9132) .. (14.2406, -1.9081).. controls (14.2336, -1.903) and (14.228, -1.8965) .. (14.2239, -1.8886).. controls (14.2198, -1.8807) and (14.2178, -1.8713) .. (14.218, -1.8604) -- (14.2556, -1.8604).. controls (14.2556, -1.8663) and (14.2567, -1.8714) .. (14.2588, -1.8755) -- cycle;



    \path[fill=white] (14.4773, -1.7427) -- (14.5432, -1.9191) -- (14.503, -1.9191) -- (14.4896, -1.8798) -- (14.4237, -1.8798) -- (14.4098, -1.9191) -- (14.3708, -1.9191) -- (14.4375, -1.7427) -- (14.4773, -1.7427) -- cycle(14.4795, -1.8508) -- (14.4573, -1.7862) -- (14.4568, -1.7862) -- (14.4338, -1.8508) -- (14.4795, -1.8508) -- cycle;



    \path[fill=white] (13.5686, -1.7427) -- (13.6423, -1.861) -- (13.6427, -1.861) -- (13.6427, -1.7427) -- (13.6791, -1.7427) -- (13.6791, -1.9191) -- (13.6403, -1.9191) -- (13.5669, -1.801) -- (13.5664, -1.801) -- (13.5664, -1.9191) -- (13.5301, -1.9191) -- (13.5301, -1.7427) -- (13.5686, -1.7427) -- cycle;



    \path[fill=white] (13.8263, -1.7897).. controls (13.824, -1.786) and (13.8211, -1.7827) .. (13.8176, -1.7799).. controls (13.8142, -1.7772) and (13.8103, -1.775) .. (13.8059, -1.7734).. controls (13.8015, -1.7718) and (13.797, -1.7711) .. (13.7922, -1.7711).. controls (13.7835, -1.7711) and (13.7761, -1.7727) .. (13.77, -1.7761).. controls (13.7639, -1.7795) and (13.7589, -1.784) .. (13.7552, -1.7897).. controls (13.7514, -1.7954) and (13.7486, -1.8018) .. (13.7469, -1.8091).. controls (13.7451, -1.8163) and (13.7443, -1.8238) .. (13.7443, -1.8315).. controls (13.7443, -1.8389) and (13.7451, -1.8461) .. (13.7469, -1.8531).. controls (13.7486, -1.8601) and (13.7514, -1.8664) .. (13.7552, -1.872).. controls (13.7589, -1.8776) and (13.7639, -1.8821) .. (13.77, -1.8855).. controls (13.7761, -1.8889) and (13.7835, -1.8905) .. (13.7922, -1.8905).. controls (13.8041, -1.8905) and (13.8133, -1.8869) .. (13.82, -1.8797).. controls (13.8267, -1.8724) and (13.8307, -1.8629) .. (13.8322, -1.851) -- (13.8698, -1.851).. controls (13.8688, -1.862) and (13.8662, -1.872) .. (13.8621, -1.8809).. controls (13.858, -1.8898) and (13.8526, -1.8974) .. (13.8458, -1.9037).. controls (13.8391, -1.9099) and (13.8312, -1.9147) .. (13.8221, -1.918).. controls (13.813, -1.9213) and (13.8031, -1.923) .. (13.7922, -1.923).. controls (13.7787, -1.923) and (13.7666, -1.9206) .. (13.7558, -1.9159).. controls (13.745, -1.9113) and (13.7359, -1.9048) .. (13.7285, -1.8966).. controls (13.721, -1.8883) and (13.7153, -1.8786) .. (13.7114, -1.8675).. controls (13.7074, -1.8564) and (13.7054, -1.8444) .. (13.7054, -1.8315).. controls (13.7054, -1.8184) and (13.7074, -1.8061) .. (13.7114, -1.7949).. controls (13.7153, -1.7836) and (13.721, -1.7737) .. (13.7285, -1.7653).. controls (13.7359, -1.7569) and (13.745, -1.7503) .. (13.7558, -1.7456).. controls (13.7666, -1.7408) and (13.7787, -1.7384) .. (13.7922, -1.7384).. controls (13.8019, -1.7384) and (13.8111, -1.7398) .. (13.8197, -1.7426).. controls (13.8284, -1.7454) and (13.8361, -1.7495) .. (13.843, -1.7549).. controls (13.8498, -1.7602) and (13.8554, -1.7668) .. (13.8599, -1.7747).. controls (13.8643, -1.7827) and (13.8671, -1.7917) .. (13.8683, -1.8019) -- (13.8307, -1.8019).. controls (13.8301, -1.7975) and (13.8286, -1.7934) .. (13.8263, -1.7897) -- cycle;



    \begin{scope}[cm={ 0.625,-0.0,-0.0,0.625,(4.0296, -7.9688)}]
      \path[fill=white] (17.0544, 10.7187).. controls (17.0546, 10.4697) and (16.8529, 10.2676) .. (16.6038, 10.2674).. controls (16.3548, 10.2673) and (16.1528, 10.469) .. (16.1526, 10.7181).. controls (16.1526, 10.7183) and (16.1526, 10.7185) .. (16.1526, 10.7187).. controls (16.1524, 10.9677) and (16.3542, 11.1697) .. (16.6032, 11.1699).. controls (16.8522, 11.1701) and (17.0542, 10.9684) .. (17.0544, 10.7194).. controls (17.0544, 10.7191) and (17.0544, 10.7189) .. (17.0544, 10.7187) -- cycle;



      \begin{scope}[shift={(-0.6337, 2.3741)}]
        \path[fill] (17.234, 8.8386).. controls (17.0972, 8.8386) and (16.9813, 8.7908) .. (16.8866, 8.6954).. controls (16.7893, 8.5965) and (16.7406, 8.4796) .. (16.7406, 8.3446).. controls (16.7406, 8.2096) and (16.7893, 8.0935) .. (16.8866, 7.9963).. controls (16.9838, 7.8992) and (17.0997, 7.8506) .. (17.234, 7.8506).. controls (17.37, 7.8506) and (17.4879, 7.8996) .. (17.5876, 7.9975).. controls (17.6816, 8.0906) and (17.7286, 8.2063) .. (17.7286, 8.3446).. controls (17.7286, 8.4829) and (17.6808, 8.5998) .. (17.5852, 8.6954).. controls (17.4895, 8.7908) and (17.3725, 8.8386) .. (17.234, 8.8386) -- cycle(17.2353, 8.7497).. controls (17.3474, 8.7497) and (17.4425, 8.7102) .. (17.5209, 8.6311).. controls (17.6, 8.5529) and (17.6396, 8.4574) .. (17.6396, 8.3446).. controls (17.6396, 8.231) and (17.6008, 8.1367) .. (17.5233, 8.0618).. controls (17.4418, 7.9811) and (17.3457, 7.9408) .. (17.2353, 7.9408).. controls (17.1248, 7.9408) and (17.0296, 7.9807) .. (16.9496, 8.0606).. controls (16.8697, 8.1404) and (16.8297, 8.2351) .. (16.8297, 8.3446).. controls (16.8297, 8.4541) and (16.87, 8.5496) .. (16.9509, 8.6311).. controls (17.0283, 8.7102) and (17.1231, 8.7497) .. (17.2353, 8.7497) -- cycle;



        \path[fill] (17.0151, 8.4143).. controls (17.0347, 8.5386) and (17.1222, 8.605) .. (17.2319, 8.605).. controls (17.3896, 8.605) and (17.4857, 8.4906) .. (17.4857, 8.338).. controls (17.4857, 8.1891) and (17.3834, 8.0735) .. (17.2294, 8.0735).. controls (17.1235, 8.0735) and (17.0286, 8.1386) .. (17.0113, 8.2666) -- (17.1358, 8.2666).. controls (17.1395, 8.2002) and (17.1826, 8.1768) .. (17.2442, 8.1768).. controls (17.3144, 8.1768) and (17.36, 8.242) .. (17.36, 8.3417).. controls (17.36, 8.4463) and (17.3206, 8.5017) .. (17.2467, 8.5017).. controls (17.1925, 8.5017) and (17.1457, 8.482) .. (17.1358, 8.4143) -- (17.172, 8.4145) -- (17.074, 8.3166) -- (16.9761, 8.4145) -- (17.0151, 8.4143) -- cycle;



      \end{scope}
    \end{scope}
    \begin{scope}[cm={ 1.1468,-0.0,-0.0,1.1468,(-2.2734, -4.5593)}]
      \path[fill=white] (14.1879, 2.8685).. controls (14.188, 2.7301) and (14.076, 2.618) .. (13.9377, 2.6178).. controls (13.7994, 2.6178) and (13.6871, 2.7298) .. (13.6871, 2.8681).. controls (13.6871, 2.8682) and (13.6871, 2.8684) .. (13.6871, 2.8685).. controls (13.6869, 3.0068) and (13.799, 3.119) .. (13.9373, 3.1191).. controls (14.0756, 3.1192) and (14.1878, 3.0071) .. (14.1879, 2.8688).. controls (14.1879, 2.8687) and (14.1879, 2.8686) .. (14.1879, 2.8685) -- cycle;



      \path[fill] (13.9371, 3.1377).. controls (14.0126, 3.1377) and (14.0764, 3.1116) .. (14.1285, 3.0596).. controls (14.1807, 3.0076) and (14.2067, 2.9438) .. (14.2067, 2.8685).. controls (14.2067, 2.7931) and (14.1811, 2.7301) .. (14.1299, 2.6793).. controls (14.0755, 2.6259) and (14.0113, 2.5993) .. (13.9371, 2.5993).. controls (13.8639, 2.5993) and (13.8008, 2.6257) .. (13.7478, 2.6787).. controls (13.6948, 2.7316) and (13.6683, 2.7949) .. (13.6683, 2.8685).. controls (13.6683, 2.9421) and (13.6948, 3.0058) .. (13.7478, 3.0596).. controls (13.7995, 3.1116) and (13.8626, 3.1377) .. (13.9371, 3.1377) -- cycle(13.7289, 2.9408).. controls (13.7208, 2.9181) and (13.7168, 2.894) .. (13.7168, 2.8685).. controls (13.7168, 2.8088) and (13.7386, 2.7572) .. (13.7822, 2.7137).. controls (13.8257, 2.6702) and (13.8776, 2.6484) .. (13.9378, 2.6484).. controls (13.998, 2.6484) and (14.0504, 2.6704) .. (14.0948, 2.7144).. controls (14.1097, 2.7287) and (14.122, 2.7444) .. (14.1316, 2.7614) -- (14.0301, 2.8066).. controls (14.0233, 2.7725) and (13.9928, 2.7494) .. (13.956, 2.7467) -- (13.956, 2.7052) -- (13.9251, 2.7052) -- (13.9251, 2.7467).. controls (13.8949, 2.7471) and (13.8657, 2.7594) .. (13.8434, 2.779) -- (13.8805, 2.8163).. controls (13.8983, 2.7995) and (13.9162, 2.792) .. (13.9406, 2.792).. controls (13.9564, 2.792) and (13.9739, 2.7982) .. (13.9739, 2.8188).. controls (13.9739, 2.826) and (13.9711, 2.8311) .. (13.9666, 2.8349) -- (13.941, 2.8464) -- (13.909, 2.8606).. controls (13.8932, 2.8676) and (13.8798, 2.8736) .. (13.8664, 2.8796) -- (13.7289, 2.9408) -- cycle(13.9378, 3.0892).. controls (13.8767, 3.0892) and (13.8251, 3.0677) .. (13.7828, 3.0246).. controls (13.7713, 3.013) and (13.7613, 3.0009) .. (13.7528, 2.9882) -- (13.8558, 2.9424).. controls (13.8651, 2.971) and (13.8922, 2.9883) .. (13.9251, 2.9902) -- (13.9251, 3.0317) -- (13.956, 3.0317) -- (13.956, 2.9902).. controls (13.9773, 2.9892) and (14.0007, 2.9834) .. (14.0237, 2.9655) -- (13.9883, 2.9292).. controls (13.9752, 2.9384) and (13.9588, 2.9449) .. (13.9423, 2.9449).. controls (13.9289, 2.9449) and (13.91, 2.9408) .. (13.91, 2.9241).. controls (13.91, 2.9215) and (13.9109, 2.9192) .. (13.9124, 2.9172) -- (13.9469, 2.9019) -- (13.9702, 2.8915).. controls (13.9851, 2.8848) and (13.9993, 2.8785) .. (14.0134, 2.8723) -- (14.1513, 2.8108).. controls (14.1559, 2.8289) and (14.1582, 2.8481) .. (14.1582, 2.8685).. controls (14.1582, 2.93) and (14.1366, 2.982) .. (14.0935, 3.0246).. controls (14.0508, 3.0677) and (13.9989, 3.0892) .. (13.9378, 3.0892) -- cycle;



    \end{scope}
  \end{scope}

\end{tikzpicture}
 \\ {\tt cc-by-nc-sa}

        \vspace{3pt}
\definecolor{caab2ab}{RGB}{170,178,171}


\def \globalscale {0.550000}
\begin{tikzpicture}[y=1cm, x=1cm, yscale=\globalscale,xscale=\globalscale, every node/.append style={scale=\globalscale}, inner sep=0pt, outer sep=0pt]
  \begin{scope}[cm={ 0.9938,-0.0,-0.0,0.9937,(-11.5655, 0.007)}]
    \path[fill=caab2ab] (11.7287, 1.0994) -- (14.7466, 1.094).. controls (14.7888, 1.094) and (14.8265, 1.1003) .. (14.8265, 1.0099) -- (14.8228, 0.0159) -- (11.6525, 0.0159) -- (11.6525, 1.0136).. controls (11.6525, 1.0581) and (11.6568, 1.0994) .. (11.7287, 1.0994) -- cycle;



    \path[fill] (14.773, 1.1113) -- (11.6978, 1.1113).. controls (11.6648, 1.1113) and (11.6379, 1.0844) .. (11.6379, 1.0514) -- (11.6379, 0.0064).. controls (11.6379, -0.001) and (11.644, -0.0071) .. (11.6514, -0.0071) -- (14.8193, -0.0071).. controls (14.8268, -0.0071) and (14.8328, -0.001) .. (14.8328, 0.0064) -- (14.8328, 1.0514).. controls (14.8328, 1.0844) and (14.806, 1.1113) .. (14.773, 1.1113) -- cycle(11.6978, 1.0842) -- (14.773, 1.0842).. controls (14.7911, 1.0842) and (14.8058, 1.0695) .. (14.8058, 1.0514).. controls (14.8058, 1.0514) and (14.8058, 0.631) .. (14.8058, 0.3268) -- (12.6078, 0.3268).. controls (12.5273, 0.1811) and (12.372, 0.0822) .. (12.1939, 0.0822).. controls (12.0157, 0.0822) and (11.8605, 0.181) .. (11.78, 0.3268) -- (11.665, 0.3268).. controls (11.665, 0.631) and (11.665, 1.0514) .. (11.665, 1.0514).. controls (11.665, 1.0695) and (11.6797, 1.0842) .. (11.6978, 1.0842) -- cycle;



    \begin{scope}[cm={ 0.8729,-0.0,-0.0,0.8729,(1.3262, -3.648)}]
      \path[fill=white] (12.8658, 4.8559).. controls (12.866, 4.6262) and (12.6799, 4.44) .. (12.4502, 4.4398).. controls (12.2205, 4.4397) and (12.0342, 4.6257) .. (12.0341, 4.8554).. controls (12.0341, 4.8556) and (12.0341, 4.8557) .. (12.0341, 4.8559).. controls (12.0339, 5.0856) and (12.22, 5.2719) .. (12.4497, 5.272).. controls (12.6794, 5.2722) and (12.8657, 5.0861) .. (12.8658, 4.8564).. controls (12.8658, 4.8563) and (12.8658, 4.8561) .. (12.8658, 4.8559) -- cycle;



      \begin{scope}[shift={(-7.6627, -2.6211)}]
        \path[fill] (20.4508, 7.8157).. controls (20.543, 7.7235) and (20.5891, 7.6106) .. (20.5891, 7.477).. controls (20.5891, 7.3434) and (20.5438, 7.2317) .. (20.4531, 7.1418).. controls (20.3569, 7.0472) and (20.2433, 6.9999) .. (20.1121, 6.9999).. controls (19.9825, 6.9999) and (19.8708, 7.0468) .. (19.777, 7.1407).. controls (19.6832, 7.2345) and (19.6363, 7.3466) .. (19.6363, 7.477).. controls (19.6363, 7.6074) and (19.6832, 7.7203) .. (19.777, 7.8157).. controls (19.8684, 7.908) and (19.9801, 7.9541) .. (20.1121, 7.9541).. controls (20.2457, 7.9541) and (20.3585, 7.908) .. (20.4508, 7.8157) -- cycle(19.8391, 7.7537).. controls (19.7611, 7.675) and (19.7222, 7.5827) .. (19.7222, 7.4769).. controls (19.7222, 7.3711) and (19.7607, 7.2796) .. (19.8379, 7.2025).. controls (19.9151, 7.1253) and (20.0069, 7.0868) .. (20.1135, 7.0868).. controls (20.2201, 7.0868) and (20.3128, 7.1257) .. (20.3915, 7.2037).. controls (20.4663, 7.2761) and (20.5037, 7.3671) .. (20.5037, 7.4769).. controls (20.5037, 7.5859) and (20.4657, 7.6783) .. (20.3897, 7.7543).. controls (20.3137, 7.8303) and (20.2217, 7.8682) .. (20.1135, 7.8682).. controls (20.0054, 7.8682) and (19.9138, 7.8301) .. (19.8391, 7.7537) -- cycle(20.0443, 7.5234).. controls (20.0324, 7.5494) and (20.0145, 7.5624) .. (19.9907, 7.5624).. controls (19.9487, 7.5624) and (19.9277, 7.5341) .. (19.9277, 7.4775).. controls (19.9277, 7.4209) and (19.9487, 7.3926) .. (19.9907, 7.3926).. controls (20.0185, 7.3926) and (20.0383, 7.4064) .. (20.0502, 7.434) -- (20.1085, 7.403).. controls (20.0807, 7.3536) and (20.039, 7.3289) .. (19.9835, 7.3289).. controls (19.9406, 7.3289) and (19.9063, 7.3421) .. (19.8805, 7.3683).. controls (19.8547, 7.3946) and (19.8418, 7.4308) .. (19.8418, 7.477).. controls (19.8418, 7.5224) and (19.8551, 7.5584) .. (19.8817, 7.5851).. controls (19.9083, 7.6118) and (19.9414, 7.6251) .. (19.9811, 7.6251).. controls (20.0399, 7.6251) and (20.0819, 7.602) .. (20.1074, 7.5557) -- (20.0443, 7.5234) -- cycle(20.3185, 7.5234).. controls (20.3065, 7.5494) and (20.289, 7.5624) .. (20.266, 7.5624).. controls (20.2231, 7.5624) and (20.2017, 7.5341) .. (20.2017, 7.4775).. controls (20.2017, 7.4209) and (20.2231, 7.3926) .. (20.266, 7.3926).. controls (20.2938, 7.3926) and (20.3133, 7.4064) .. (20.3244, 7.434) -- (20.384, 7.403).. controls (20.3563, 7.3536) and (20.3146, 7.3289) .. (20.2592, 7.3289).. controls (20.2164, 7.3289) and (20.1821, 7.3421) .. (20.1563, 7.3683).. controls (20.1306, 7.3946) and (20.1177, 7.4308) .. (20.1177, 7.477).. controls (20.1177, 7.5224) and (20.1308, 7.5584) .. (20.157, 7.5851).. controls (20.1831, 7.6118) and (20.2164, 7.6251) .. (20.2568, 7.6251).. controls (20.3154, 7.6251) and (20.3574, 7.602) .. (20.3828, 7.5557) -- (20.3185, 7.5234) -- cycle;



      \end{scope}
    \end{scope}
    \path[fill=white] (12.9184, 0.2393).. controls (12.9268, 0.2393) and (12.9345, 0.2385) .. (12.9414, 0.2371).. controls (12.9483, 0.2356) and (12.9542, 0.2332) .. (12.9592, 0.2298).. controls (12.9641, 0.2264) and (12.968, 0.2219) .. (12.9707, 0.2163).. controls (12.9734, 0.2107) and (12.9748, 0.2038) .. (12.9748, 0.1956).. controls (12.9748, 0.1867) and (12.9728, 0.1793) .. (12.9687, 0.1734).. controls (12.9647, 0.1675) and (12.9587, 0.1626) .. (12.9508, 0.1588).. controls (12.9617, 0.1557) and (12.9698, 0.1502) .. (12.9752, 0.1424).. controls (12.9806, 0.1346) and (12.9833, 0.1251) .. (12.9833, 0.1141).. controls (12.9833, 0.1052) and (12.9815, 0.0975) .. (12.9781, 0.091).. controls (12.9746, 0.0845) and (12.9699, 0.0792) .. (12.9641, 0.0751).. controls (12.9582, 0.071) and (12.9515, 0.0679) .. (12.944, 0.0659).. controls (12.9365, 0.0639) and (12.9288, 0.063) .. (12.9209, 0.063) -- (12.8353, 0.063) -- (12.8353, 0.2393) -- (12.9184, 0.2393) -- (12.9184, 0.2393) -- cycle(12.9134, 0.168).. controls (12.9203, 0.168) and (12.926, 0.1696) .. (12.9305, 0.1729).. controls (12.9349, 0.1762) and (12.9372, 0.1815) .. (12.9372, 0.1889).. controls (12.9372, 0.193) and (12.9364, 0.1964) .. (12.9349, 0.199).. controls (12.9334, 0.2016) and (12.9315, 0.2037) .. (12.929, 0.2052).. controls (12.9265, 0.2067) and (12.9237, 0.2077) .. (12.9205, 0.2083).. controls (12.9173, 0.2088) and (12.9139, 0.2091) .. (12.9104, 0.2091) -- (12.8741, 0.2091) -- (12.8741, 0.168) -- (12.9134, 0.168) -- cycle(12.9157, 0.0931).. controls (12.9195, 0.0931) and (12.9231, 0.0935) .. (12.9266, 0.0943).. controls (12.93, 0.095) and (12.9331, 0.0962) .. (12.9357, 0.098).. controls (12.9384, 0.0997) and (12.9405, 0.102) .. (12.942, 0.105).. controls (12.9436, 0.108) and (12.9444, 0.1118) .. (12.9444, 0.1163).. controls (12.9444, 0.1254) and (12.9418, 0.1319) .. (12.9367, 0.1357).. controls (12.9316, 0.1396) and (12.9249, 0.1415) .. (12.9164, 0.1415) -- (12.8741, 0.1415) -- (12.8741, 0.0931) -- (12.9157, 0.0931) -- cycle;



    \path[fill=white] (12.9901, 0.2393) -- (13.0336, 0.2393) -- (13.0749, 0.1697) -- (13.1159, 0.2393) -- (13.1591, 0.2393) -- (13.0937, 0.1306) -- (13.0937, 0.063) -- (13.0548, 0.063) -- (13.0548, 0.1316) -- (12.9901, 0.2393) -- cycle;



    \path[fill=white] (13.5686, 0.2393) -- (13.6423, 0.121) -- (13.6427, 0.121) -- (13.6427, 0.2393) -- (13.6791, 0.2393) -- (13.6791, 0.063) -- (13.6403, 0.063) -- (13.5669, 0.181) -- (13.5664, 0.181) -- (13.5664, 0.063) -- (13.5301, 0.063) -- (13.5301, 0.2393) -- (13.5686, 0.2393) -- cycle;



    \path[fill=white] (13.8263, 0.1923).. controls (13.824, 0.196) and (13.8211, 0.1993) .. (13.8176, 0.2021).. controls (13.8142, 0.2049) and (13.8103, 0.2071) .. (13.8059, 0.2086).. controls (13.8015, 0.2102) and (13.797, 0.211) .. (13.7922, 0.211).. controls (13.7835, 0.211) and (13.7761, 0.2093) .. (13.77, 0.2059).. controls (13.7639, 0.2025) and (13.7589, 0.198) .. (13.7552, 0.1923).. controls (13.7514, 0.1867) and (13.7486, 0.1802) .. (13.7469, 0.173).. controls (13.7451, 0.1657) and (13.7443, 0.1582) .. (13.7443, 0.1505).. controls (13.7443, 0.1431) and (13.7451, 0.1359) .. (13.7469, 0.1289).. controls (13.7486, 0.1219) and (13.7514, 0.1156) .. (13.7552, 0.11).. controls (13.7589, 0.1044) and (13.7639, 0.0999) .. (13.77, 0.0966).. controls (13.7761, 0.0932) and (13.7835, 0.0915) .. (13.7922, 0.0915).. controls (13.8041, 0.0915) and (13.8133, 0.0951) .. (13.82, 0.1024).. controls (13.8267, 0.1096) and (13.8307, 0.1192) .. (13.8322, 0.131) -- (13.8698, 0.131).. controls (13.8688, 0.12) and (13.8662, 0.11) .. (13.8621, 0.1011).. controls (13.858, 0.0922) and (13.8526, 0.0846) .. (13.8458, 0.0784).. controls (13.8391, 0.0721) and (13.8312, 0.0673) .. (13.8221, 0.064).. controls (13.813, 0.0607) and (13.8031, 0.059) .. (13.7922, 0.059).. controls (13.7787, 0.059) and (13.7666, 0.0614) .. (13.7558, 0.0661).. controls (13.745, 0.0708) and (13.7359, 0.0773) .. (13.7285, 0.0855).. controls (13.721, 0.0937) and (13.7153, 0.1034) .. (13.7114, 0.1145).. controls (13.7074, 0.1257) and (13.7054, 0.1376) .. (13.7054, 0.1505).. controls (13.7054, 0.1637) and (13.7074, 0.1759) .. (13.7114, 0.1872).. controls (13.7153, 0.1985) and (13.721, 0.2083) .. (13.7285, 0.2167).. controls (13.7359, 0.2251) and (13.745, 0.2317) .. (13.7558, 0.2365).. controls (13.7666, 0.2413) and (13.7787, 0.2436) .. (13.7922, 0.2436).. controls (13.8019, 0.2436) and (13.8111, 0.2422) .. (13.8198, 0.2394).. controls (13.8284, 0.2366) and (13.8361, 0.2325) .. (13.843, 0.2272).. controls (13.8498, 0.2218) and (13.8554, 0.2152) .. (13.8599, 0.2073).. controls (13.8643, 0.1994) and (13.8671, 0.1903) .. (13.8683, 0.1801) -- (13.8307, 0.1801).. controls (13.8301, 0.1845) and (13.8286, 0.1886) .. (13.8263, 0.1923) -- cycle;



    \path[fill=white] (14.2566, 0.2393) -- (14.3302, 0.121) -- (14.3306, 0.121) -- (14.3306, 0.2393) -- (14.367, 0.2393) -- (14.367, 0.063) -- (14.3282, 0.063) -- (14.2549, 0.181) -- (14.2544, 0.181) -- (14.2544, 0.063) -- (14.218, 0.063) -- (14.218, 0.2393) -- (14.2566, 0.2393) -- cycle;



    \path[fill=white] (14.4771, 0.2393).. controls (14.4885, 0.2393) and (14.4991, 0.2375) .. (14.5089, 0.2339).. controls (14.5187, 0.2302) and (14.5272, 0.2248) .. (14.5344, 0.2176).. controls (14.5416, 0.2103) and (14.5472, 0.2013) .. (14.5512, 0.1904).. controls (14.5552, 0.1795) and (14.5573, 0.1668) .. (14.5573, 0.1521).. controls (14.5573, 0.1393) and (14.5556, 0.1274) .. (14.5523, 0.1166).. controls (14.549, 0.1057) and (14.544, 0.0963) .. (14.5374, 0.0884).. controls (14.5307, 0.0805) and (14.5224, 0.0743) .. (14.5124, 0.0698).. controls (14.5024, 0.0652) and (14.4907, 0.063) .. (14.4771, 0.063) -- (14.401, 0.063) -- (14.401, 0.2393) -- (14.4771, 0.2393) -- (14.4771, 0.2393) -- cycle(14.4744, 0.0956).. controls (14.48, 0.0956) and (14.4855, 0.0965) .. (14.4907, 0.0983).. controls (14.496, 0.1001) and (14.5007, 0.1031) .. (14.5048, 0.1074).. controls (14.5089, 0.1115) and (14.5122, 0.117) .. (14.5147, 0.1237).. controls (14.5172, 0.1305) and (14.5184, 0.1387) .. (14.5184, 0.1484).. controls (14.5184, 0.1573) and (14.5176, 0.1653) .. (14.5158, 0.1725).. controls (14.5141, 0.1796) and (14.5112, 0.1858) .. (14.5073, 0.1908).. controls (14.5033, 0.1959) and (14.4981, 0.1999) .. (14.4916, 0.2026).. controls (14.4851, 0.2053) and (14.4771, 0.2066) .. (14.4675, 0.2066) -- (14.4399, 0.2066) -- (14.4399, 0.0956) -- (14.4744, 0.0956) -- (14.4744, 0.0956) -- cycle;



    \begin{scope}[shift={(7.571, -5.5047)}]
      \begin{scope}[cm={ 1.1468,-0.0,-0.0,1.1468,(-1.7764, 0.9454)}]
        \path[fill=white] (7.1336, 4.5967).. controls (7.1338, 4.469) and (7.0303, 4.3654) .. (6.9026, 4.3653).. controls (6.7749, 4.3653) and (6.6713, 4.4687) .. (6.6712, 4.5964).. controls (6.6712, 4.5965) and (6.6712, 4.5966) .. (6.6712, 4.5967).. controls (6.6711, 4.7245) and (6.7746, 4.828) .. (6.9023, 4.8281).. controls (7.03, 4.8282) and (7.1336, 4.7248) .. (7.1336, 4.5971).. controls (7.1336, 4.597) and (7.1336, 4.5968) .. (7.1336, 4.5967) -- cycle;



        \path[fill] (6.9021, 4.8659).. controls (6.9776, 4.8659) and (7.0413, 4.8399) .. (7.0935, 4.7879).. controls (7.1456, 4.7358) and (7.1716, 4.6721) .. (7.1716, 4.5967).. controls (7.1716, 4.5214) and (7.146, 4.4583) .. (7.0948, 4.4076).. controls (7.0404, 4.3542) and (6.9762, 4.3275) .. (6.9021, 4.3275).. controls (6.8289, 4.3275) and (6.7657, 4.354) .. (6.7127, 4.407).. controls (6.6597, 4.4599) and (6.6332, 4.5231) .. (6.6332, 4.5967).. controls (6.6332, 4.6703) and (6.6597, 4.7341) .. (6.7127, 4.7879).. controls (6.7644, 4.8399) and (6.8275, 4.8659) .. (6.9021, 4.8659) -- cycle(6.6938, 4.6691).. controls (6.6857, 4.6463) and (6.6817, 4.6222) .. (6.6817, 4.5967).. controls (6.6817, 4.5371) and (6.7035, 4.4855) .. (6.7471, 4.4419).. controls (6.7907, 4.3984) and (6.8426, 4.3767) .. (6.9028, 4.3767).. controls (6.9629, 4.3767) and (7.0153, 4.3986) .. (7.0597, 4.4426).. controls (7.0746, 4.457) and (7.0869, 4.4727) .. (7.0965, 4.4897) -- (6.995, 4.5349).. controls (6.9882, 4.5008) and (6.9577, 4.4777) .. (6.921, 4.475) -- (6.921, 4.4335) -- (6.8901, 4.4335) -- (6.8901, 4.475).. controls (6.8598, 4.4753) and (6.8307, 4.4877) .. (6.8083, 4.5072) -- (6.8454, 4.5446).. controls (6.8633, 4.5278) and (6.8811, 4.5203) .. (6.9055, 4.5203).. controls (6.9213, 4.5203) and (6.9388, 4.5264) .. (6.9388, 4.547).. controls (6.9388, 4.5543) and (6.936, 4.5594) .. (6.9316, 4.5632) -- (6.9059, 4.5746) -- (6.8739, 4.5888).. controls (6.8581, 4.5959) and (6.8447, 4.6018) .. (6.8313, 4.6078) -- (6.6938, 4.6691) -- cycle(6.9028, 4.8175).. controls (6.8416, 4.8175) and (6.79, 4.796) .. (6.7478, 4.7529).. controls (6.7363, 4.7413) and (6.7263, 4.7292) .. (6.7178, 4.7165) -- (6.8207, 4.6707).. controls (6.83, 4.6992) and (6.8571, 4.7166) .. (6.8901, 4.7185) -- (6.8901, 4.76) -- (6.921, 4.76) -- (6.921, 4.7185).. controls (6.9423, 4.7175) and (6.9656, 4.7116) .. (6.9886, 4.6938) -- (6.9532, 4.6574).. controls (6.9402, 4.6667) and (6.9237, 4.6732) .. (6.9072, 4.6732).. controls (6.8938, 4.6732) and (6.8749, 4.6691) .. (6.8749, 4.6523).. controls (6.8749, 4.6497) and (6.8758, 4.6475) .. (6.8774, 4.6455) -- (6.9118, 4.6302) -- (6.9351, 4.6198).. controls (6.95, 4.6131) and (6.9642, 4.6068) .. (6.9783, 4.6005) -- (7.1163, 4.5391).. controls (7.1208, 4.5572) and (7.1231, 4.5764) .. (7.1231, 4.5967).. controls (7.1231, 4.6582) and (7.1015, 4.7103) .. (7.0584, 4.7529).. controls (7.0157, 4.796) and (6.9639, 4.8175) .. (6.9028, 4.8175) -- cycle;



      \end{scope}
    \end{scope}
    \begin{scope}[cm={ 0.625,-0.0,-0.0,0.625,(8.2775, -7.9688)}]
      \path[fill=white] (10.2615, 13.8899).. controls (10.2617, 13.6409) and (10.0599, 13.4389) .. (9.8109, 13.4387).. controls (9.5619, 13.4385) and (9.3598, 13.6402) .. (9.3597, 13.8893).. controls (9.3597, 13.8895) and (9.3597, 13.8897) .. (9.3597, 13.8899).. controls (9.3595, 14.1389) and (9.5612, 14.341) .. (9.8102, 14.3412).. controls (10.0592, 14.3414) and (10.2613, 14.1396) .. (10.2615, 13.8906).. controls (10.2615, 13.8904) and (10.2615, 13.8902) .. (10.2615, 13.8899) -- cycle;



      \begin{scope}[shift={(-0.6337, 2.3262)}]
        \path[fill] (10.4372, 12.0577).. controls (10.3004, 12.0577) and (10.1846, 12.01) .. (10.0898, 11.9144).. controls (9.9925, 11.8157) and (9.9439, 11.6988) .. (9.9439, 11.5637).. controls (9.9439, 11.4287) and (9.9925, 11.3126) .. (10.0898, 11.2154).. controls (10.187, 11.1183) and (10.3029, 11.0697) .. (10.4372, 11.0697).. controls (10.5733, 11.0697) and (10.6911, 11.1187) .. (10.7909, 11.2166).. controls (10.8848, 11.3097) and (10.9319, 11.4254) .. (10.9319, 11.5637).. controls (10.9319, 11.7021) and (10.884, 11.8189) .. (10.7884, 11.9144).. controls (10.6928, 12.01) and (10.5757, 12.0577) .. (10.4372, 12.0577) -- cycle(10.4384, 11.9688).. controls (10.5506, 11.9688) and (10.6458, 11.9293) .. (10.7241, 11.8502).. controls (10.8032, 11.772) and (10.8428, 11.6765) .. (10.8428, 11.5637).. controls (10.8428, 11.4501) and (10.8041, 11.3558) .. (10.7266, 11.2809).. controls (10.6449, 11.2002) and (10.5489, 11.1599) .. (10.4384, 11.1599).. controls (10.328, 11.1599) and (10.2328, 11.1998) .. (10.1528, 11.2797).. controls (10.0729, 11.3595) and (10.0329, 11.4542) .. (10.0329, 11.5637).. controls (10.0329, 11.6732) and (10.0733, 11.7687) .. (10.1541, 11.8502).. controls (10.2316, 11.9293) and (10.3263, 11.9688) .. (10.4384, 11.9688) -- cycle;



        \path[fill] (10.6245, 11.6808) -- (10.2654, 11.6808) -- (10.2654, 11.5957) -- (10.6245, 11.5957) -- (10.6245, 11.6808) -- cycle(10.6245, 11.522) -- (10.2654, 11.522) -- (10.2654, 11.437) -- (10.6245, 11.437) -- (10.6245, 11.522) -- cycle;



      \end{scope}
    \end{scope}
    \path[fill=white] (13.0161, 0.7061) circle (0.2859cm);



    \path[fill] (13.0989, 0.7888).. controls (13.0989, 0.7998) and (13.0899, 0.8087) .. (13.0789, 0.8087) -- (12.9526, 0.8087).. controls (12.9416, 0.8087) and (12.9327, 0.7998) .. (12.9327, 0.7888) -- (12.9327, 0.6625) -- (12.9679, 0.6625) -- (12.9679, 0.513) -- (13.0636, 0.513) -- (13.0636, 0.6625) -- (13.0989, 0.6625) -- (13.0989, 0.7888) -- (13.0989, 0.7888) -- cycle;



    \path[fill] (13.0158, 0.8686) circle (0.0432cm);



    \path[fill,even odd rule] (13.0154, 1.0211).. controls (12.9299, 1.0211) and (12.8575, 0.9913) .. (12.7982, 0.9316).. controls (12.7374, 0.8698) and (12.707, 0.7967) .. (12.707, 0.7124).. controls (12.707, 0.628) and (12.7374, 0.5554) .. (12.7982, 0.4947).. controls (12.859, 0.434) and (12.9314, 0.4036) .. (13.0154, 0.4036).. controls (13.1004, 0.4036) and (13.1741, 0.4342) .. (13.2364, 0.4955).. controls (13.2951, 0.5536) and (13.3245, 0.6259) .. (13.3245, 0.7124).. controls (13.3245, 0.7988) and (13.2946, 0.8719) .. (13.2349, 0.9316).. controls (13.1751, 0.9913) and (13.102, 1.0211) .. (13.0154, 1.0211) -- cycle(13.0162, 0.9655).. controls (13.0862, 0.9655) and (13.1457, 0.9408) .. (13.1947, 0.8914).. controls (13.2441, 0.8426) and (13.2689, 0.7829) .. (13.2689, 0.7124).. controls (13.2689, 0.6414) and (13.2447, 0.5824) .. (13.1962, 0.5356).. controls (13.1452, 0.4852) and (13.0852, 0.46) .. (13.0162, 0.46).. controls (12.9471, 0.46) and (12.8876, 0.4849) .. (12.8376, 0.5348).. controls (12.7877, 0.5848) and (12.7627, 0.6439) .. (12.7627, 0.7124).. controls (12.7627, 0.7808) and (12.7879, 0.8405) .. (12.8384, 0.8914).. controls (12.8868, 0.9408) and (12.9461, 0.9655) .. (13.0162, 0.9655) -- cycle;



  \end{scope}

\end{tikzpicture}
 \\ {\tt cc-by-cs-nd}

        \vspace{3pt}
\definecolor{caab2ab}{RGB}{170,178,171}


\def \globalscale {0.550000}
\begin{tikzpicture}[y=1cm, x=1cm, yscale=\globalscale,xscale=\globalscale, every node/.append style={scale=\globalscale}, inner sep=0pt, outer sep=0pt]
  \begin{scope}[cm={ 0.9938,-0.0,-0.0,0.9937,(-4.7015, 3.946)}]
    \path[fill=caab2ab] (4.8216, -2.8645) -- (7.8396, -2.8699).. controls (7.8818, -2.8699) and (7.9194, -2.8636) .. (7.9194, -2.954) -- (7.9157, -3.948) -- (4.7455, -3.948) -- (4.7455, -2.9503).. controls (4.7455, -2.9057) and (4.7498, -2.8645) .. (4.8216, -2.8645) -- cycle;



    \begin{scope}[cm={ 0.8729,-0.0,-0.0,0.8729,(1.3262, -3.648)}]
      \path[fill=white] (4.9533, 0.315).. controls (4.9534, 0.0853) and (4.7673, -0.101) .. (4.5376, -0.1011).. controls (4.308, -0.1012) and (4.1217, 0.0848) .. (4.1215, 0.3145).. controls (4.1215, 0.3147) and (4.1215, 0.3148) .. (4.1215, 0.315).. controls (4.1214, 0.5447) and (4.3075, 0.731) .. (4.5371, 0.7311).. controls (4.7668, 0.7312) and (4.9531, 0.5452) .. (4.9533, 0.3155).. controls (4.9533, 0.3153) and (4.9533, 0.3152) .. (4.9533, 0.315) -- cycle;



      \begin{scope}[shift={(-7.6627, -2.6211)}]
        \path[fill] (12.5382, 3.2748).. controls (12.6304, 3.1826) and (12.6765, 3.0697) .. (12.6765, 2.9361).. controls (12.6765, 2.8025) and (12.6312, 2.6908) .. (12.5406, 2.6009).. controls (12.4444, 2.5063) and (12.3307, 2.459) .. (12.1995, 2.459).. controls (12.07, 2.459) and (11.9582, 2.5059) .. (11.8644, 2.5998).. controls (11.7706, 2.6936) and (11.7237, 2.8057) .. (11.7237, 2.9361).. controls (11.7237, 3.0665) and (11.7706, 3.1794) .. (11.8644, 3.2748).. controls (11.9559, 3.3671) and (12.0676, 3.4132) .. (12.1995, 3.4132).. controls (12.3331, 3.4132) and (12.446, 3.3671) .. (12.5382, 3.2748) -- cycle(11.9265, 3.2128).. controls (11.8486, 3.1341) and (11.8096, 3.0418) .. (11.8096, 2.936).. controls (11.8096, 2.8302) and (11.8482, 2.7387) .. (11.9253, 2.6616).. controls (12.0025, 2.5844) and (12.0944, 2.5458) .. (12.201, 2.5458).. controls (12.3075, 2.5458) and (12.4002, 2.5848) .. (12.4789, 2.6628).. controls (12.5537, 2.7352) and (12.5911, 2.8262) .. (12.5911, 2.936).. controls (12.5911, 3.0449) and (12.5531, 3.1374) .. (12.4771, 3.2134).. controls (12.4012, 3.2893) and (12.3091, 3.3273) .. (12.201, 3.3273).. controls (12.0928, 3.3273) and (12.0013, 3.2891) .. (11.9265, 3.2128) -- cycle(12.1317, 2.9825).. controls (12.1198, 3.0085) and (12.102, 3.0215) .. (12.0782, 3.0215).. controls (12.0361, 3.0215) and (12.0151, 2.9932) .. (12.0151, 2.9366).. controls (12.0151, 2.88) and (12.0361, 2.8517) .. (12.0782, 2.8517).. controls (12.1059, 2.8517) and (12.1258, 2.8655) .. (12.1377, 2.8931) -- (12.1959, 2.862).. controls (12.1682, 2.8127) and (12.1265, 2.788) .. (12.0709, 2.788).. controls (12.0281, 2.788) and (11.9937, 2.8011) .. (11.9679, 2.8274).. controls (11.9421, 2.8537) and (11.9292, 2.8899) .. (11.9292, 2.9361).. controls (11.9292, 2.9815) and (11.9425, 3.0175) .. (11.9691, 3.0442).. controls (11.9957, 3.0709) and (12.0288, 3.0842) .. (12.0686, 3.0842).. controls (12.1273, 3.0842) and (12.1694, 3.0611) .. (12.1948, 3.0148) -- (12.1317, 2.9825) -- cycle(12.4059, 2.9825).. controls (12.394, 3.0085) and (12.3765, 3.0215) .. (12.3534, 3.0215).. controls (12.3106, 3.0215) and (12.2891, 2.9932) .. (12.2891, 2.9366).. controls (12.2891, 2.88) and (12.3106, 2.8517) .. (12.3534, 2.8517).. controls (12.3813, 2.8517) and (12.4007, 2.8655) .. (12.4118, 2.8931) -- (12.4714, 2.862).. controls (12.4437, 2.8127) and (12.4021, 2.788) .. (12.3466, 2.788).. controls (12.3038, 2.788) and (12.2695, 2.8011) .. (12.2438, 2.8274).. controls (12.2181, 2.8537) and (12.2052, 2.8899) .. (12.2052, 2.9361).. controls (12.2052, 2.9815) and (12.2183, 3.0175) .. (12.2444, 3.0442).. controls (12.2705, 3.0709) and (12.3038, 3.0842) .. (12.3442, 3.0842).. controls (12.4029, 3.0842) and (12.4449, 3.0611) .. (12.4702, 3.0148) -- (12.4059, 2.9825) -- cycle;



      \end{scope}
    \end{scope}
    \path[fill] (7.866, -2.8528) -- (4.7907, -2.8528).. controls (4.7578, -2.8528) and (4.7309, -2.8796) .. (4.7309, -2.9126) -- (4.7309, -3.9576).. controls (4.7309, -3.9651) and (4.737, -3.9711) .. (4.7444, -3.9711) -- (7.9123, -3.9711).. controls (7.9197, -3.9711) and (7.9258, -3.9651) .. (7.9258, -3.9576) -- (7.9258, -2.9126).. controls (7.9258, -2.8796) and (7.8989, -2.8528) .. (7.866, -2.8528) -- cycle(4.7907, -2.8798) -- (7.866, -2.8798).. controls (7.8841, -2.8798) and (7.8988, -2.8945) .. (7.8988, -2.9126).. controls (7.8988, -2.9126) and (7.8988, -3.3329) .. (7.8988, -3.6371) -- (5.7008, -3.6371).. controls (5.6202, -3.7827) and (5.465, -3.8816) .. (5.2869, -3.8816).. controls (5.1087, -3.8816) and (4.9535, -3.7828) .. (4.873, -3.6371) -- (4.7579, -3.6371).. controls (4.7579, -3.3329) and (4.7579, -2.9126) .. (4.7579, -2.9126).. controls (4.7579, -2.8945) and (4.7727, -2.8798) .. (4.7907, -2.8798) -- cycle;



    \path[fill=white] (6.3283, -3.7247).. controls (6.3367, -3.7247) and (6.3443, -3.7255) .. (6.3513, -3.727).. controls (6.3582, -3.7284) and (6.3641, -3.7309) .. (6.3691, -3.7342).. controls (6.374, -3.7376) and (6.3779, -3.7421) .. (6.3806, -3.7477).. controls (6.3833, -3.7533) and (6.3847, -3.7602) .. (6.3847, -3.7684).. controls (6.3847, -3.7773) and (6.3826, -3.7847) .. (6.3786, -3.7906).. controls (6.3746, -3.7966) and (6.3686, -3.8014) .. (6.3607, -3.8052).. controls (6.3716, -3.8083) and (6.3797, -3.8138) .. (6.3851, -3.8216).. controls (6.3904, -3.8295) and (6.3931, -3.8389) .. (6.3931, -3.8499).. controls (6.3931, -3.8588) and (6.3914, -3.8665) .. (6.3879, -3.873).. controls (6.3845, -3.8795) and (6.3798, -3.8848) .. (6.374, -3.8889).. controls (6.3681, -3.8931) and (6.3614, -3.8961) .. (6.3539, -3.8981).. controls (6.3464, -3.9001) and (6.3387, -3.9011) .. (6.3308, -3.9011) -- (6.2451, -3.9011) -- (6.2451, -3.7247) -- (6.3283, -3.7247) -- (6.3283, -3.7247) -- cycle(6.3233, -3.796).. controls (6.3302, -3.796) and (6.3359, -3.7944) .. (6.3404, -3.7911).. controls (6.3448, -3.7878) and (6.347, -3.7825) .. (6.347, -3.7751).. controls (6.347, -3.771) and (6.3463, -3.7676) .. (6.3448, -3.765).. controls (6.3433, -3.7624) and (6.3414, -3.7603) .. (6.3389, -3.7588).. controls (6.3364, -3.7574) and (6.3336, -3.7563) .. (6.3303, -3.7558).. controls (6.3271, -3.7552) and (6.3238, -3.7549) .. (6.3203, -3.7549) -- (6.284, -3.7549) -- (6.284, -3.796) -- (6.3233, -3.796) -- cycle(6.3256, -3.8709).. controls (6.3294, -3.8709) and (6.333, -3.8705) .. (6.3365, -3.8698).. controls (6.3399, -3.869) and (6.343, -3.8678) .. (6.3456, -3.8661).. controls (6.3483, -3.8643) and (6.3503, -3.862) .. (6.3519, -3.859).. controls (6.3535, -3.8561) and (6.3543, -3.8523) .. (6.3543, -3.8477).. controls (6.3543, -3.8386) and (6.3517, -3.8322) .. (6.3466, -3.8283).. controls (6.3415, -3.8244) and (6.3347, -3.8225) .. (6.3263, -3.8225) -- (6.284, -3.8225) -- (6.284, -3.8709) -- (6.3256, -3.8709) -- (6.3256, -3.8709) -- cycle;



    \path[fill=white] (6.4, -3.7247) -- (6.4435, -3.7247) -- (6.4848, -3.7944) -- (6.5258, -3.7247) -- (6.569, -3.7247) -- (6.5036, -3.8334) -- (6.5036, -3.9011) -- (6.4647, -3.9011) -- (6.4647, -3.8324) -- (6.4, -3.7247) -- cycle;



    \path[fill=white] (7.0321, -3.7247) -- (7.1058, -3.843) -- (7.1062, -3.843) -- (7.1062, -3.7247) -- (7.1426, -3.7247) -- (7.1426, -3.9011) -- (7.1038, -3.9011) -- (7.0304, -3.783) -- (7.0299, -3.783) -- (7.0299, -3.9011) -- (6.9935, -3.9011) -- (6.9935, -3.7247) -- (7.0321, -3.7247) -- cycle;



    \path[fill=white] (7.2898, -3.7717).. controls (7.2875, -3.768) and (7.2846, -3.7647) .. (7.2811, -3.7619).. controls (7.2777, -3.7592) and (7.2738, -3.757) .. (7.2694, -3.7554).. controls (7.265, -3.7538) and (7.2604, -3.7531) .. (7.2557, -3.7531).. controls (7.247, -3.7531) and (7.2395, -3.7547) .. (7.2334, -3.7581).. controls (7.2273, -3.7615) and (7.2224, -3.766) .. (7.2186, -3.7717).. controls (7.2148, -3.7774) and (7.2121, -3.7838) .. (7.2103, -3.7911).. controls (7.2086, -3.7983) and (7.2078, -3.8058) .. (7.2078, -3.8135).. controls (7.2078, -3.8209) and (7.2086, -3.8281) .. (7.2103, -3.8351).. controls (7.2121, -3.8421) and (7.2148, -3.8484) .. (7.2186, -3.854).. controls (7.2224, -3.8596) and (7.2273, -3.8641) .. (7.2334, -3.8675).. controls (7.2395, -3.8708) and (7.247, -3.8725) .. (7.2557, -3.8725).. controls (7.2675, -3.8725) and (7.2768, -3.8689) .. (7.2835, -3.8617).. controls (7.2901, -3.8544) and (7.2942, -3.8449) .. (7.2957, -3.833) -- (7.3332, -3.833).. controls (7.3323, -3.844) and (7.3297, -3.854) .. (7.3256, -3.8629).. controls (7.3215, -3.8718) and (7.316, -3.8794) .. (7.3093, -3.8857).. controls (7.3025, -3.8919) and (7.2946, -3.8967) .. (7.2856, -3.9).. controls (7.2765, -3.9033) and (7.2665, -3.905) .. (7.2557, -3.905).. controls (7.2422, -3.905) and (7.23, -3.9026) .. (7.2192, -3.8979).. controls (7.2085, -3.8933) and (7.1993, -3.8868) .. (7.1919, -3.8785).. controls (7.1845, -3.8703) and (7.1788, -3.8606) .. (7.1748, -3.8495).. controls (7.1709, -3.8384) and (7.1689, -3.8264) .. (7.1689, -3.8135).. controls (7.1689, -3.8004) and (7.1709, -3.7881) .. (7.1748, -3.7768).. controls (7.1788, -3.7656) and (7.1845, -3.7557) .. (7.1919, -3.7473).. controls (7.1993, -3.7389) and (7.2085, -3.7323) .. (7.2192, -3.7276).. controls (7.23, -3.7228) and (7.2422, -3.7204) .. (7.2557, -3.7204).. controls (7.2654, -3.7204) and (7.2746, -3.7218) .. (7.2832, -3.7246).. controls (7.2919, -3.7274) and (7.2996, -3.7315) .. (7.3064, -3.7369).. controls (7.3133, -3.7422) and (7.3189, -3.7488) .. (7.3234, -3.7567).. controls (7.3278, -3.7647) and (7.3306, -3.7737) .. (7.3318, -3.7839) -- (7.2942, -3.7839).. controls (7.2936, -3.7795) and (7.2921, -3.7754) .. (7.2898, -3.7717) -- cycle;



    \begin{scope}[cm={ 1.1468,-0.0,-0.0,1.1468,(12.1163, -4.5593)}]
      \path[fill=white] (-4.0784, 1.1402).. controls (-4.0783, 1.0125) and (-4.1818, 0.9089) .. (-4.3095, 0.9088).. controls (-4.4372, 0.9087) and (-4.5408, 1.0122) .. (-4.5409, 1.1399).. controls (-4.5409, 1.14) and (-4.5409, 1.1401) .. (-4.5409, 1.1402).. controls (-4.541, 1.2679) and (-4.4375, 1.3715) .. (-4.3098, 1.3716).. controls (-4.1821, 1.3717) and (-4.0785, 1.2682) .. (-4.0784, 1.1406).. controls (-4.0784, 1.1404) and (-4.0784, 1.1403) .. (-4.0784, 1.1402) -- cycle;



      \path[fill] (-4.31, 1.4094).. controls (-4.2345, 1.4094) and (-4.1707, 1.3834) .. (-4.1186, 1.3314).. controls (-4.0665, 1.2793) and (-4.0404, 1.2156) .. (-4.0404, 1.1402).. controls (-4.0404, 1.0648) and (-4.0661, 1.0018) .. (-4.1173, 0.9511).. controls (-4.1716, 0.8977) and (-4.2359, 0.871) .. (-4.31, 0.871).. controls (-4.3832, 0.871) and (-4.4463, 0.8975) .. (-4.4994, 0.9504).. controls (-4.5524, 1.0033) and (-4.5789, 1.0666) .. (-4.5789, 1.1402).. controls (-4.5789, 1.2138) and (-4.5524, 1.2775) .. (-4.4994, 1.3314).. controls (-4.4477, 1.3834) and (-4.3846, 1.4094) .. (-4.31, 1.4094) -- cycle(-4.5183, 1.2125).. controls (-4.5263, 1.1898) and (-4.5303, 1.1657) .. (-4.5303, 1.1402).. controls (-4.5303, 1.0805) and (-4.5086, 1.029) .. (-4.465, 0.9854).. controls (-4.4214, 0.9419) and (-4.3695, 0.9201) .. (-4.3093, 0.9201).. controls (-4.2491, 0.9201) and (-4.1968, 0.9421) .. (-4.1523, 0.9861).. controls (-4.1374, 1.0005) and (-4.1251, 1.0162) .. (-4.1155, 1.0332) -- (-4.217, 1.0784).. controls (-4.2239, 1.0442) and (-4.2543, 1.0212) .. (-4.2911, 1.0185) -- (-4.2911, 0.977) -- (-4.322, 0.977) -- (-4.322, 1.0185).. controls (-4.3522, 1.0188) and (-4.3814, 1.0312) .. (-4.4037, 1.0507) -- (-4.3667, 1.0881).. controls (-4.3488, 1.0713) and (-4.3309, 1.0637) .. (-4.3066, 1.0637).. controls (-4.2908, 1.0637) and (-4.2733, 1.0699) .. (-4.2733, 1.0905).. controls (-4.2733, 1.0978) and (-4.2761, 1.1029) .. (-4.2805, 1.1067) -- (-4.3062, 1.1181) -- (-4.3382, 1.1323).. controls (-4.354, 1.1394) and (-4.3673, 1.1453) .. (-4.3808, 1.1513) -- (-4.5183, 1.2125) -- cycle(-4.3093, 1.361).. controls (-4.3704, 1.361) and (-4.4221, 1.3394) .. (-4.4643, 1.2964).. controls (-4.4758, 1.2847) and (-4.4858, 1.2726) .. (-4.4943, 1.26) -- (-4.3914, 1.2142).. controls (-4.3821, 1.2427) and (-4.355, 1.26) .. (-4.322, 1.262) -- (-4.322, 1.3035) -- (-4.2911, 1.3035) -- (-4.2911, 1.262).. controls (-4.2698, 1.2609) and (-4.2465, 1.2551) .. (-4.2235, 1.2373) -- (-4.2588, 1.2009).. controls (-4.2719, 1.2102) and (-4.2884, 1.2167) .. (-4.3048, 1.2167).. controls (-4.3182, 1.2167) and (-4.3371, 1.2126) .. (-4.3371, 1.1958).. controls (-4.3371, 1.1932) and (-4.3363, 1.191) .. (-4.3347, 1.1889) -- (-4.3003, 1.1736) -- (-4.277, 1.1632).. controls (-4.2621, 1.1566) and (-4.2478, 1.1503) .. (-4.2337, 1.144) -- (-4.0958, 1.0826).. controls (-4.0912, 1.1006) and (-4.089, 1.1199) .. (-4.089, 1.1402).. controls (-4.089, 1.2017) and (-4.1105, 1.2537) .. (-4.1536, 1.2964).. controls (-4.1963, 1.3394) and (-4.2482, 1.361) .. (-4.3093, 1.361) -- cycle;



    \end{scope}
    \path[fill=white] (6.4178, -3.258) circle (0.2859cm);



    \path[fill] (6.5005, -3.1752).. controls (6.5005, -3.1642) and (6.4916, -3.1553) .. (6.4806, -3.1553) -- (6.3543, -3.1553).. controls (6.3433, -3.1553) and (6.3344, -3.1642) .. (6.3344, -3.1752) -- (6.3344, -3.3015) -- (6.3696, -3.3015) -- (6.3696, -3.4511) -- (6.4653, -3.4511) -- (6.4653, -3.3015) -- (6.5005, -3.3015) -- (6.5005, -3.1752) -- (6.5005, -3.1752) -- cycle;



    \path[fill] (6.4175, -3.0955) circle (0.0432cm);



    \path[fill,even odd rule] (6.4171, -2.9429).. controls (6.3316, -2.9429) and (6.2592, -2.9728) .. (6.1999, -3.0325).. controls (6.1391, -3.0942) and (6.1087, -3.1673) .. (6.1087, -3.2517).. controls (6.1087, -3.336) and (6.1391, -3.4086) .. (6.1999, -3.4693).. controls (6.2607, -3.53) and (6.3331, -3.5604) .. (6.4171, -3.5604).. controls (6.5021, -3.5604) and (6.5758, -3.5298) .. (6.6381, -3.4686).. controls (6.6968, -3.4104) and (6.7262, -3.3381) .. (6.7262, -3.2517).. controls (6.7262, -3.1652) and (6.6963, -3.0922) .. (6.6366, -3.0325).. controls (6.5768, -2.9728) and (6.5036, -2.9429) .. (6.4171, -2.9429) -- cycle(6.4178, -2.9985).. controls (6.4879, -2.9985) and (6.5474, -3.0232) .. (6.5964, -3.0726).. controls (6.6458, -3.1215) and (6.6706, -3.1812) .. (6.6706, -3.2517).. controls (6.6706, -3.3227) and (6.6464, -3.3816) .. (6.5979, -3.4284).. controls (6.5469, -3.4789) and (6.4869, -3.5041) .. (6.4178, -3.5041).. controls (6.3488, -3.5041) and (6.2893, -3.4791) .. (6.2393, -3.4292).. controls (6.1893, -3.3793) and (6.1644, -3.3201) .. (6.1644, -3.2517).. controls (6.1644, -3.1832) and (6.1896, -3.1235) .. (6.2401, -3.0726).. controls (6.2885, -3.0232) and (6.3478, -2.9985) .. (6.4178, -2.9985) -- cycle;



  \end{scope}

\end{tikzpicture}
 \\ {\tt cc-by-nc}

        \column{0.25\textwidth}
        \centering

        
\definecolor{caab2ab}{RGB}{170,178,171}


\def \globalscale {0.550000}
\begin{tikzpicture}[y=1cm, x=1cm, yscale=\globalscale,xscale=\globalscale, every node/.append style={scale=\globalscale}, inner sep=0pt, outer sep=0pt]
  \begin{scope}[cm={ 0.9938,-0.0,-0.0,0.9937,(-4.7015, 5.9154)}]
    \path[fill=caab2ab] (4.8216, -4.8464) -- (7.8396, -4.8518).. controls (7.8818, -4.8518) and (7.9194, -4.8455) .. (7.9194, -4.936) -- (7.9157, -5.9299) -- (4.7455, -5.9299) -- (4.7455, -4.9323).. controls (4.7455, -4.8877) and (4.7498, -4.8464) .. (4.8216, -4.8464) -- cycle;



    \begin{scope}[cm={ 0.8729,-0.0,-0.0,0.8729,(1.3262, -3.648)}]
      \path[fill=white] (4.9533, -1.9555).. controls (4.9534, -2.1851) and (4.7673, -2.3714) .. (4.5376, -2.3716).. controls (4.308, -2.3717) and (4.1217, -2.1856) .. (4.1215, -1.956).. controls (4.1215, -1.9558) and (4.1215, -1.9556) .. (4.1215, -1.9555).. controls (4.1214, -1.7258) and (4.3075, -1.5395) .. (4.5371, -1.5394).. controls (4.7668, -1.5392) and (4.9531, -1.7253) .. (4.9533, -1.955).. controls (4.9533, -1.9551) and (4.9533, -1.9553) .. (4.9533, -1.9555) -- cycle;



      \begin{scope}[shift={(-7.6627, -2.6211)}]
        \path[fill] (12.5382, 1.0043).. controls (12.6304, 0.9121) and (12.6765, 0.7992) .. (12.6765, 0.6656).. controls (12.6765, 0.532) and (12.6312, 0.4203) .. (12.5406, 0.3304).. controls (12.4444, 0.2358) and (12.3307, 0.1885) .. (12.1995, 0.1885).. controls (12.07, 0.1885) and (11.9582, 0.2354) .. (11.8644, 0.3293).. controls (11.7706, 0.4231) and (11.7237, 0.5352) .. (11.7237, 0.6656).. controls (11.7237, 0.796) and (11.7706, 0.9089) .. (11.8644, 1.0043).. controls (11.9559, 1.0966) and (12.0676, 1.1427) .. (12.1995, 1.1427).. controls (12.3331, 1.1427) and (12.446, 1.0966) .. (12.5382, 1.0043) -- cycle(11.9265, 0.9423).. controls (11.8486, 0.8636) and (11.8096, 0.7713) .. (11.8096, 0.6655).. controls (11.8096, 0.5597) and (11.8482, 0.4683) .. (11.9253, 0.3911).. controls (12.0025, 0.314) and (12.0944, 0.2754) .. (12.201, 0.2754).. controls (12.3075, 0.2754) and (12.4002, 0.3143) .. (12.4789, 0.3923).. controls (12.5537, 0.4647) and (12.5911, 0.5557) .. (12.5911, 0.6655).. controls (12.5911, 0.7745) and (12.5531, 0.8669) .. (12.4771, 0.9429).. controls (12.4012, 1.0188) and (12.3091, 1.0569) .. (12.201, 1.0569).. controls (12.0928, 1.0569) and (12.0013, 1.0187) .. (11.9265, 0.9423) -- cycle(12.1317, 0.712).. controls (12.1198, 0.738) and (12.102, 0.751) .. (12.0782, 0.751).. controls (12.0361, 0.751) and (12.0151, 0.7227) .. (12.0151, 0.6661).. controls (12.0151, 0.6095) and (12.0361, 0.5812) .. (12.0782, 0.5812).. controls (12.1059, 0.5812) and (12.1258, 0.595) .. (12.1377, 0.6226) -- (12.1959, 0.5916).. controls (12.1682, 0.5422) and (12.1265, 0.5175) .. (12.0709, 0.5175).. controls (12.0281, 0.5175) and (11.9937, 0.5307) .. (11.9679, 0.5569).. controls (11.9421, 0.5832) and (11.9292, 0.6195) .. (11.9292, 0.6656).. controls (11.9292, 0.711) and (11.9425, 0.747) .. (11.9691, 0.7737).. controls (11.9957, 0.8004) and (12.0288, 0.8137) .. (12.0686, 0.8137).. controls (12.1273, 0.8137) and (12.1694, 0.7906) .. (12.1948, 0.7443) -- (12.1317, 0.712) -- cycle(12.4059, 0.712).. controls (12.394, 0.738) and (12.3765, 0.751) .. (12.3534, 0.751).. controls (12.3106, 0.751) and (12.2891, 0.7227) .. (12.2891, 0.6661).. controls (12.2891, 0.6095) and (12.3106, 0.5812) .. (12.3534, 0.5812).. controls (12.3813, 0.5812) and (12.4007, 0.595) .. (12.4118, 0.6226) -- (12.4714, 0.5916).. controls (12.4437, 0.5422) and (12.4021, 0.5175) .. (12.3466, 0.5175).. controls (12.3038, 0.5175) and (12.2695, 0.5307) .. (12.2438, 0.5569).. controls (12.2181, 0.5832) and (12.2052, 0.6195) .. (12.2052, 0.6656).. controls (12.2052, 0.711) and (12.2183, 0.747) .. (12.2444, 0.7737).. controls (12.2705, 0.8004) and (12.3038, 0.8137) .. (12.3442, 0.8137).. controls (12.4029, 0.8137) and (12.4449, 0.7906) .. (12.4702, 0.7443) -- (12.4059, 0.712) -- cycle;



      \end{scope}
    \end{scope}
    \path[fill=white] (6.4178, -5.2389) circle (0.2859cm);



    \path[fill] (6.5005, -5.1562).. controls (6.5005, -5.1452) and (6.4916, -5.1362) .. (6.4806, -5.1362) -- (6.3543, -5.1362).. controls (6.3433, -5.1362) and (6.3344, -5.1452) .. (6.3344, -5.1562) -- (6.3344, -5.2825) -- (6.3696, -5.2825) -- (6.3696, -5.432) -- (6.4653, -5.432) -- (6.4653, -5.2825) -- (6.5005, -5.2825) -- (6.5005, -5.1562) -- (6.5005, -5.1562) -- cycle;



    \path[fill] (6.4175, -5.0764) circle (0.0432cm);



    \path[fill,even odd rule] (6.4171, -4.9239).. controls (6.3316, -4.9239) and (6.2592, -4.9537) .. (6.1999, -5.0134).. controls (6.1391, -5.0752) and (6.1087, -5.1483) .. (6.1087, -5.2326).. controls (6.1087, -5.317) and (6.1391, -5.3896) .. (6.1999, -5.4503).. controls (6.2607, -5.511) and (6.3331, -5.5414) .. (6.4171, -5.5414).. controls (6.5021, -5.5414) and (6.5758, -5.5107) .. (6.6381, -5.4495).. controls (6.6968, -5.3914) and (6.7262, -5.3191) .. (6.7262, -5.2326).. controls (6.7262, -5.1462) and (6.6963, -5.0731) .. (6.6366, -5.0134).. controls (6.5768, -4.9537) and (6.5036, -4.9239) .. (6.4171, -4.9239) -- cycle(6.4178, -4.9795).. controls (6.4879, -4.9795) and (6.5474, -5.0042) .. (6.5964, -5.0536).. controls (6.6458, -5.1024) and (6.6706, -5.1621) .. (6.6706, -5.2326).. controls (6.6706, -5.3036) and (6.6464, -5.3626) .. (6.5979, -5.4094).. controls (6.5469, -5.4598) and (6.4869, -5.485) .. (6.4178, -5.485).. controls (6.3488, -5.485) and (6.2893, -5.4601) .. (6.2393, -5.4102).. controls (6.1893, -5.3602) and (6.1644, -5.3011) .. (6.1644, -5.2326).. controls (6.1644, -5.1642) and (6.1896, -5.1045) .. (6.2401, -5.0536).. controls (6.2885, -5.0042) and (6.3478, -4.9795) .. (6.4178, -4.9795) -- cycle;



    \path[fill] (7.866, -4.8348) -- (4.7907, -4.8348).. controls (4.7578, -4.8348) and (4.7309, -4.8616) .. (4.7309, -4.8946) -- (4.7309, -5.9396).. controls (4.7309, -5.9471) and (4.737, -5.9531) .. (4.7444, -5.9531) -- (7.9123, -5.9531).. controls (7.9197, -5.9531) and (7.9258, -5.9471) .. (7.9258, -5.9396) -- (7.9258, -4.8946).. controls (7.9258, -4.8616) and (7.8989, -4.8348) .. (7.866, -4.8348) -- cycle(4.7907, -4.8618) -- (7.866, -4.8618).. controls (7.8841, -4.8618) and (7.8988, -4.8765) .. (7.8988, -4.8946).. controls (7.8988, -4.8946) and (7.8988, -5.3148) .. (7.8988, -5.619) -- (5.7008, -5.619).. controls (5.6202, -5.7647) and (5.465, -5.8636) .. (5.2869, -5.8636).. controls (5.1087, -5.8636) and (4.9535, -5.7648) .. (4.873, -5.619) -- (4.7579, -5.619).. controls (4.7579, -5.3148) and (4.7579, -4.8946) .. (4.7579, -4.8946).. controls (4.7579, -4.8765) and (4.7727, -4.8618) .. (4.7907, -4.8618) -- cycle;



    \path[fill=white] (6.3283, -5.7068).. controls (6.3367, -5.7068) and (6.3443, -5.7075) .. (6.3513, -5.709).. controls (6.3582, -5.7105) and (6.3641, -5.7129) .. (6.3691, -5.7163).. controls (6.374, -5.7196) and (6.3779, -5.7241) .. (6.3806, -5.7297).. controls (6.3833, -5.7353) and (6.3847, -5.7422) .. (6.3847, -5.7504).. controls (6.3847, -5.7593) and (6.3826, -5.7667) .. (6.3786, -5.7727).. controls (6.3746, -5.7786) and (6.3686, -5.7834) .. (6.3607, -5.7872).. controls (6.3716, -5.7903) and (6.3797, -5.7958) .. (6.3851, -5.8037).. controls (6.3904, -5.8115) and (6.3931, -5.8209) .. (6.3931, -5.8319).. controls (6.3931, -5.8408) and (6.3914, -5.8485) .. (6.3879, -5.855).. controls (6.3845, -5.8616) and (6.3798, -5.8669) .. (6.374, -5.871).. controls (6.3681, -5.8751) and (6.3614, -5.8781) .. (6.3539, -5.8801).. controls (6.3464, -5.8821) and (6.3387, -5.8831) .. (6.3308, -5.8831) -- (6.2451, -5.8831) -- (6.2451, -5.7068) -- (6.3283, -5.7068) -- (6.3283, -5.7068) -- cycle(6.3233, -5.7781).. controls (6.3302, -5.7781) and (6.3359, -5.7764) .. (6.3404, -5.7731).. controls (6.3448, -5.7699) and (6.347, -5.7645) .. (6.347, -5.7571).. controls (6.347, -5.753) and (6.3463, -5.7497) .. (6.3448, -5.747).. controls (6.3433, -5.7444) and (6.3414, -5.7424) .. (6.3389, -5.7409).. controls (6.3364, -5.7394) and (6.3336, -5.7384) .. (6.3303, -5.7378).. controls (6.3271, -5.7372) and (6.3238, -5.7369) .. (6.3203, -5.7369) -- (6.284, -5.7369) -- (6.284, -5.7781) -- (6.3233, -5.7781) -- cycle(6.3256, -5.8529).. controls (6.3294, -5.8529) and (6.333, -5.8525) .. (6.3365, -5.8518).. controls (6.3399, -5.851) and (6.343, -5.8498) .. (6.3456, -5.8481).. controls (6.3483, -5.8464) and (6.3503, -5.844) .. (6.3519, -5.8411).. controls (6.3535, -5.8381) and (6.3543, -5.8343) .. (6.3543, -5.8297).. controls (6.3543, -5.8207) and (6.3517, -5.8142) .. (6.3466, -5.8103).. controls (6.3415, -5.8065) and (6.3347, -5.8045) .. (6.3263, -5.8045) -- (6.284, -5.8045) -- (6.284, -5.8529) -- (6.3256, -5.8529) -- (6.3256, -5.8529) -- cycle;



    \path[fill=white] (6.4, -5.7068) -- (6.4435, -5.7068) -- (6.4848, -5.7764) -- (6.5258, -5.7068) -- (6.569, -5.7068) -- (6.5036, -5.8154) -- (6.5036, -5.8831) -- (6.4647, -5.8831) -- (6.4647, -5.8144) -- (6.4, -5.7068) -- cycle;



    \path[fill=white] (7.0188, -5.7068) -- (7.0924, -5.825) -- (7.0929, -5.825) -- (7.0929, -5.7068) -- (7.1292, -5.7068) -- (7.1292, -5.8831) -- (7.0905, -5.8831) -- (7.0171, -5.765) -- (7.0166, -5.765) -- (7.0166, -5.8831) -- (6.9802, -5.8831) -- (6.9802, -5.7068) -- (7.0188, -5.7068) -- cycle;



    \path[fill=white] (7.2394, -5.7068).. controls (7.2508, -5.7068) and (7.2613, -5.7086) .. (7.2711, -5.7122).. controls (7.281, -5.7158) and (7.2894, -5.7213) .. (7.2966, -5.7285).. controls (7.3038, -5.7357) and (7.3094, -5.7448) .. (7.3134, -5.7556).. controls (7.3175, -5.7665) and (7.3195, -5.7793) .. (7.3195, -5.7939).. controls (7.3195, -5.8068) and (7.3178, -5.8186) .. (7.3146, -5.8295).. controls (7.3112, -5.8404) and (7.3063, -5.8497) .. (7.2996, -5.8577).. controls (7.2929, -5.8655) and (7.2846, -5.8718) .. (7.2746, -5.8763).. controls (7.2646, -5.8808) and (7.2529, -5.8831) .. (7.2394, -5.8831) -- (7.1632, -5.8831) -- (7.1632, -5.7068) -- (7.2394, -5.7068) -- (7.2394, -5.7068) -- cycle(7.2367, -5.8504).. controls (7.2423, -5.8504) and (7.2477, -5.8495) .. (7.253, -5.8477).. controls (7.2582, -5.8459) and (7.2629, -5.8429) .. (7.267, -5.8387).. controls (7.2711, -5.8345) and (7.2745, -5.829) .. (7.2769, -5.8223).. controls (7.2794, -5.8156) and (7.2806, -5.8073) .. (7.2806, -5.7976).. controls (7.2806, -5.7887) and (7.2798, -5.7807) .. (7.278, -5.7736).. controls (7.2763, -5.7664) and (7.2735, -5.7603) .. (7.2695, -5.7552).. controls (7.2656, -5.7501) and (7.2603, -5.7462) .. (7.2538, -5.7435).. controls (7.2473, -5.7408) and (7.2393, -5.7394) .. (7.2298, -5.7394) -- (7.2021, -5.7394) -- (7.2021, -5.8504) -- (7.2367, -5.8504) -- (7.2367, -5.8504) -- cycle;



    \begin{scope}[cm={ 0.625,-0.0,-0.0,0.625,(-4.8422, -7.9688)}]
      \path[fill=white] (19.6831, 4.3762).. controls (19.6833, 4.1272) and (19.4816, 3.9252) .. (19.2325, 3.925).. controls (18.9835, 3.9248) and (18.7815, 4.1265) .. (18.7813, 4.3756).. controls (18.7813, 4.3758) and (18.7813, 4.376) .. (18.7813, 4.3762).. controls (18.7811, 4.6252) and (18.9829, 4.8272) .. (19.2319, 4.8274).. controls (19.4809, 4.8277) and (19.6829, 4.6259) .. (19.6831, 4.3769).. controls (19.6831, 4.3767) and (19.6831, 4.3764) .. (19.6831, 4.3762) -- cycle;



      \begin{scope}[shift={(-0.6337, 2.3262)}]
        \path[fill] (19.8589, 2.5439).. controls (19.722, 2.5439) and (19.6062, 2.4962) .. (19.5114, 2.4007).. controls (19.4142, 2.3019) and (19.3655, 2.185) .. (19.3655, 2.05).. controls (19.3655, 1.9149) and (19.4142, 1.7989) .. (19.5114, 1.7017).. controls (19.6087, 1.6045) and (19.7245, 1.556) .. (19.8589, 1.556).. controls (19.9949, 1.556) and (20.1128, 1.605) .. (20.2125, 1.7029).. controls (20.3065, 1.796) and (20.3535, 1.9117) .. (20.3535, 2.05).. controls (20.3535, 2.1883) and (20.3057, 2.3052) .. (20.2101, 2.4007).. controls (20.1144, 2.4962) and (19.9974, 2.5439) .. (19.8589, 2.5439) -- cycle(19.8601, 2.4551).. controls (19.9722, 2.4551) and (20.0674, 2.4155) .. (20.1458, 2.3365).. controls (20.2249, 2.2583) and (20.2645, 2.1628) .. (20.2645, 2.05).. controls (20.2645, 1.9364) and (20.2257, 1.8421) .. (20.1482, 1.7672).. controls (20.0666, 1.6865) and (19.9706, 1.6462) .. (19.8601, 1.6462).. controls (19.7496, 1.6462) and (19.6544, 1.6861) .. (19.5745, 1.766).. controls (19.4945, 1.8458) and (19.4545, 1.9405) .. (19.4545, 2.05).. controls (19.4545, 2.1595) and (19.4949, 2.255) .. (19.5757, 2.3365).. controls (19.6532, 2.4155) and (19.748, 2.4551) .. (19.8601, 2.4551) -- cycle;



        \path[fill] (20.0462, 2.167) -- (19.687, 2.167) -- (19.687, 2.082) -- (20.0462, 2.082) -- (20.0462, 2.167) -- cycle(20.0462, 2.0083) -- (19.687, 2.0083) -- (19.687, 1.9232) -- (20.0462, 1.9232) -- (20.0462, 2.0083) -- cycle;



      \end{scope}
    \end{scope}
  \end{scope}

\end{tikzpicture}
 \\ {\tt cc-by-nd}

        \vspace{3pt}
\definecolor{caab2ab}{RGB}{170,178,171}


\def \globalscale {0.550000}
\begin{tikzpicture}[y=1cm, x=1cm, yscale=\globalscale,xscale=\globalscale, every node/.append style={scale=\globalscale}, inner sep=0pt, outer sep=0pt]
  \begin{scope}[cm={ 0.9938,-0.0,-0.0,0.9937,(-4.7015, 1.9765)}]
    \path[fill=caab2ab] (4.8216, -0.8834) -- (7.8396, -0.8888).. controls (7.8818, -0.8888) and (7.9194, -0.8826) .. (7.9194, -0.973) -- (7.9157, -1.9669) -- (4.7455, -1.9669) -- (4.7455, -0.9693).. controls (4.7455, -0.9247) and (4.7498, -0.8834) .. (4.8216, -0.8834) -- cycle;



    \begin{scope}[cm={ 0.8729,-0.0,-0.0,0.8729,(1.3262, -3.648)}]
      \path[fill=white] (4.9532, 2.5844).. controls (4.9534, 2.3548) and (4.7673, 2.1685) .. (4.5376, 2.1683).. controls (4.308, 2.1682) and (4.1217, 2.3543) .. (4.1215, 2.5839).. controls (4.1215, 2.5841) and (4.1215, 2.5843) .. (4.1215, 2.5844).. controls (4.1214, 2.8141) and (4.3075, 3.0004) .. (4.5371, 3.0005).. controls (4.7668, 3.0007) and (4.9531, 2.8146) .. (4.9532, 2.5849).. controls (4.9532, 2.5848) and (4.9532, 2.5846) .. (4.9532, 2.5844) -- cycle;



      \begin{scope}[shift={(-7.6627, -2.6211)}]
        \path[fill] (12.5382, 5.5443).. controls (12.6304, 5.452) and (12.6765, 5.3391) .. (12.6765, 5.2055).. controls (12.6765, 5.0719) and (12.6312, 4.9602) .. (12.5406, 4.8704).. controls (12.4444, 4.7757) and (12.3307, 4.7284) .. (12.1995, 4.7284).. controls (12.07, 4.7284) and (11.9582, 4.7753) .. (11.8644, 4.8692).. controls (11.7706, 4.963) and (11.7237, 5.0751) .. (11.7237, 5.2055).. controls (11.7237, 5.3359) and (11.7706, 5.4488) .. (11.8644, 5.5443).. controls (11.9559, 5.6365) and (12.0676, 5.6826) .. (12.1995, 5.6826).. controls (12.3331, 5.6826) and (12.446, 5.6365) .. (12.5382, 5.5443) -- cycle(11.9265, 5.4822).. controls (11.8486, 5.4035) and (11.8096, 5.3112) .. (11.8096, 5.2054).. controls (11.8096, 5.0996) and (11.8482, 5.0082) .. (11.9253, 4.931).. controls (12.0025, 4.8539) and (12.0944, 4.8153) .. (12.201, 4.8153).. controls (12.3075, 4.8153) and (12.4002, 4.8542) .. (12.4789, 4.9322).. controls (12.5537, 5.0046) and (12.5911, 5.0956) .. (12.5911, 5.2054).. controls (12.5911, 5.3144) and (12.5531, 5.4068) .. (12.4771, 5.4828).. controls (12.4012, 5.5588) and (12.3091, 5.5968) .. (12.201, 5.5968).. controls (12.0928, 5.5968) and (12.0013, 5.5586) .. (11.9265, 5.4822) -- cycle(12.1317, 5.252).. controls (12.1198, 5.2779) and (12.102, 5.2909) .. (12.0782, 5.2909).. controls (12.0361, 5.2909) and (12.0151, 5.2626) .. (12.0151, 5.206).. controls (12.0151, 5.1494) and (12.0361, 5.1211) .. (12.0782, 5.1211).. controls (12.1059, 5.1211) and (12.1258, 5.1349) .. (12.1377, 5.1625) -- (12.1959, 5.1315).. controls (12.1682, 5.0821) and (12.1265, 5.0574) .. (12.0709, 5.0574).. controls (12.0281, 5.0574) and (11.9937, 5.0706) .. (11.9679, 5.0968).. controls (11.9421, 5.1231) and (11.9292, 5.1594) .. (11.9292, 5.2055).. controls (11.9292, 5.2509) and (11.9425, 5.2869) .. (11.9691, 5.3136).. controls (11.9957, 5.3403) and (12.0288, 5.3536) .. (12.0686, 5.3536).. controls (12.1273, 5.3536) and (12.1694, 5.3305) .. (12.1948, 5.2843) -- (12.1317, 5.252) -- cycle(12.4059, 5.252).. controls (12.394, 5.2779) and (12.3765, 5.2909) .. (12.3534, 5.2909).. controls (12.3106, 5.2909) and (12.2891, 5.2626) .. (12.2891, 5.206).. controls (12.2891, 5.1494) and (12.3106, 5.1211) .. (12.3534, 5.1211).. controls (12.3813, 5.1211) and (12.4007, 5.1349) .. (12.4118, 5.1625) -- (12.4714, 5.1315).. controls (12.4437, 5.0821) and (12.4021, 5.0574) .. (12.3466, 5.0574).. controls (12.3038, 5.0574) and (12.2695, 5.0706) .. (12.2438, 5.0968).. controls (12.2181, 5.1231) and (12.2052, 5.1594) .. (12.2052, 5.2055).. controls (12.2052, 5.2509) and (12.2183, 5.2869) .. (12.2444, 5.3136).. controls (12.2705, 5.3403) and (12.3038, 5.3536) .. (12.3442, 5.3536).. controls (12.4029, 5.3536) and (12.4449, 5.3305) .. (12.4702, 5.2843) -- (12.4059, 5.252) -- cycle;



      \end{scope}
    \end{scope}
    \path[fill] (7.866, -0.8708) -- (4.7907, -0.8708).. controls (4.7577, -0.8708) and (4.7309, -0.8976) .. (4.7309, -0.9306) -- (4.7309, -1.9756).. controls (4.7309, -1.983) and (4.737, -1.9891) .. (4.7444, -1.9891) -- (7.9123, -1.9891).. controls (7.9197, -1.9891) and (7.9258, -1.983) .. (7.9258, -1.9756) -- (7.9258, -0.9306).. controls (7.9258, -0.8976) and (7.8989, -0.8708) .. (7.866, -0.8708) -- cycle(4.7907, -0.8978) -- (7.866, -0.8978).. controls (7.8841, -0.8978) and (7.8988, -0.9125) .. (7.8988, -0.9306).. controls (7.8988, -0.9306) and (7.8988, -1.3518) .. (7.8988, -1.656) -- (5.7008, -1.656).. controls (5.6202, -1.8017) and (5.465, -1.9006) .. (5.2869, -1.9006).. controls (5.1087, -1.9006) and (4.9535, -1.8018) .. (4.873, -1.656) -- (4.7579, -1.656).. controls (4.7579, -1.3518) and (4.7579, -0.9306) .. (4.7579, -0.9306).. controls (4.7579, -0.9125) and (4.7726, -0.8978) .. (4.7907, -0.8978) -- cycle;



    \path[fill=white] (7.0276, -1.8755).. controls (7.0297, -1.8796) and (7.0326, -1.8829) .. (7.0361, -1.8855).. controls (7.0396, -1.888) and (7.0438, -1.8899) .. (7.0485, -1.8911).. controls (7.0533, -1.8924) and (7.0582, -1.893) .. (7.0633, -1.893).. controls (7.0668, -1.893) and (7.0705, -1.8927) .. (7.0744, -1.8921).. controls (7.0783, -1.8916) and (7.082, -1.8904) .. (7.0855, -1.8888).. controls (7.0889, -1.8872) and (7.0918, -1.8849) .. (7.0941, -1.882).. controls (7.0964, -1.8791) and (7.0976, -1.8755) .. (7.0976, -1.871).. controls (7.0976, -1.8662) and (7.096, -1.8624) .. (7.093, -1.8594).. controls (7.09, -1.8564) and (7.086, -1.854) .. (7.081, -1.852).. controls (7.0761, -1.85) and (7.0705, -1.8483) .. (7.0643, -1.8468).. controls (7.058, -1.8453) and (7.0517, -1.8437) .. (7.0453, -1.8419).. controls (7.0387, -1.8402) and (7.0323, -1.8382) .. (7.026, -1.8358).. controls (7.0198, -1.8334) and (7.0142, -1.8303) .. (7.0093, -1.8266).. controls (7.0043, -1.8228) and (7.0004, -1.818) .. (6.9973, -1.8124).. controls (6.9943, -1.8067) and (6.9928, -1.7998) .. (6.9928, -1.7917).. controls (6.9928, -1.7827) and (6.9947, -1.7748) .. (6.9986, -1.7681).. controls (7.0024, -1.7615) and (7.0075, -1.7559) .. (7.0137, -1.7515).. controls (7.02, -1.747) and (7.0271, -1.7437) .. (7.035, -1.7416).. controls (7.0428, -1.7395) and (7.0507, -1.7384) .. (7.0586, -1.7384).. controls (7.0679, -1.7384) and (7.0767, -1.7394) .. (7.0852, -1.7415).. controls (7.0936, -1.7435) and (7.1012, -1.7469) .. (7.1077, -1.7515).. controls (7.1143, -1.7561) and (7.1195, -1.762) .. (7.1234, -1.7692).. controls (7.1273, -1.7763) and (7.1292, -1.785) .. (7.1292, -1.7952) -- (7.0916, -1.7952).. controls (7.0912, -1.79) and (7.0901, -1.7856) .. (7.0883, -1.7822).. controls (7.0864, -1.7787) and (7.0839, -1.776) .. (7.0808, -1.774).. controls (7.0776, -1.772) and (7.0741, -1.7706) .. (7.0701, -1.7698).. controls (7.066, -1.769) and (7.0617, -1.7686) .. (7.0569, -1.7686).. controls (7.0538, -1.7686) and (7.0507, -1.7689) .. (7.0476, -1.7695).. controls (7.0445, -1.7702) and (7.0417, -1.7714) .. (7.0391, -1.773).. controls (7.0366, -1.7747) and (7.0345, -1.7767) .. (7.0328, -1.7792).. controls (7.0312, -1.7817) and (7.0304, -1.7848) .. (7.0304, -1.7886).. controls (7.0304, -1.792) and (7.031, -1.7948) .. (7.0323, -1.797).. controls (7.0337, -1.7991) and (7.0363, -1.8011) .. (7.0401, -1.8029).. controls (7.044, -1.8047) and (7.0493, -1.8065) .. (7.0561, -1.8083).. controls (7.063, -1.8101) and (7.0719, -1.8125) .. (7.0829, -1.8153).. controls (7.0862, -1.8159) and (7.0907, -1.8171) .. (7.0966, -1.8188).. controls (7.1024, -1.8206) and (7.1082, -1.8233) .. (7.114, -1.8271).. controls (7.1197, -1.8309) and (7.1247, -1.836) .. (7.1289, -1.8423).. controls (7.1331, -1.8486) and (7.1352, -1.8568) .. (7.1352, -1.8666).. controls (7.1352, -1.8747) and (7.1336, -1.8822) .. (7.1305, -1.8891).. controls (7.1274, -1.896) and (7.1227, -1.902) .. (7.1165, -1.907).. controls (7.1104, -1.9121) and (7.1027, -1.916) .. (7.0936, -1.9188).. controls (7.0844, -1.9216) and (7.0738, -1.923) .. (7.0619, -1.923).. controls (7.0521, -1.923) and (7.0427, -1.9218) .. (7.0336, -1.9194).. controls (7.0244, -1.917) and (7.0164, -1.9132) .. (7.0094, -1.9081).. controls (7.0024, -1.903) and (6.9968, -1.8965) .. (6.9927, -1.8886).. controls (6.9886, -1.8806) and (6.9866, -1.8713) .. (6.9868, -1.8604) -- (7.0244, -1.8604).. controls (7.0244, -1.8663) and (7.0254, -1.8713) .. (7.0276, -1.8755) -- cycle;



    \path[fill=white] (7.2461, -1.7427) -- (7.312, -1.919) -- (7.2717, -1.919) -- (7.2584, -1.8798) -- (7.1924, -1.8798) -- (7.1786, -1.919) -- (7.1396, -1.919) -- (7.2063, -1.7427) -- (7.2461, -1.7427) -- cycle(7.2483, -1.8508) -- (7.2261, -1.7862) -- (7.2256, -1.7862) -- (7.2026, -1.8508) -- (7.2483, -1.8508) -- cycle;



    \path[fill=white] (6.3283, -1.7427).. controls (6.3367, -1.7427) and (6.3443, -1.7435) .. (6.3513, -1.7449).. controls (6.3582, -1.7464) and (6.3641, -1.7489) .. (6.3691, -1.7522).. controls (6.374, -1.7556) and (6.3779, -1.7601) .. (6.3806, -1.7657).. controls (6.3833, -1.7713) and (6.3847, -1.7782) .. (6.3847, -1.7864).. controls (6.3847, -1.7953) and (6.3826, -1.8027) .. (6.3786, -1.8086).. controls (6.3746, -1.8146) and (6.3686, -1.8194) .. (6.3607, -1.8232).. controls (6.3716, -1.8263) and (6.3797, -1.8318) .. (6.3851, -1.8396).. controls (6.3904, -1.8474) and (6.3931, -1.8569) .. (6.3931, -1.8679).. controls (6.3931, -1.8768) and (6.3914, -1.8845) .. (6.3879, -1.891).. controls (6.3845, -1.8975) and (6.3798, -1.9028) .. (6.374, -1.9069).. controls (6.3681, -1.9111) and (6.3614, -1.9141) .. (6.3539, -1.9161).. controls (6.3464, -1.9181) and (6.3387, -1.919) .. (6.3308, -1.919) -- (6.2451, -1.919) -- (6.2451, -1.7427) -- (6.3283, -1.7427) -- (6.3283, -1.7427) -- cycle(6.3233, -1.814).. controls (6.3302, -1.814) and (6.3359, -1.8124) .. (6.3404, -1.8091).. controls (6.3448, -1.8058) and (6.347, -1.8005) .. (6.347, -1.7931).. controls (6.347, -1.789) and (6.3463, -1.7856) .. (6.3448, -1.783).. controls (6.3433, -1.7804) and (6.3414, -1.7783) .. (6.3389, -1.7768).. controls (6.3364, -1.7754) and (6.3336, -1.7743) .. (6.3303, -1.7738).. controls (6.3271, -1.7732) and (6.3238, -1.7729) .. (6.3203, -1.7729) -- (6.284, -1.7729) -- (6.284, -1.814) -- (6.3233, -1.814) -- cycle(6.3256, -1.8889).. controls (6.3294, -1.8889) and (6.333, -1.8885) .. (6.3364, -1.8878).. controls (6.3399, -1.887) and (6.343, -1.8858) .. (6.3456, -1.8841).. controls (6.3483, -1.8823) and (6.3503, -1.88) .. (6.3519, -1.877).. controls (6.3535, -1.8741) and (6.3543, -1.8703) .. (6.3543, -1.8657).. controls (6.3543, -1.8566) and (6.3517, -1.8502) .. (6.3466, -1.8463).. controls (6.3415, -1.8424) and (6.3347, -1.8405) .. (6.3263, -1.8405) -- (6.284, -1.8405) -- (6.284, -1.8889) -- (6.3256, -1.8889) -- cycle;



    \path[fill=white] (6.4, -1.7427) -- (6.4435, -1.7427) -- (6.4848, -1.8124) -- (6.5258, -1.7427) -- (6.569, -1.7427) -- (6.5036, -1.8514) -- (6.5036, -1.919) -- (6.4647, -1.919) -- (6.4647, -1.8504) -- (6.4, -1.7427) -- cycle;



    \begin{scope}[cm={ 0.625,-0.0,-0.0,0.625,(10.3513, -4.2646)}]
      \path[fill=white] (-4.6304, 4.792).. controls (-4.6303, 4.543) and (-4.832, 4.341) .. (-5.081, 4.3408).. controls (-5.33, 4.3406) and (-5.532, 4.5423) .. (-5.5323, 4.7913).. controls (-5.5323, 4.7916) and (-5.5323, 4.7918) .. (-5.5323, 4.792).. controls (-5.5324, 5.041) and (-5.3307, 5.243) .. (-5.0817, 5.2433).. controls (-4.8326, 5.2434) and (-4.6306, 5.0417) .. (-4.6304, 4.7927).. controls (-4.6304, 4.7924) and (-4.6304, 4.7922) .. (-4.6304, 4.792) -- cycle;



      \begin{scope}[shift={(-0.6337, 2.3741)}]
        \path[fill] (-4.4508, 2.9119).. controls (-4.5877, 2.9119) and (-4.7035, 2.8641) .. (-4.7983, 2.7687).. controls (-4.8955, 2.6698) and (-4.9442, 2.5529) .. (-4.9442, 2.4179).. controls (-4.9442, 2.2829) and (-4.8955, 2.1668) .. (-4.7983, 2.0697).. controls (-4.701, 1.9725) and (-4.5852, 1.9239) .. (-4.4508, 1.9239).. controls (-4.3148, 1.9239) and (-4.197, 1.9729) .. (-4.0972, 2.0709).. controls (-4.0032, 2.1639) and (-3.9562, 2.2796) .. (-3.9562, 2.4179).. controls (-3.9562, 2.5562) and (-4.0041, 2.6732) .. (-4.0997, 2.7687).. controls (-4.1953, 2.8641) and (-4.3123, 2.9119) .. (-4.4508, 2.9119) -- cycle(-4.4496, 2.823).. controls (-4.3375, 2.823) and (-4.2423, 2.7835) .. (-4.1639, 2.7044).. controls (-4.0848, 2.6262) and (-4.0453, 2.5307) .. (-4.0453, 2.4179).. controls (-4.0453, 2.3043) and (-4.084, 2.21) .. (-4.1615, 2.1351).. controls (-4.2431, 2.0544) and (-4.3391, 2.0141) .. (-4.4496, 2.0141).. controls (-4.5601, 2.0141) and (-4.6553, 2.054) .. (-4.7352, 2.1339).. controls (-4.8152, 2.2137) and (-4.8552, 2.3084) .. (-4.8552, 2.4179).. controls (-4.8552, 2.5274) and (-4.8148, 2.6229) .. (-4.734, 2.7044).. controls (-4.6565, 2.7835) and (-4.5617, 2.823) .. (-4.4496, 2.823) -- cycle;



        \path[fill] (-4.6698, 2.4876).. controls (-4.6501, 2.6119) and (-4.5626, 2.6784) .. (-4.4529, 2.6784).. controls (-4.2952, 2.6784) and (-4.1991, 2.5639) .. (-4.1991, 2.4113).. controls (-4.1991, 2.2624) and (-4.3014, 2.1468) .. (-4.4554, 2.1468).. controls (-4.5614, 2.1468) and (-4.6562, 2.212) .. (-4.6735, 2.3399) -- (-4.5491, 2.3399).. controls (-4.5453, 2.2735) and (-4.5022, 2.2501) .. (-4.4406, 2.2501).. controls (-4.3704, 2.2501) and (-4.3248, 2.3153) .. (-4.3248, 2.415).. controls (-4.3248, 2.5196) and (-4.3642, 2.575) .. (-4.4382, 2.575).. controls (-4.4924, 2.575) and (-4.5392, 2.5553) .. (-4.5491, 2.4876) -- (-4.5128, 2.4878) -- (-4.6108, 2.3899) -- (-4.7088, 2.4878) -- (-4.6698, 2.4876) -- cycle;



      \end{scope}
    \end{scope}
    \path[fill=white] (6.4178, -1.2759) circle (0.2859cm);



    \path[fill] (6.5005, -1.1932).. controls (6.5005, -1.1822) and (6.4916, -1.1733) .. (6.4806, -1.1733) -- (6.3543, -1.1733).. controls (6.3433, -1.1733) and (6.3344, -1.1822) .. (6.3344, -1.1932) -- (6.3344, -1.3195) -- (6.3696, -1.3195) -- (6.3696, -1.469) -- (6.4653, -1.469) -- (6.4653, -1.3195) -- (6.5005, -1.3195) -- (6.5005, -1.1932) -- (6.5005, -1.1932) -- cycle;



    \path[fill] (6.4175, -1.1134) circle (0.0432cm);



    \path[fill,even odd rule] (6.4171, -0.9609).. controls (6.3316, -0.9609) and (6.2592, -0.9908) .. (6.1999, -1.0505).. controls (6.1391, -1.1122) and (6.1087, -1.1853) .. (6.1087, -1.2696).. controls (6.1087, -1.354) and (6.1391, -1.4266) .. (6.1999, -1.4873).. controls (6.2607, -1.548) and (6.3331, -1.5784) .. (6.4171, -1.5784).. controls (6.5021, -1.5784) and (6.5758, -1.5478) .. (6.6381, -1.4865).. controls (6.6968, -1.4284) and (6.7262, -1.3561) .. (6.7262, -1.2696).. controls (6.7262, -1.1832) and (6.6963, -1.1101) .. (6.6366, -1.0505).. controls (6.5768, -0.9908) and (6.5036, -0.9609) .. (6.4171, -0.9609) -- cycle(6.4178, -1.0165).. controls (6.4879, -1.0165) and (6.5474, -1.0412) .. (6.5964, -1.0906).. controls (6.6458, -1.1395) and (6.6706, -1.1992) .. (6.6706, -1.2696).. controls (6.6706, -1.3406) and (6.6464, -1.3996) .. (6.5979, -1.4464).. controls (6.5469, -1.4968) and (6.4869, -1.522) .. (6.4178, -1.522).. controls (6.3488, -1.522) and (6.2893, -1.4971) .. (6.2393, -1.4472).. controls (6.1893, -1.3973) and (6.1644, -1.3381) .. (6.1644, -1.2696).. controls (6.1644, -1.2012) and (6.1896, -1.1415) .. (6.2401, -1.0906).. controls (6.2885, -1.0412) and (6.3478, -1.0165) .. (6.4178, -1.0165) -- cycle;



  \end{scope}

\end{tikzpicture}
 \\ {\tt cc-by-sa}

        \vspace{3pt}
\definecolor{caab2ab}{RGB}{170,178,171}


\def \globalscale {0.550000}
\begin{tikzpicture}[y=1cm, x=1cm, yscale=\globalscale,xscale=\globalscale, every node/.append style={scale=\globalscale}, inner sep=0pt, outer sep=0pt]
  \begin{scope}[cm={ 0.9938,-0.0,-0.0,0.9937,(-4.7015, 0.007)}]
    \path[fill=caab2ab] (4.8145, 1.0977) -- (7.8325, 1.0924).. controls (7.8746, 1.0924) and (7.9123, 1.0986) .. (7.9123, 1.0082) -- (7.9086, 0.0143) -- (4.7384, 0.0143) -- (4.7384, 1.0119).. controls (4.7384, 1.0565) and (4.7427, 1.0977) .. (4.8145, 1.0977) -- cycle;



    \path[fill] (7.866, 1.1113) -- (4.7907, 1.1113).. controls (4.7578, 1.1113) and (4.7309, 1.0844) .. (4.7309, 1.0514) -- (4.7309, 0.0064).. controls (4.7309, -0.001) and (4.737, -0.0071) .. (4.7444, -0.0071) -- (7.9123, -0.0071).. controls (7.9197, -0.0071) and (7.9258, -0.001) .. (7.9258, 0.0064) -- (7.9258, 1.0514).. controls (7.9258, 1.0844) and (7.899, 1.1113) .. (7.866, 1.1113) -- cycle(4.7907, 1.0842) -- (7.866, 1.0842).. controls (7.8841, 1.0842) and (7.8988, 1.0695) .. (7.8988, 1.0514).. controls (7.8988, 1.0514) and (7.8988, 0.6295) .. (7.8988, 0.3252) -- (5.6936, 0.3252).. controls (5.6131, 0.1795) and (5.4579, 0.0806) .. (5.2797, 0.0806).. controls (5.1015, 0.0806) and (4.9464, 0.1794) .. (4.8658, 0.3252) -- (4.7579, 0.3252).. controls (4.7579, 0.6295) and (4.7579, 1.0514) .. (4.7579, 1.0514).. controls (4.7579, 1.0695) and (4.7727, 1.0842) .. (4.7907, 1.0842) -- cycle;



    \path[fill=white] (6.696, 0.2393).. controls (6.7044, 0.2393) and (6.7121, 0.2385) .. (6.719, 0.2371).. controls (6.726, 0.2356) and (6.7319, 0.2332) .. (6.7368, 0.2298).. controls (6.7418, 0.2264) and (6.7456, 0.2219) .. (6.7483, 0.2163).. controls (6.751, 0.2107) and (6.7524, 0.2038) .. (6.7524, 0.1956).. controls (6.7524, 0.1867) and (6.7504, 0.1793) .. (6.7463, 0.1734).. controls (6.7423, 0.1675) and (6.7363, 0.1626) .. (6.7284, 0.1588).. controls (6.7393, 0.1557) and (6.7475, 0.1502) .. (6.7528, 0.1424).. controls (6.7582, 0.1346) and (6.7609, 0.1251) .. (6.7609, 0.1141).. controls (6.7609, 0.1052) and (6.7592, 0.0975) .. (6.7557, 0.091).. controls (6.7522, 0.0845) and (6.7476, 0.0792) .. (6.7417, 0.0751).. controls (6.7359, 0.071) and (6.7292, 0.0679) .. (6.7217, 0.0659).. controls (6.7142, 0.0639) and (6.7065, 0.063) .. (6.6985, 0.063) -- (6.6129, 0.063) -- (6.6129, 0.2393) -- (6.696, 0.2393) -- (6.696, 0.2393) -- cycle(6.6911, 0.168).. controls (6.698, 0.168) and (6.7037, 0.1696) .. (6.7081, 0.1729).. controls (6.7126, 0.1762) and (6.7148, 0.1815) .. (6.7148, 0.1889).. controls (6.7148, 0.193) and (6.714, 0.1964) .. (6.7126, 0.199).. controls (6.7111, 0.2016) and (6.7091, 0.2037) .. (6.7066, 0.2052).. controls (6.7042, 0.2067) and (6.7013, 0.2077) .. (6.6981, 0.2083).. controls (6.6949, 0.2088) and (6.6915, 0.2091) .. (6.6881, 0.2091) -- (6.6518, 0.2091) -- (6.6518, 0.168) -- (6.6911, 0.168) -- cycle(6.6933, 0.0931).. controls (6.6971, 0.0931) and (6.7007, 0.0935) .. (6.7042, 0.0943).. controls (6.7077, 0.095) and (6.7107, 0.0962) .. (6.7134, 0.098).. controls (6.716, 0.0997) and (6.7181, 0.102) .. (6.7197, 0.105).. controls (6.7213, 0.108) and (6.722, 0.1118) .. (6.722, 0.1163).. controls (6.722, 0.1254) and (6.7195, 0.1319) .. (6.7144, 0.1357).. controls (6.7092, 0.1396) and (6.7025, 0.1415) .. (6.6941, 0.1415) -- (6.6518, 0.1415) -- (6.6518, 0.0931) -- (6.6933, 0.0931) -- cycle;



    \path[fill=white] (6.7677, 0.2393) -- (6.8112, 0.2393) -- (6.8525, 0.1697) -- (6.8935, 0.2393) -- (6.9368, 0.2393) -- (6.8713, 0.1306) -- (6.8713, 0.063) -- (6.8325, 0.063) -- (6.8325, 0.1316) -- (6.7677, 0.2393) -- cycle;



    \begin{scope}[cm={ 0.8729,-0.0,-0.0,0.8729,(1.3262, -3.648)}]
      \path[fill=white] (4.9451, 4.854).. controls (4.9452, 4.6244) and (4.7591, 4.4381) .. (4.5295, 4.4379).. controls (4.2998, 4.4378) and (4.1135, 4.6239) .. (4.1134, 4.8535).. controls (4.1134, 4.8537) and (4.1134, 4.8539) .. (4.1134, 4.854).. controls (4.1132, 5.0837) and (4.2993, 5.27) .. (4.529, 5.2701).. controls (4.7587, 5.2703) and (4.9449, 5.0842) .. (4.9451, 4.8545).. controls (4.9451, 4.8544) and (4.9451, 4.8542) .. (4.9451, 4.854) -- cycle;



      \begin{scope}[shift={(-7.6627, -2.6211)}]
        \path[fill] (12.53, 7.8139).. controls (12.6222, 7.7217) and (12.6684, 7.6087) .. (12.6684, 7.4751).. controls (12.6684, 7.3416) and (12.623, 7.2298) .. (12.5324, 7.14).. controls (12.4362, 7.0453) and (12.3225, 6.998) .. (12.1914, 6.998).. controls (12.0618, 6.998) and (11.9501, 7.045) .. (11.8563, 7.1388).. controls (11.7625, 7.2326) and (11.7156, 7.3447) .. (11.7156, 7.4751).. controls (11.7156, 7.6056) and (11.7625, 7.7185) .. (11.8563, 7.8139).. controls (11.9477, 7.9061) and (12.0594, 7.9522) .. (12.1914, 7.9522).. controls (12.3249, 7.9522) and (12.4378, 7.9061) .. (12.53, 7.8139) -- cycle(11.9184, 7.7518).. controls (11.8404, 7.6731) and (11.8014, 7.5808) .. (11.8014, 7.475).. controls (11.8014, 7.3692) and (11.84, 7.2778) .. (11.9172, 7.2006).. controls (11.9943, 7.1235) and (12.0862, 7.0849) .. (12.1928, 7.0849).. controls (12.2994, 7.0849) and (12.392, 7.1238) .. (12.4708, 7.2018).. controls (12.5455, 7.2742) and (12.5829, 7.3652) .. (12.5829, 7.475).. controls (12.5829, 7.584) and (12.5449, 7.6765) .. (12.469, 7.7524).. controls (12.393, 7.8284) and (12.301, 7.8664) .. (12.1928, 7.8664).. controls (12.0846, 7.8664) and (11.9931, 7.8282) .. (11.9184, 7.7518) -- cycle(12.1235, 7.5216).. controls (12.1116, 7.5475) and (12.0938, 7.5605) .. (12.07, 7.5605).. controls (12.028, 7.5605) and (12.0069, 7.5322) .. (12.0069, 7.4756).. controls (12.0069, 7.419) and (12.028, 7.3907) .. (12.07, 7.3907).. controls (12.0978, 7.3907) and (12.1176, 7.4045) .. (12.1295, 7.4321) -- (12.1878, 7.4011).. controls (12.16, 7.3517) and (12.1183, 7.327) .. (12.0627, 7.327).. controls (12.0199, 7.327) and (11.9856, 7.3402) .. (11.9598, 7.3664).. controls (11.9339, 7.3927) and (11.9211, 7.429) .. (11.9211, 7.4751).. controls (11.9211, 7.5205) and (11.9344, 7.5565) .. (11.961, 7.5832).. controls (11.9876, 7.6099) and (12.0207, 7.6233) .. (12.0604, 7.6233).. controls (12.1191, 7.6233) and (12.1612, 7.6001) .. (12.1866, 7.5539) -- (12.1235, 7.5216) -- cycle(12.3977, 7.5216).. controls (12.3858, 7.5475) and (12.3683, 7.5605) .. (12.3453, 7.5605).. controls (12.3024, 7.5605) and (12.2809, 7.5322) .. (12.2809, 7.4756).. controls (12.2809, 7.419) and (12.3024, 7.3907) .. (12.3453, 7.3907).. controls (12.3731, 7.3907) and (12.3926, 7.4045) .. (12.4037, 7.4321) -- (12.4633, 7.4011).. controls (12.4355, 7.3517) and (12.3939, 7.327) .. (12.3384, 7.327).. controls (12.2956, 7.327) and (12.2614, 7.3402) .. (12.2356, 7.3664).. controls (12.2099, 7.3927) and (12.197, 7.429) .. (12.197, 7.4751).. controls (12.197, 7.5205) and (12.2101, 7.5565) .. (12.2362, 7.5832).. controls (12.2624, 7.6099) and (12.2956, 7.6233) .. (12.3361, 7.6233).. controls (12.3947, 7.6233) and (12.4367, 7.6001) .. (12.462, 7.5539) -- (12.3977, 7.5216) -- cycle;



      \end{scope}
    \end{scope}
    \path[fill=white] (6.7615, 0.7061) circle (0.2859cm);



    \path[fill] (6.8442, 0.7888).. controls (6.8442, 0.7998) and (6.8353, 0.8087) .. (6.8243, 0.8087) -- (6.698, 0.8087).. controls (6.687, 0.8087) and (6.6781, 0.7998) .. (6.6781, 0.7888) -- (6.6781, 0.6625) -- (6.7133, 0.6625) -- (6.7133, 0.513) -- (6.809, 0.513) -- (6.809, 0.6625) -- (6.8442, 0.6625) -- (6.8442, 0.7888) -- (6.8442, 0.7888) -- cycle;



    \path[fill] (6.7611, 0.8686) circle (0.0432cm);



    \path[fill,even odd rule] (6.7607, 1.0211).. controls (6.6752, 1.0211) and (6.6028, 0.9913) .. (6.5436, 0.9316).. controls (6.4828, 0.8698) and (6.4524, 0.7967) .. (6.4524, 0.7124).. controls (6.4524, 0.628) and (6.4828, 0.5554) .. (6.5436, 0.4947).. controls (6.6044, 0.434) and (6.6768, 0.4036) .. (6.7607, 0.4036).. controls (6.8457, 0.4036) and (6.9194, 0.4342) .. (6.9818, 0.4955).. controls (7.0405, 0.5536) and (7.0699, 0.6259) .. (7.0699, 0.7124).. controls (7.0699, 0.7988) and (7.04, 0.8719) .. (6.9802, 0.9316).. controls (6.9204, 0.9913) and (6.8473, 1.0211) .. (6.7607, 1.0211) -- cycle(6.7615, 0.9655).. controls (6.8316, 0.9655) and (6.8911, 0.9408) .. (6.94, 0.8914).. controls (6.9895, 0.8426) and (7.0142, 0.7829) .. (7.0142, 0.7124).. controls (7.0142, 0.6414) and (6.99, 0.5824) .. (6.9416, 0.5356).. controls (6.8905, 0.4852) and (6.8306, 0.46) .. (6.7615, 0.46).. controls (6.6924, 0.46) and (6.6329, 0.4849) .. (6.583, 0.5349).. controls (6.533, 0.5848) and (6.508, 0.6439) .. (6.508, 0.7124).. controls (6.508, 0.7808) and (6.5333, 0.8405) .. (6.5838, 0.8914).. controls (6.6322, 0.9408) and (6.6914, 0.9655) .. (6.7615, 0.9655) -- cycle;



  \end{scope}

\end{tikzpicture}
 \\ {\tt cc-by}

        \column{0.5\textwidth}
        Specify licence by:
\begin{lstlisting}[language=TeX]
\def\licence{cc-by-nc-sa}
\end{lstlisting}

    or omit the command entirely if you do not want to specify one.

    \end{columns}

%    If you do not want to specify licence, do not \lstinline{\def\licence{}}.

\end{frame}

% -----------------------------------------------------------------------------

\begin{frame}[fragile]
    \frametitle{Sections}

    Sections have a title slide. To disable the title slide do:

    \begin{lstlisting}[language=TeX]
        \AtBeginSection{}
    \end{lstlisting}

    You can generate an outline of the slides by:

    \begin{lstlisting}[language=TeX]
        \outline{Title of your outline slide}
        \outlinecurrent{Outline with highlighted current section}
    \end{lstlisting}

\end{frame}


% -----------------------------------------------------------------------------

\begin{frame}[fragile]
    \frametitle{Equations}

    \centering Serif font also for equations

    $$i\hbar\frac{\partial}{\partial t} \Psi(\mathbf{r},t) = \left [ \frac{-\hbar^2}{2\mu}\nabla^2 + V(\mathbf{r},t)\right ] \Psi(\mathbf{r},t) ]$$

    \begin{lstlisting}[language=TeX]
$$ i\hbar\frac{\partial}{\partial t} \Psi(\mathbf{r},t) = \left [ \frac{-\hbar^2}{2\mu}\nabla^2 + V(\mathbf{r},t)\right ] \Psi(\mathbf{r},t) ]$$
    \end{lstlisting}

    Equations may break compilation with Mik\TeX. If this is your case use the
    package with option \lstinline[language=TeX]{miktex}.

\begin{lstlisting}[language=TeX]
    \usepackage[miktex]{ufallslides}
\end{lstlisting}

\end{frame}

% -----------------------------------------------------------------------------

\begin{frame}
    \frametitle{Color Palette}

    Named colors which are the same as the boostrap colors at ÚFAL web page.\\[15pt]

    \begin{center}
    \scalebox{1.2}{%
    \begin{tabular}{cc}

    \colorbox{ufal}{\bf ufal} &     \color{ufal} \bf ufal \\
    \colorbox{ufalblue}{\bf ufalblue} &     \color{ufalblue} \bf ufalblue \\
    \colorbox{ufalred}{\bf ufalred} &     \color{ufalred} \bf ufalred \\
    \colorbox{ufallightblue}{\bf ufallightblue} &     \color{ufallightblue} \bf ufallightblue \\
    \colorbox{ufalgreen}{\bf ufalgreen} &     \color{ufalgreen} \bf ufalgreen \\
    \end{tabular}}
    \end{center}

\end{frame}


% -----------------------------------------------------------------------------

\begin{frame}[fragile]
    \frametitle{Labels from the web}

    \slidesbox{Slides}
    \readingbox{Reading}
    \hwbox{Homework}
    \questionbox{Question}
    \timebox{1 h}
    \calendarbox{Oct 15}
    \pointsbox{100 points}

    \begin{lstlisting}[language=TeX]
\slidesbox{Slides}
\readingbox{Reading}
\hwbox{Homework}
\questionbox{Question}
\timebox{1 h}
\calendarbox{Oct 15}
\pointsbox{100 points}
\slidesbox{Slides}
    \end{lstlisting}
\end{frame}

% -----------------------------------------------------------------------------
\section{Another section}
% -----------------------------------------------------------------------------

\begin{frame}[fragile]
    \frametitle{Code listings}

    This code snippet:
    \begin{lstlisting}[language=Python]
print("Hello, ÚFAL.")
x = 3 + 5
    \end{lstlisting}
%
    is produced by putting the code between {\tt  \textbackslash
    begin\{lstlisting\}} and {\tt \textbackslash end\{lstlisting\}}

    Inline code (\lstinline[language=Python]{import numpy as np}) can be
    inserted with the \lstinline[language=TeX]{\lstinline} command.

    Do not forget to start the frame with \lstinline[language=TeX]{fragile}
    option to beginning of the frame.

\end{frame}

% -----------------------------------------------------------------------------

\begin{frame}[fragile]
    \frametitle{References}

    Full citation on slide: \\ {\tiny \bibentry{helcl2017neural}}

\begin{lstlisting}[language=TeX]
Full citation: \\ {\tiny \bibentry{helcl2017neural}}
\end{lstlisting}

    \citet{sennrich2016neural} uses attention \citep{bahdanau2015neural}.

    \begin{lstlisting}[language=TeX]
\citet{sennrich2016neural} uses attention \citep{bahdanau2015neural}.
    \end{lstlisting}

    \tiny
    \citet{snover2006study,lu2016knowing,tu2016modeling,feng2016improving,zhang2016recurrent,alkhouli2016alignment,graves2014neural,specia2016shared,elliott2016multi30k}

If you prefer managing bibliography on your own, use the package with option
    \lstinline[language=TeX]{custombibset}.

If you want bibliography to have its own section, use 
    \lstinline[language=TeX]!\section{References}! before the 
	\lstinline[language=TeX]{\references} command. 

\end{frame}

% -----------------------------------------------------------------------------

\begin{frame}[fragile]
    \frametitle{Summary, outline, references}

    The summary slide can be inserted by calling:

    \begin{lstlisting}[language=TeX]
\summary{Name of the summary slide}{%
    Content of the summary slide
}
    \end{lstlisting}

    Outline with optionally highlighted current section can be inserted by:
    \begin{lstlisting}[language=TeX]
\outline{Outline slide title}
\outlinecurrent{Whatever outline title you wish}
    \end{lstlisting}

    To show the references do:

    \begin{lstlisting}[language=TeX]
\references{pathToYourBibFile.bib}
    \end{lstlisting}

\end{frame}

% -----------------------------------------------------------------------------

\begin{frame}
    \frametitle{Itemize}

    \begin{itemize}[<+->]

        \item All human beings are born free and equal in dignity and rights.

        \item They are endowed with reason and conscience and should act
            towards one another in a spirit of brotherhood.

        \item Everyone is entitled to all the rights and freedoms set forth in
            this Declaration, without distinction of any kind, such as race,
            colour, sex, language, religion, political or other opinion,
            national or social origin, property, birth or other status.

    \end{itemize}

\end{frame}

% -----------------------------------------------------------------------------

\begin{frame}
    \frametitle{Enumerate}

    \begin{enumerate}[<+->]

        \item All human beings are born free and equal in dignity and rights.

        \item They are endowed with reason and conscience and should act
            towards one another in a spirit of brotherhood.

        \item Everyone is entitled to all the rights and freedoms set forth in
            this Declaration, without distinction of any kind, such as race,
            colour, sex, language, religion, political or other opinion,
            national or social origin, property, birth or other status.

    \end{enumerate}

\end{frame}

% -----------------------------------------------------------------------------

\begin{frame}[allowframebreaks]
    \frametitle{What happens with too much content?}

    Babakotia, an extinct genus of sloth lemurs, lived in the northern part of
    Madagascar. The name comes from the Malagasy word for the indri, to which
    all sloth lemurs are closely related. Its morphological traits show
    intermediate stages between the slow-moving smaller sloth lemurs and the
    suspensory large sloth lemurs, and suggest a close relationship between
    both groups and the extinct monkey lemurs. All sloth lemurs share many
    traits with living sloths, demonstrating convergent evolution. Babakotia
    had long forearms, curved digits, and highly mobile hip and ankle joints.
    It shared its range with other sloth lemurs, including Palaeopropithecus
    ingens and Mesopropithecus dolichobrachion. It was primarily a leaf-eater,
    though it also ate fruit and hard seeds. It is known only from subfossil
    remains and may have died out shortly after the arrival of humans on the
    island, but not enough radiocarbon dating has been done with this species
    to know for certain. Babakotia radofilai is the sole member of the genus
    Babakotia and belongs to the family Palaeopropithecidae, which includes
    three other genera of sloth lemurs: Palaeopropithecus, Archaeoindris, and
    Mesopropithecus. This family in turn belongs to the infraorder
    Lemuriformes, which includes all the Malagasy lemurs.[5][2]

\end{frame}


% -----------------------------------------------------------------------------

\summary{Summary}{%
    \begin{enumerate}

        \item This template is tremendous.

        \item If you don't use the template you will be very very sad.

        \item Believe me. It's tremendous.

    \end{enumerate}
}
% -----------------------------------------------------------------------------

%Use this if you want references as a section,
%i.e. in the ouline and with a sectionpage
%\section{References}

\references{references.bib}

\end{document}
